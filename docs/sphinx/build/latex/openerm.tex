%% Generated by Sphinx.
\def\sphinxdocclass{report}
\documentclass[a4paper,12pt,spanish]{sphinxmanual}
\ifdefined\pdfpxdimen
   \let\sphinxpxdimen\pdfpxdimen\else\newdimen\sphinxpxdimen
\fi \sphinxpxdimen=.75bp\relax

\PassOptionsToPackage{warn}{textcomp}
\usepackage[utf8]{inputenc}
\ifdefined\DeclareUnicodeCharacter
% support both utf8 and utf8x syntaxes
  \ifdefined\DeclareUnicodeCharacterAsOptional
    \def\sphinxDUC#1{\DeclareUnicodeCharacter{"#1}}
  \else
    \let\sphinxDUC\DeclareUnicodeCharacter
  \fi
  \sphinxDUC{00A0}{\nobreakspace}
  \sphinxDUC{2500}{\sphinxunichar{2500}}
  \sphinxDUC{2502}{\sphinxunichar{2502}}
  \sphinxDUC{2514}{\sphinxunichar{2514}}
  \sphinxDUC{251C}{\sphinxunichar{251C}}
  \sphinxDUC{2572}{\textbackslash}
\fi
\usepackage{cmap}
\usepackage[T1]{fontenc}
\usepackage{amsmath,amssymb,amstext}
\usepackage{babel}



	\PassOptionsToPackage{bookmarksnumbered}{hyperref}
	

\usepackage[Sonny]{fncychap}
\ChNameVar{\Large\normalfont\sffamily}
\ChTitleVar{\Large\normalfont\sffamily}
\usepackage{sphinx}

\fvset{fontsize=\small}
\usepackage{geometry}

% Include hyperref last.
\usepackage{hyperref}
% Fix anchor placement for figures with captions.
\usepackage{hypcap}% it must be loaded after hyperref.
% Set up styles of URL: it should be placed after hyperref.
\urlstyle{same}

\usepackage{sphinxmessages}
\setcounter{tocdepth}{1}


	\usepackage{setspace}
	

\title{openerm}
\date{08 de mayo de 2019}
\release{0.0.1}
\author{Patricio Moracho}
\newcommand{\sphinxlogo}{\vbox{}}
\renewcommand{\releasename}{Versión}
\makeindex
\begin{document}

\ifdefined\shorthandoff
  \ifnum\catcode`\=\string=\active\shorthandoff{=}\fi
  \ifnum\catcode`\"=\active\shorthandoff{"}\fi
\fi

\pagestyle{empty}

	\pagenumbering{arabic}
	
\pagestyle{plain}
\sphinxtableofcontents
\pagestyle{normal}
\phantomsection\label{\detokenize{index::doc}}


\sphinxstylestrong{OpenERM} es la primera especificación «abierta» para el almacenamiento de
reportes electrónicos. Las siglas \sphinxstyleemphasis{OERM} hacen referencia a \sphinxstyleemphasis{Open electronic
report management} una forma moderna de llamar lo que hace algunos años se
conocía como \sphinxstylestrong{{}`COLD{}`}, \sphinxstyleemphasis{Computer output to laser disk}. Asimismo es la primer
implementación oficial de dicha especificación.


\chapter{Un poco de historia}
\label{\detokenize{index:un-poco-de-historia}}
Desde los inicios y en todo tipo de apliciones informaticas, se han generado
enormes cantidades de informes, estos reportes terminaban su ciclo de vida en
el papel, informes de 80 o 132 columnas, de formato habitualmente tabular.
Millones de hojas de papel fueron impresas y distribuídas de esta forma en todo
tipo de empresa a lo largo y ancho del mundo. Sin embargo el acceso a la
información mediante el «papel», en poco tiempo hizo notar sus limitaciones.
Surge una teconolgía, que tuvo su esplendor en las decada del 80 y del 90, se
trata de lo que en ese entonces se habia bautizado como \sphinxstyleemphasis{«COLD»}, es decir
\sphinxstyleemphasis{«Computer output to laser disk»}. El concepto era simple, se «capturaba» la
salida hacia la impresora de los sistemas centralizados, normalmente
«Mainframes» o grandes computadores, y esta salida era guardada en archivos
electrónicos, almacenada, indizada y distribuída mediante discos ópticos (laser
disk), pudiendo luego ser visualizada mediante un software \sphinxstyleemphasis{«COLD»} en PCs u
otras mini computadoras.

Este «paradigma», un gran computador central, aplicaciones que
explotan la información en reportes de tipo texto, distribución final
en papel, tuvo una larga vida. Sin embargo los cambios en la
tecnología, el abaratamiento de los costos y otros factores han dejado
de lado este «modelo» por otros. Los sistemas «COLD» han ido langideciendo
de a poco, sin embargo sigue existiendo un «nicho» importante para este
tipo de herramientas: aún hoy existen empresas que apoyan la gestión
de su negocio en grande computadores o «Mainframes» y siguen generando
enormes cantidades de listados.


\chapter{Estatus del proyecto a abril 2019}
\label{\detokenize{index:estatus-del-proyecto-a-abril-2019}}
Esta es la situación actual del proyecto.
\begin{quote}

Definiciones:
\begin{itemize}
\item {} 
Estructura fisica dónde salvar los reportes

\end{itemize}

Funcionalidad:
\begin{itemize}
\item {} 
Varios algoritmos de compresión, ver: {\hyperref[\detokenize{openerm.Compressor:module-openerm.Compressor}]{\sphinxcrossref{\sphinxcode{\sphinxupquote{openerm.Compressor}}}}}

\item {} 
Cifrado (Spritz y Fernet), ver: {\hyperref[\detokenize{openerm.Cipher:module-openerm.Cipher}]{\sphinxcrossref{\sphinxcode{\sphinxupquote{openerm.Cipher}}}}}

\item {} 
Se implementó una clase para el guardado y recuperación de los reportes y sus páginas, ver: {\hyperref[\detokenize{openerm.Database:module-openerm.Database}]{\sphinxcrossref{\sphinxcode{\sphinxupquote{openerm.Database}}}}}

\end{itemize}

Herramientas:
\begin{itemize}
\item {} 
Spl2oerm - Procesador básico de spooles:
\begin{itemize}
\item {} 
Procesamiento de spooles ASCII/EBCDIC de registro de longitud fija, ver: {\hyperref[\detokenize{openerm.SpoolFixedRecordLength:module-openerm.SpoolFixedRecordLength}]{\sphinxcrossref{\sphinxcode{\sphinxupquote{openerm.SpoolFixedRecordLength}}}}}

\item {} 
Procesamiento de spooles ASCII/EBCDIC de registro de longitud variable con info de canal, ver: {\hyperref[\detokenize{openerm.SpoolHostReprint:module-openerm.SpoolHostReprint}]{\sphinxcrossref{\sphinxcode{\sphinxupquote{openerm.SpoolHostReprint}}}}}

\item {} 
Identificación de páginas, por texto de salto de página o info de canal

\item {} 
Identificación simple de reportes por texto encontrado en página

\item {} 
Configuración completa del proceso definido en archivo de configuración yaml

\item {} 
Salvado de los reportes en el Database final

\end{itemize}

\item {} 
readoermdb - Lectura de un database OERM:
\begin{itemize}
\item {} 
Lectura de un Database

\item {} 
Recuperación de reportes

\item {} 
Lectura de cualquier reporte

\item {} 
Extracción de páginas

\end{itemize}

\end{itemize}

To do:
\begin{itemize}
\item {} 
Mejorar la identificación de reportes

\item {} 
Mas opciones, multiples textos a ubicar

\item {} 
Configurar ventanas de búsquedas (cabeceras/footers)

\item {} 
Optimizar identificación mediante algún algoritmo mejorado

\item {} 
Paralelizar el proceso del spool, separando la lectura de páginas de la identificación de archivos y el salvado final

\item {} 
Captura de datos (fecha, sistema, aplicación, etc) desde el mismo reporte

\item {} 
Definir mejor los datos adicionales de los reportes. Hoy el único dato real que identifica un reporte el el nombre del mismo

\end{itemize}
\end{quote}


\chapter{Openerm - API}
\label{\detokenize{index:openerm-api}}
Estos son los módulos principales


\section{OERM - Especificación v1}
\label{\detokenize{especificacion:oerm-especificacion-v1}}\label{\detokenize{especificacion::doc}}

\section{Estructura de un database OpenErm}
\label{\detokenize{especificacion:estructura-de-un-database-openerm}}
Un Database OpenErm es un archivo que almacena reportes electrónicos de forma comprimida y/o
encriptada. Se representa físicamente por tres archivos básicos:
\begin{itemize}
\item {} 
\sphinxstylestrong{\textless{}database\textgreater{}}.oerm: (o «DATA») Es el archivo físico principal, es simplemente un contenedor
de bloques. Los bloques son conjuntos arbitrarios y variables de bytes. Los
bloques pueden ser de dos tipos
\begin{itemize}
\item {} 
Contenedor de metadatos

\item {} 
Contenedor de páginas

\end{itemize}

\item {} 
\sphinxstylestrong{\textless{}database\textgreater{}}.cidx.oerm: índice de bloques. Básicamente es una lista con los offsets o
posiciones físicas dónde comienza cada bloque del archivo principal.

\item {} 
\sphinxstylestrong{\textless{}database\textgreater{}}.ridx.oerm: índice de reportes. Es la lista de offsets o posiciones
físicas de los bloques contenedores de metadatos de cada uno de los reportes que se
almacenan en el archivo princiapl.

\end{itemize}

El archivo principal \sphinxstylestrong{\textless{}database\textgreater{}}.oerm o «DATA» contiene toda la información fundamental,
tanto el índice de bloques como el de reporte puede ser regenerado en cualquier momento a
partir de archivo «DATA»

\begin{sphinxVerbatim}[commandchars=\\\{\}]
Estructura de un \PYGZlt{}database\PYGZgt{}.oerm

+========+
\textbar{} \PYGZdq{}oerm\PYGZdq{} \textbar{}              \PYGZhy{}\PYGZhy{}\PYGZgt{} \PYGZdq{}Magic number\PYGZdq{} (4 bytes)
+===================+
\textbar{}     Bloque 1      \textbar{}   \PYGZhy{}\PYGZhy{}\PYGZgt{} Bytes (longitud variable)
+===================+
\textbar{}                   \textbar{}
\textbar{}     Bloque 2      \textbar{}   \PYGZhy{}\PYGZhy{}\PYGZgt{} Bytes (longitud variable)
\textbar{}                   \textbar{}
+===================+
..
+===================+
\textbar{}     Bloque  N     \textbar{}   \PYGZhy{}\PYGZhy{}\PYGZgt{} Bytes (longitud variable)
+===================+


Estructura de cualquier Bloque

+=======================+
\textbar{} Long.Total del Bloque \textbar{}   \PYGZhy{}\PYGZhy{}\PYGZgt{} long (4 bytes)
+=======================+
\textbar{} Tipo de Bloque  \textbar{}         \PYGZhy{}\PYGZhy{}\PYGZgt{} int (1 bytes)
+=================+
\textbar{} Alg. Compresión \textbar{}         \PYGZhy{}\PYGZhy{}\PYGZgt{} int (1 bytes)
+=================+
\textbar{} Alg. Cifrado    \textbar{}         \PYGZhy{}\PYGZhy{}\PYGZgt{} int (1 bytes)
+=======================+
\textbar{} Long. de los Datos    \textbar{}   \PYGZhy{}\PYGZhy{}\PYGZgt{} long (4 bytes)
+=======================+
\textbar{}                       \textbar{}
\textbar{}        Datos          \textbar{}   \PYGZhy{}\PYGZhy{}\PYGZgt{} Longitud variable (datos comprimibles)
\textbar{}                       \textbar{}
+=======================+
\textbar{}                       \textbar{}
\textbar{}    Datos variables    \textbar{}   \PYGZhy{}\PYGZhy{}\PYGZgt{} (Opcional) Longitud variable (datos NO comprimibles)
\textbar{}                       \textbar{}
+=======================+
\end{sphinxVerbatim}


\section{Openerm - API}
\label{\detokenize{openerm:openerm-api}}\label{\detokenize{openerm::doc}}
El API básica de OpenErm ofrece por un lado clases públicas, que debieran ser
las que utilicemos en nuestras aplicaciones, y otras que son de uso interno a las
que debieramos escaparles.


\subsection{Clases de uso público}
\label{\detokenize{openerm:clases-de-uso-publico}}\phantomsection\label{\detokenize{openerm.OermClient:oermclient}}\phantomsection\label{\detokenize{openerm.OermClient:module-openerm.OermClient}}\phantomsection\label{\detokenize{openerm.OermClient:oermclient}}\index{openerm.OermClient (módulo)@\spxentry{openerm.OermClient}\spxextra{módulo}}

\subsubsection{OermClient}
\label{\detokenize{openerm.OermClient:id1}}\label{\detokenize{openerm.OermClient::doc}}
Un \sphinxstylestrong{OermClient} es el objeto que permite conectarse a uno o más repositorios
de reportes. Es la clase «oficial» para acceder a los reportes, conceptualmente
un \sphinxstylestrong{OermClient} ofrece uno o más catálogos, los catálogos son organizaciones
lógicas de los reportes físicos


\sphinxstrong{Ver también:}

\begin{itemize}
\item {} 
{\hyperref[\detokenize{openerm.MetadataContainer:module-openerm.MetadataContainer}]{\sphinxcrossref{\sphinxcode{\sphinxupquote{openerm.MetadataContainer}}}}}

\item {} 
{\hyperref[\detokenize{openerm.PageContainer:module-openerm.PageContainer}]{\sphinxcrossref{\sphinxcode{\sphinxupquote{openerm.PageContainer}}}}}

\end{itemize}


\index{OermClient (clase en openerm.OermClient)@\spxentry{OermClient}\spxextra{clase en openerm.OermClient}}

\begin{fulllineitems}
\phantomsection\label{\detokenize{openerm.OermClient:openerm.OermClient.OermClient}}\pysiglinewithargsret{\sphinxbfcode{\sphinxupquote{class }}\sphinxcode{\sphinxupquote{openerm.OermClient.}}\sphinxbfcode{\sphinxupquote{OermClient}}}{\emph{configfile=None}}{}
Bases: \sphinxcode{\sphinxupquote{object}}

Clase Cliente para acceso a reportes Oerm. Los reportes OERM
se clasifican en: Catalogos y Repositorios. Un catalogo en
realidad representa un conjunto lógico de repositorios, los
repositorios representan carpetas físicas.
\begin{quote}\begin{description}
\item[{Parámetros}] \leavevmode
\sphinxstyleliteralstrong{\sphinxupquote{configfile}} (\sphinxstyleliteralemphasis{\sphinxupquote{string}}) \textendash{} Nombre del archivo de configuración (Formato YAML)

\end{description}\end{quote}
\index{add\_repo() (método de openerm.OermClient.OermClient)@\spxentry{add\_repo()}\spxextra{método de openerm.OermClient.OermClient}}

\begin{fulllineitems}
\phantomsection\label{\detokenize{openerm.OermClient:openerm.OermClient.OermClient.add_repo}}\pysiglinewithargsret{\sphinxbfcode{\sphinxupquote{add\_repo}}}{\emph{catalog\_id}, \emph{path}, \emph{update=False}}{}
Procesa el path de un repositorio de datbases Oerm y genera el
repo.db (sqlite). Basicamente cataloga cada database y genera una
base sqlite (repo.db) en el directorio root del repositorio.
\begin{quote}\begin{description}
\item[{Parámetros}] \leavevmode\begin{itemize}
\item {} 
\sphinxstyleliteralstrong{\sphinxupquote{catalog\_id}} (\sphinxstyleliteralemphasis{\sphinxupquote{string}}) \textendash{} Id del catálogo al cual se le agregará este repositorio

\item {} 
\sphinxstyleliteralstrong{\sphinxupquote{path}} (\sphinxstyleliteralemphasis{\sphinxupquote{string}}) \textendash{} Carpeta principal del repositorio

\item {} 
\sphinxstyleliteralstrong{\sphinxupquote{update}} (\sphinxstyleliteralemphasis{\sphinxupquote{bool}}) \textendash{} (Opcional) Se actualiza o regenera completamente el catalogo

\end{itemize}

\end{description}\end{quote}
\begin{description}
\item[{Ejemplo:}] \leavevmode
\begin{sphinxVerbatim}[commandchars=\\\{\}]
\PYG{g+gp}{\PYGZgt{}\PYGZgt{}\PYGZgt{} }\PYG{k+kn}{from} \PYG{n+nn}{openerm}\PYG{n+nn}{.}\PYG{n+nn}{OermClient} \PYG{k}{import} \PYG{n}{OermClient}
\PYG{g+gp}{\PYGZgt{}\PYGZgt{}\PYGZgt{} }\PYG{n}{c} \PYG{o}{=} \PYG{n}{OermClient}\PYG{p}{(}\PYG{l+s+s2}{\PYGZdq{}}\PYG{l+s+s2}{samples/openermcfg.yaml}\PYG{l+s+s2}{\PYGZdq{}}\PYG{p}{)}
\PYG{g+gp}{\PYGZgt{}\PYGZgt{}\PYGZgt{} }\PYG{n}{c}\PYG{o}{.}\PYG{n}{add\PYGZus{}repo}\PYG{p}{(}\PYG{l+s+s2}{\PYGZdq{}}\PYG{l+s+s2}{catalogo1}\PYG{l+s+s2}{\PYGZdq{}}\PYG{p}{,} \PYG{l+s+s2}{\PYGZdq{}}\PYG{l+s+s2}{/var/repo1}\PYG{l+s+s2}{\PYGZdq{}}\PYG{p}{)}
\end{sphinxVerbatim}

\end{description}

\end{fulllineitems}

\index{catalog\_create() (método de openerm.OermClient.OermClient)@\spxentry{catalog\_create()}\spxextra{método de openerm.OermClient.OermClient}}

\begin{fulllineitems}
\phantomsection\label{\detokenize{openerm.OermClient:openerm.OermClient.OermClient.catalog_create}}\pysiglinewithargsret{\sphinxbfcode{\sphinxupquote{catalog\_create}}}{\emph{catalogdict}}{}
Crear un catálogo (lógico) de repositorios Oerm.
\begin{quote}\begin{description}
\item[{Parámetros}] \leavevmode
\sphinxstyleliteralstrong{\sphinxupquote{catalogdict}} (\sphinxstyleliteralemphasis{\sphinxupquote{dict}}) \textendash{} Configuración del catálogo

\end{description}\end{quote}
\begin{description}
\item[{Raise:}] \leavevmode
ValueError si el catalogo \textless{}id\textgreater{} ya existe

\item[{Ejemplo:}] \leavevmode
\begin{sphinxVerbatim}[commandchars=\\\{\}]
\PYG{g+gp}{\PYGZgt{}\PYGZgt{}\PYGZgt{} }\PYG{k+kn}{from} \PYG{n+nn}{openerm}\PYG{n+nn}{.}\PYG{n+nn}{OermClient} \PYG{k}{import} \PYG{n}{OermClient}
\PYG{g+gp}{\PYGZgt{}\PYGZgt{}\PYGZgt{} }\PYG{n}{c} \PYG{o}{=} \PYG{n}{OermClient}\PYG{p}{(}\PYG{l+s+s2}{\PYGZdq{}}\PYG{l+s+s2}{samples/openermcfg.yaml}\PYG{l+s+s2}{\PYGZdq{}}\PYG{p}{)}
\PYG{g+gp}{\PYGZgt{}\PYGZgt{}\PYGZgt{} }\PYG{n}{catalog\PYGZus{}config} \PYG{o}{=} \PYG{p}{\PYGZob{}}\PYG{l+s+s2}{\PYGZdq{}}\PYG{l+s+s2}{catalogo1}\PYG{l+s+s2}{\PYGZdq{}}\PYG{p}{:} \PYG{p}{\PYGZob{}} \PYG{l+s+s2}{\PYGZdq{}}\PYG{l+s+s2}{name}\PYG{l+s+s2}{\PYGZdq{}}\PYG{p}{:} \PYG{l+s+s2}{\PYGZdq{}}\PYG{l+s+s2}{Ejemplo catalogo local}\PYG{l+s+s2}{\PYGZdq{}}\PYG{p}{,} \PYG{l+s+s2}{\PYGZdq{}}\PYG{l+s+s2}{type}\PYG{l+s+s2}{\PYGZdq{}}\PYG{p}{:} \PYG{l+s+s2}{\PYGZdq{}}\PYG{l+s+s2}{path}\PYG{l+s+s2}{\PYGZdq{}}\PYG{p}{,} \PYG{l+s+s2}{\PYGZdq{}}\PYG{l+s+s2}{enabled}\PYG{l+s+s2}{\PYGZdq{}}\PYG{p}{:} \PYG{k+kc}{True}\PYG{p}{,} \PYG{l+s+s2}{\PYGZdq{}}\PYG{l+s+s2}{url}\PYG{l+s+s2}{\PYGZdq{}}\PYG{p}{:} \PYG{l+s+s2}{\PYGZdq{}}\PYG{l+s+s2}{c:/oerm/}\PYG{l+s+s2}{\PYGZdq{}}\PYG{p}{\PYGZcb{}}\PYG{p}{\PYGZcb{}}
\PYG{g+gp}{\PYGZgt{}\PYGZgt{}\PYGZgt{} }\PYG{n}{c}\PYG{o}{.}\PYG{n}{catalog\PYGZus{}create}\PYG{p}{(}\PYG{n}{catalog\PYGZus{}config}\PYG{p}{)}
\end{sphinxVerbatim}

\end{description}

\end{fulllineitems}

\index{catalogs() (método de openerm.OermClient.OermClient)@\spxentry{catalogs()}\spxextra{método de openerm.OermClient.OermClient}}

\begin{fulllineitems}
\phantomsection\label{\detokenize{openerm.OermClient:openerm.OermClient.OermClient.catalogs}}\pysiglinewithargsret{\sphinxbfcode{\sphinxupquote{catalogs}}}{\emph{enabled=True}}{}
Lista los catalogos disponibles
\begin{quote}\begin{description}
\item[{Parámetros}] \leavevmode
\sphinxstyleliteralstrong{\sphinxupquote{enabled}} (\sphinxstyleliteralemphasis{\sphinxupquote{bool}}) \textendash{} (Opcional)      \sphinxstylestrong{True}: Solo los habilitados,
\sphinxstylestrong{False}: Los deshabilitados,
\sphinxstylestrong{None}: Todos

\item[{Devuelve}] \leavevmode
Lista de catalogos

\item[{Tipo del valor devuelto}] \leavevmode
List

\end{description}\end{quote}
\subsubsection*{Ejemplo}

\begin{sphinxVerbatim}[commandchars=\\\{\}]
\PYG{g+gp}{\PYGZgt{}\PYGZgt{}\PYGZgt{} }\PYG{k+kn}{from} \PYG{n+nn}{openerm}\PYG{n+nn}{.}\PYG{n+nn}{OermClient} \PYG{k}{import} \PYG{n}{OermClient}
\PYG{g+gp}{\PYGZgt{}\PYGZgt{}\PYGZgt{} }\PYG{n}{c} \PYG{o}{=} \PYG{n}{OermClient}\PYG{p}{(}\PYG{l+s+s2}{\PYGZdq{}}\PYG{l+s+s2}{samples/openermcfg.yaml}\PYG{l+s+s2}{\PYGZdq{}}\PYG{p}{)}
\PYG{g+gp}{\PYGZgt{}\PYGZgt{}\PYGZgt{} }\PYG{n}{c}\PYG{o}{.}\PYG{n}{catalogs}\PYG{p}{(}\PYG{n}{enabled}\PYG{o}{=}\PYG{k+kc}{None}\PYG{p}{)}
\PYG{g+gp}{\PYGZgt{}\PYGZgt{}\PYGZgt{} }\PYG{n+nb}{print}\PYG{p}{(}\PYG{l+s+s2}{\PYGZdq{}}\PYG{l+s+s2}{Catalogos disponibles: }\PYG{l+s+si}{\PYGZob{}0\PYGZcb{}}\PYG{l+s+s2}{\PYGZdq{}}\PYG{o}{.}\PYG{n}{format}\PYG{p}{(}\PYG{n}{c}\PYG{o}{.}\PYG{n}{catalogs}\PYG{p}{(}\PYG{n}{enabled}\PYG{o}{=}\PYG{k+kc}{None}\PYG{p}{)}\PYG{p}{)}\PYG{p}{)}
\PYG{g+go}{Catalogos disponibles: \PYGZob{}\PYGZsq{}sql\PYGZhy{}test\PYGZsq{}: \PYGZob{}\PYGZsq{}enabled\PYGZsq{}: False, \PYGZsq{}name\PYGZsq{}: \PYGZsq{}Ejemplo catalogo SQL\PYGZsq{}, \PYGZsq{}type\PYGZsq{}: \PYGZsq{}sql\PYGZsq{}\PYGZcb{}, \PYGZsq{}}
\PYG{g+go}{local\PYGZhy{}test\PYGZsq{}: \PYGZob{}\PYGZsq{}enabled\PYGZsq{}: True, \PYGZsq{}name\PYGZsq{}: \PYGZsq{}Ejemplo catalogo local\PYGZsq{}, \PYGZsq{}type\PYGZsq{}: \PYGZsq{}path\PYGZsq{}, \PYGZsq{}urls\PYGZsq{}:}
\PYG{g+go}{[\PYGZsq{}D:\PYGZbs{}pm\PYGZbs{}data\PYGZbs{}git.repo\PYGZbs{}openerm\PYGZbs{}samples\PYGZbs{}repo\PYGZsq{},}
\PYG{g+go}{\PYGZsq{}D:\PYGZbs{}pm\PYGZbs{}data\PYGZbs{}git.repo\PYGZbs{}openerm\PYGZbs{}samples\PYGZbs{}otro\PYGZsq{}\PYGZcb{}]\PYGZcb{}}
\end{sphinxVerbatim}

\end{fulllineitems}

\index{close\_catalog() (método de openerm.OermClient.OermClient)@\spxentry{close\_catalog()}\spxextra{método de openerm.OermClient.OermClient}}

\begin{fulllineitems}
\phantomsection\label{\detokenize{openerm.OermClient:openerm.OermClient.OermClient.close_catalog}}\pysiglinewithargsret{\sphinxbfcode{\sphinxupquote{close\_catalog}}}{}{}
Cierra el catalgo activo

\end{fulllineitems}

\index{current\_catalog() (método de openerm.OermClient.OermClient)@\spxentry{current\_catalog()}\spxextra{método de openerm.OermClient.OermClient}}

\begin{fulllineitems}
\phantomsection\label{\detokenize{openerm.OermClient:openerm.OermClient.OermClient.current_catalog}}\pysiglinewithargsret{\sphinxbfcode{\sphinxupquote{current\_catalog}}}{}{}
Retorna el catalogo activo
\begin{quote}\begin{description}
\item[{Devuelve}] \leavevmode
dict

\end{description}\end{quote}

\end{fulllineitems}

\index{open\_catalog() (método de openerm.OermClient.OermClient)@\spxentry{open\_catalog()}\spxextra{método de openerm.OermClient.OermClient}}

\begin{fulllineitems}
\phantomsection\label{\detokenize{openerm.OermClient:openerm.OermClient.OermClient.open_catalog}}\pysiglinewithargsret{\sphinxbfcode{\sphinxupquote{open\_catalog}}}{\emph{catalogid}}{}
Apertura de un catalogo
\begin{quote}\begin{description}
\item[{Parámetros}] \leavevmode
\sphinxstyleliteralstrong{\sphinxupquote{catalogid}} (\sphinxstyleliteralemphasis{\sphinxupquote{string}}) \textendash{} Id del catalogo que se desea abrir

\end{description}\end{quote}

Tipos de catalogos:
\begin{itemize}
\item {} 
Lista de paths + repo.db

\item {} 
Centralizado en Servidor SQL (TO DO)

\end{itemize}

\end{fulllineitems}

\index{open\_repo() (método de openerm.OermClient.OermClient)@\spxentry{open\_repo()}\spxextra{método de openerm.OermClient.OermClient}}

\begin{fulllineitems}
\phantomsection\label{\detokenize{openerm.OermClient:openerm.OermClient.OermClient.open_repo}}\pysiglinewithargsret{\sphinxbfcode{\sphinxupquote{open\_repo}}}{\emph{reponum}}{}
Abre un repositorio
\begin{quote}\begin{description}
\item[{Parámetros}] \leavevmode
\sphinxstyleliteralstrong{\sphinxupquote{repo}} (\sphinxstyleliteralemphasis{\sphinxupquote{int}}) \textendash{} Número del repositorio a abrir

\end{description}\end{quote}
\begin{description}
\item[{Raise:}] \leavevmode
ValueError: Si el repositorio no existe en el catalgo abierto

\end{description}

\end{fulllineitems}

\index{query\_reports() (método de openerm.OermClient.OermClient)@\spxentry{query\_reports()}\spxextra{método de openerm.OermClient.OermClient}}

\begin{fulllineitems}
\phantomsection\label{\detokenize{openerm.OermClient:openerm.OermClient.OermClient.query_reports}}\pysiglinewithargsret{\sphinxbfcode{\sphinxupquote{query\_reports}}}{\emph{reporte=None}, \emph{sistema=None}, \emph{aplicacion=None}, \emph{departamento=None}, \emph{fecha=None}, \emph{limit=None}, \emph{returntype='list'}}{}
Consulta básica para buscar un reporte en los repositorios del catalogo.
La busqueda se hace por cualquier de los atributos básicos de un reporte, y
se puede hacer búsquedas parciales tipo LIKE en sql
\begin{quote}\begin{description}
\item[{Parámetros}] \leavevmode\begin{itemize}
\item {} 
\sphinxstyleliteralstrong{\sphinxupquote{report}} (\sphinxstyleliteralemphasis{\sphinxupquote{string}}) \textendash{} Nombre del reporte

\item {} 
\sphinxstyleliteralstrong{\sphinxupquote{sistema}} (\sphinxstyleliteralemphasis{\sphinxupquote{string}}) \textendash{} Nombre del sistema

\item {} 
\sphinxstyleliteralstrong{\sphinxupquote{aplicacion}} (\sphinxstyleliteralemphasis{\sphinxupquote{string}}) \textendash{} Nombre de la aplicación

\item {} 
\sphinxstyleliteralstrong{\sphinxupquote{departamento}} (\sphinxstyleliteralemphasis{\sphinxupquote{string}}) \textendash{} Nombre del departamento

\item {} 
\sphinxstyleliteralstrong{\sphinxupquote{fecha}} (\sphinxstyleliteralemphasis{\sphinxupquote{string}}) \textendash{} Fecha de emsión del reporte

\item {} 
\sphinxstyleliteralstrong{\sphinxupquote{limit}} (\sphinxstyleliteralemphasis{\sphinxupquote{int}}) \textendash{} cantidad máxima de resultados

\item {} 
\sphinxstyleliteralstrong{\sphinxupquote{returntype}} (\sphinxstyleliteralemphasis{\sphinxupquote{string}}) \textendash{} Tipo de retorno

\end{itemize}

\item[{Devuelve}] \leavevmode
list/string

\end{description}\end{quote}
\begin{description}
\item[{Ejemplo:}] \leavevmode
\begin{sphinxVerbatim}[commandchars=\\\{\}]
\PYG{g+gp}{\PYGZgt{}\PYGZgt{}\PYGZgt{} }\PYG{k+kn}{from} \PYG{n+nn}{openerm}\PYG{n+nn}{.}\PYG{n+nn}{OermClient} \PYG{k}{import} \PYG{n}{OermClient}
\PYG{g+gp}{\PYGZgt{}\PYGZgt{}\PYGZgt{} }\PYG{n}{c} \PYG{o}{=} \PYG{n}{OermClient}\PYG{p}{(}\PYG{l+s+s2}{\PYGZdq{}}\PYG{l+s+s2}{samples/openermcfg.yaml}\PYG{l+s+s2}{\PYGZdq{}}\PYG{p}{)}
\PYG{g+gp}{\PYGZgt{}\PYGZgt{}\PYGZgt{} }\PYG{n}{c}\PYG{o}{.}\PYG{n}{open\PYGZus{}catalog}\PYG{p}{(}\PYG{l+s+s2}{\PYGZdq{}}\PYG{l+s+s2}{local\PYGZhy{}test}\PYG{l+s+s2}{\PYGZdq{}}\PYG{p}{)}
\PYG{g+gp}{\PYGZgt{}\PYGZgt{}\PYGZgt{} }\PYG{n}{c}\PYG{o}{.}\PYG{n}{open\PYGZus{}repo}\PYG{p}{(}\PYG{l+s+s2}{\PYGZdq{}}\PYG{l+s+s2}{Prueba1}\PYG{l+s+s2}{\PYGZdq{}}\PYG{p}{)}
\PYG{g+gp}{\PYGZgt{}\PYGZgt{}\PYGZgt{} }\PYG{n}{resultados} \PYG{o}{=} \PYG{n}{c}\PYG{o}{.}\PYG{n}{query\PYGZus{}reports}\PYG{p}{(}\PYG{n}{reporte}\PYG{o}{=}\PYG{l+s+s2}{\PYGZdq{}}\PYG{l+s+s2}{Carta}\PYG{l+s+s2}{\PYGZdq{}}\PYG{p}{,} \PYG{n}{returntype}\PYG{o}{=}\PYG{l+s+s2}{\PYGZdq{}}\PYG{l+s+s2}{tablestr}\PYG{l+s+s2}{\PYGZdq{}}\PYG{p}{)}
\PYG{g+gp}{\PYGZgt{}\PYGZgt{}\PYGZgt{} }\PYG{n+nb}{print}\PYG{p}{(}\PYG{n}{resultados}\PYG{p}{)}
\PYG{g+go}{+\PYGZhy{}\PYGZhy{}\PYGZhy{}\PYGZhy{}\PYGZhy{}\PYGZhy{}\PYGZhy{}\PYGZhy{}\PYGZhy{}\PYGZhy{}\PYGZhy{}\PYGZhy{}\PYGZhy{}\PYGZhy{}\PYGZhy{}\PYGZhy{}\PYGZhy{}\PYGZhy{}\PYGZhy{}\PYGZhy{}\PYGZhy{}\PYGZhy{}\PYGZhy{}\PYGZhy{}\PYGZhy{}\PYGZhy{}\PYGZhy{}\PYGZhy{}\PYGZhy{}\PYGZhy{}\PYGZhy{}\PYGZhy{}\PYGZhy{}\PYGZhy{}\PYGZhy{}\PYGZhy{}\PYGZhy{}\PYGZhy{}\PYGZhy{}\PYGZhy{}\PYGZhy{}\PYGZhy{}\PYGZhy{}\PYGZhy{}\PYGZhy{}\PYGZhy{}\PYGZhy{}\PYGZhy{}\PYGZhy{}\PYGZhy{}\PYGZhy{}\PYGZhy{}+\PYGZhy{}\PYGZhy{}\PYGZhy{}\PYGZhy{}\PYGZhy{}\PYGZhy{}\PYGZhy{}\PYGZhy{}\PYGZhy{}\PYGZhy{}+\PYGZhy{}\PYGZhy{}\PYGZhy{}\PYGZhy{}\PYGZhy{}\PYGZhy{}\PYGZhy{}\PYGZhy{}\PYGZhy{}\PYGZhy{}\PYGZhy{}\PYGZhy{}\PYGZhy{}\PYGZhy{}\PYGZhy{}\PYGZhy{}+\PYGZhy{}\PYGZhy{}\PYGZhy{}\PYGZhy{}\PYGZhy{}\PYGZhy{}\PYGZhy{}\PYGZhy{}\PYGZhy{}\PYGZhy{}\PYGZhy{}+\PYGZhy{}\PYGZhy{}\PYGZhy{}\PYGZhy{}\PYGZhy{}\PYGZhy{}\PYGZhy{}\PYGZhy{}\PYGZhy{}\PYGZhy{}\PYGZhy{}\PYGZhy{}\PYGZhy{}\PYGZhy{}+\PYGZhy{}\PYGZhy{}\PYGZhy{}\PYGZhy{}\PYGZhy{}\PYGZhy{}\PYGZhy{}\PYGZhy{}\PYGZhy{}\PYGZhy{}\PYGZhy{}+\PYGZhy{}\PYGZhy{}\PYGZhy{}\PYGZhy{}\PYGZhy{}\PYGZhy{}\PYGZhy{}\PYGZhy{}\PYGZhy{}\PYGZhy{}\PYGZhy{}\PYGZhy{}\PYGZhy{}\PYGZhy{}\PYGZhy{}\PYGZhy{}\PYGZhy{}\PYGZhy{}\PYGZhy{}\PYGZhy{}\PYGZhy{}+}
\PYG{g+go}{\textbar{} Nombre                                             \textbar{}    Fecha \textbar{} Departamento   \textbar{} Sistema   \textbar{} Aplicación   \textbar{}   Páginas \textbar{} Path                \textbar{}}
\PYG{g+go}{\textbar{}\PYGZhy{}\PYGZhy{}\PYGZhy{}\PYGZhy{}\PYGZhy{}\PYGZhy{}\PYGZhy{}\PYGZhy{}\PYGZhy{}\PYGZhy{}\PYGZhy{}\PYGZhy{}\PYGZhy{}\PYGZhy{}\PYGZhy{}\PYGZhy{}\PYGZhy{}\PYGZhy{}\PYGZhy{}\PYGZhy{}\PYGZhy{}\PYGZhy{}\PYGZhy{}\PYGZhy{}\PYGZhy{}\PYGZhy{}\PYGZhy{}\PYGZhy{}\PYGZhy{}\PYGZhy{}\PYGZhy{}\PYGZhy{}\PYGZhy{}\PYGZhy{}\PYGZhy{}\PYGZhy{}\PYGZhy{}\PYGZhy{}\PYGZhy{}\PYGZhy{}\PYGZhy{}\PYGZhy{}\PYGZhy{}\PYGZhy{}\PYGZhy{}\PYGZhy{}\PYGZhy{}\PYGZhy{}\PYGZhy{}\PYGZhy{}\PYGZhy{}\PYGZhy{}+\PYGZhy{}\PYGZhy{}\PYGZhy{}\PYGZhy{}\PYGZhy{}\PYGZhy{}\PYGZhy{}\PYGZhy{}\PYGZhy{}\PYGZhy{}+\PYGZhy{}\PYGZhy{}\PYGZhy{}\PYGZhy{}\PYGZhy{}\PYGZhy{}\PYGZhy{}\PYGZhy{}\PYGZhy{}\PYGZhy{}\PYGZhy{}\PYGZhy{}\PYGZhy{}\PYGZhy{}\PYGZhy{}\PYGZhy{}+\PYGZhy{}\PYGZhy{}\PYGZhy{}\PYGZhy{}\PYGZhy{}\PYGZhy{}\PYGZhy{}\PYGZhy{}\PYGZhy{}\PYGZhy{}\PYGZhy{}+\PYGZhy{}\PYGZhy{}\PYGZhy{}\PYGZhy{}\PYGZhy{}\PYGZhy{}\PYGZhy{}\PYGZhy{}\PYGZhy{}\PYGZhy{}\PYGZhy{}\PYGZhy{}\PYGZhy{}\PYGZhy{}+\PYGZhy{}\PYGZhy{}\PYGZhy{}\PYGZhy{}\PYGZhy{}\PYGZhy{}\PYGZhy{}\PYGZhy{}\PYGZhy{}\PYGZhy{}\PYGZhy{}+\PYGZhy{}\PYGZhy{}\PYGZhy{}\PYGZhy{}\PYGZhy{}\PYGZhy{}\PYGZhy{}\PYGZhy{}\PYGZhy{}\PYGZhy{}\PYGZhy{}\PYGZhy{}\PYGZhy{}\PYGZhy{}\PYGZhy{}\PYGZhy{}\PYGZhy{}\PYGZhy{}\PYGZhy{}\PYGZhy{}\PYGZhy{}\textbar{}}
\PYG{g+go}{\textbar{} R8101614 \PYGZhy{} Cartas fianza activas por moneda y clas \textbar{} 20160923 \textbar{} n/a            \textbar{} n/a       \textbar{} n/a          \textbar{}         1 \textbar{} test1\PYGZbs{}database.oerm \textbar{}}
\PYG{g+go}{\textbar{} R8101614 \PYGZhy{} Cartas fianza activas por moneda y clas \textbar{} 20160923 \textbar{} n/a            \textbar{} n/a       \textbar{} n/a          \textbar{}        20 \textbar{} test1\PYGZbs{}database.oerm \textbar{}}
\PYG{g+go}{\textbar{} R8101611 \PYGZhy{} Cartas fianza requeridas por funcionari \textbar{} 20160923 \textbar{} n/a            \textbar{} n/a       \textbar{} n/a          \textbar{}         1 \textbar{} test1\PYGZbs{}database.oerm \textbar{}}
\PYG{g+go}{+\PYGZhy{}\PYGZhy{}\PYGZhy{}\PYGZhy{}\PYGZhy{}\PYGZhy{}\PYGZhy{}\PYGZhy{}\PYGZhy{}\PYGZhy{}\PYGZhy{}\PYGZhy{}\PYGZhy{}\PYGZhy{}\PYGZhy{}\PYGZhy{}\PYGZhy{}\PYGZhy{}\PYGZhy{}\PYGZhy{}\PYGZhy{}\PYGZhy{}\PYGZhy{}\PYGZhy{}\PYGZhy{}\PYGZhy{}\PYGZhy{}\PYGZhy{}\PYGZhy{}\PYGZhy{}\PYGZhy{}\PYGZhy{}\PYGZhy{}\PYGZhy{}\PYGZhy{}\PYGZhy{}\PYGZhy{}\PYGZhy{}\PYGZhy{}\PYGZhy{}\PYGZhy{}\PYGZhy{}\PYGZhy{}\PYGZhy{}\PYGZhy{}\PYGZhy{}\PYGZhy{}\PYGZhy{}\PYGZhy{}\PYGZhy{}\PYGZhy{}\PYGZhy{}+\PYGZhy{}\PYGZhy{}\PYGZhy{}\PYGZhy{}\PYGZhy{}\PYGZhy{}\PYGZhy{}\PYGZhy{}\PYGZhy{}\PYGZhy{}+\PYGZhy{}\PYGZhy{}\PYGZhy{}\PYGZhy{}\PYGZhy{}\PYGZhy{}\PYGZhy{}\PYGZhy{}\PYGZhy{}\PYGZhy{}\PYGZhy{}\PYGZhy{}\PYGZhy{}\PYGZhy{}\PYGZhy{}\PYGZhy{}+\PYGZhy{}\PYGZhy{}\PYGZhy{}\PYGZhy{}\PYGZhy{}\PYGZhy{}\PYGZhy{}\PYGZhy{}\PYGZhy{}\PYGZhy{}\PYGZhy{}+\PYGZhy{}\PYGZhy{}\PYGZhy{}\PYGZhy{}\PYGZhy{}\PYGZhy{}\PYGZhy{}\PYGZhy{}\PYGZhy{}\PYGZhy{}\PYGZhy{}\PYGZhy{}\PYGZhy{}\PYGZhy{}+\PYGZhy{}\PYGZhy{}\PYGZhy{}\PYGZhy{}\PYGZhy{}\PYGZhy{}\PYGZhy{}\PYGZhy{}\PYGZhy{}\PYGZhy{}\PYGZhy{}+\PYGZhy{}\PYGZhy{}\PYGZhy{}\PYGZhy{}\PYGZhy{}\PYGZhy{}\PYGZhy{}\PYGZhy{}\PYGZhy{}\PYGZhy{}\PYGZhy{}\PYGZhy{}\PYGZhy{}\PYGZhy{}\PYGZhy{}\PYGZhy{}\PYGZhy{}\PYGZhy{}\PYGZhy{}\PYGZhy{}\PYGZhy{}+}
\end{sphinxVerbatim}

\end{description}

\end{fulllineitems}

\index{reports() (método de openerm.OermClient.OermClient)@\spxentry{reports()}\spxextra{método de openerm.OermClient.OermClient}}

\begin{fulllineitems}
\phantomsection\label{\detokenize{openerm.OermClient:openerm.OermClient.OermClient.reports}}\pysiglinewithargsret{\sphinxbfcode{\sphinxupquote{reports}}}{}{}
Retorna la lista completa de reportes del repositorio activo
\begin{quote}\begin{description}
\item[{Devuelve}] \leavevmode
list

\end{description}\end{quote}
\begin{description}
\item[{Ejemplo:}] \leavevmode
\begin{sphinxVerbatim}[commandchars=\\\{\}]
\PYG{g+gp}{\PYGZgt{}\PYGZgt{}\PYGZgt{} }\PYG{k+kn}{from} \PYG{n+nn}{openerm}\PYG{n+nn}{.}\PYG{n+nn}{OermClient} \PYG{k}{import} \PYG{n}{OermClient}
\PYG{g+gp}{\PYGZgt{}\PYGZgt{}\PYGZgt{} }\PYG{n}{c} \PYG{o}{=} \PYG{n}{OermClient}\PYG{p}{(}\PYG{l+s+s2}{\PYGZdq{}}\PYG{l+s+s2}{samples/openermcfg.yaml}\PYG{l+s+s2}{\PYGZdq{}}\PYG{p}{)}
\PYG{g+gp}{\PYGZgt{}\PYGZgt{}\PYGZgt{} }\PYG{n}{c}\PYG{o}{.}\PYG{n}{open\PYGZus{}catalog}\PYG{p}{(}\PYG{l+s+s2}{\PYGZdq{}}\PYG{l+s+s2}{local\PYGZhy{}test}\PYG{l+s+s2}{\PYGZdq{}}\PYG{p}{)}
\PYG{g+gp}{\PYGZgt{}\PYGZgt{}\PYGZgt{} }\PYG{n}{c}\PYG{o}{.}\PYG{n}{open\PYGZus{}repo}\PYG{p}{(}\PYG{l+s+s2}{\PYGZdq{}}\PYG{l+s+s2}{Prueba1}\PYG{l+s+s2}{\PYGZdq{}}\PYG{p}{)}
\PYG{g+gp}{\PYGZgt{}\PYGZgt{}\PYGZgt{} }\PYG{n+nb}{print}\PYG{p}{(}\PYG{n}{c}\PYG{o}{.}\PYG{n}{reports}\PYG{p}{(}\PYG{p}{)}\PYG{p}{)}
\end{sphinxVerbatim}

\end{description}

\end{fulllineitems}

\index{repos() (método de openerm.OermClient.OermClient)@\spxentry{repos()}\spxextra{método de openerm.OermClient.OermClient}}

\begin{fulllineitems}
\phantomsection\label{\detokenize{openerm.OermClient:openerm.OermClient.OermClient.repos}}\pysiglinewithargsret{\sphinxbfcode{\sphinxupquote{repos}}}{}{}
Lista los repositorios disponibles del catalgo activo
\begin{quote}\begin{description}
\item[{Devuelve}] \leavevmode
Lista de repositorios

\item[{Tipo del valor devuelto}] \leavevmode
List

\end{description}\end{quote}
\subsubsection*{Ejemplo}

\begin{sphinxVerbatim}[commandchars=\\\{\}]
\PYG{g+gp}{\PYGZgt{}\PYGZgt{}\PYGZgt{} }\PYG{k+kn}{from} \PYG{n+nn}{openerm}\PYG{n+nn}{.}\PYG{n+nn}{OermClient} \PYG{k}{import} \PYG{n}{OermClient}
\PYG{g+gp}{\PYGZgt{}\PYGZgt{}\PYGZgt{} }\PYG{n}{c} \PYG{o}{=} \PYG{n}{OermClient}\PYG{p}{(}\PYG{l+s+s2}{\PYGZdq{}}\PYG{l+s+s2}{samples/openermcfg.yaml}\PYG{l+s+s2}{\PYGZdq{}}\PYG{p}{)}
\PYG{g+gp}{\PYGZgt{}\PYGZgt{}\PYGZgt{} }\PYG{n}{c}\PYG{o}{.}\PYG{n}{open\PYGZus{}catalog}\PYG{p}{(}\PYG{l+s+s2}{\PYGZdq{}}\PYG{l+s+s2}{local\PYGZhy{}test}\PYG{l+s+s2}{\PYGZdq{}}\PYG{p}{)}
\PYG{g+gp}{\PYGZgt{}\PYGZgt{}\PYGZgt{} }\PYG{n+nb}{print}\PYG{p}{(}\PYG{l+s+s2}{\PYGZdq{}}\PYG{l+s+s2}{Repositorios disponibles: }\PYG{l+s+si}{\PYGZob{}0\PYGZcb{}}\PYG{l+s+s2}{\PYGZdq{}}\PYG{o}{.}\PYG{n}{format}\PYG{p}{(}\PYG{n}{c}\PYG{o}{.}\PYG{n}{repos}\PYG{p}{(}\PYG{p}{)}\PYG{p}{)}\PYG{p}{)}
\PYG{g+go}{Repositorios disponibles: \PYGZob{}\PYGZsq{}Prueba2\PYGZsq{}: \PYGZsq{}D:\PYGZbs{}pm\PYGZbs{}data\PYGZbs{}git.repo\PYGZbs{}openerm\PYGZbs{}samples\PYGZbs{}otro\PYGZbs{}repo.db\PYGZsq{},}
\PYG{g+go}{\PYGZsq{}Prueba1\PYGZsq{}: \PYGZsq{}D:\PYGZbs{}pm\PYGZbs{}data\PYGZbs{}git.repo\PYGZbs{}openerm\PYGZbs{}samples\PYGZbs{}repo\PYGZbs{}repo.db\PYGZsq{}\PYGZcb{}}
\end{sphinxVerbatim}

\end{fulllineitems}

\index{systems() (método de openerm.OermClient.OermClient)@\spxentry{systems()}\spxextra{método de openerm.OermClient.OermClient}}

\begin{fulllineitems}
\phantomsection\label{\detokenize{openerm.OermClient:openerm.OermClient.OermClient.systems}}\pysiglinewithargsret{\sphinxbfcode{\sphinxupquote{systems}}}{}{}
Retorna la lista completa de sistemas del repositorio activo
\begin{quote}\begin{description}
\item[{Devuelve}] \leavevmode
list

\end{description}\end{quote}
\begin{description}
\item[{Ejemplo:}] \leavevmode
\begin{sphinxVerbatim}[commandchars=\\\{\}]
\PYG{g+gp}{\PYGZgt{}\PYGZgt{}\PYGZgt{} }\PYG{k+kn}{from} \PYG{n+nn}{openerm}\PYG{n+nn}{.}\PYG{n+nn}{OermClient} \PYG{k}{import} \PYG{n}{OermClient}
\PYG{g+gp}{\PYGZgt{}\PYGZgt{}\PYGZgt{} }\PYG{n}{c} \PYG{o}{=} \PYG{n}{OermClient}\PYG{p}{(}\PYG{l+s+s2}{\PYGZdq{}}\PYG{l+s+s2}{samples/openermcfg.yaml}\PYG{l+s+s2}{\PYGZdq{}}\PYG{p}{)}
\PYG{g+gp}{\PYGZgt{}\PYGZgt{}\PYGZgt{} }\PYG{n}{c}\PYG{o}{.}\PYG{n}{open\PYGZus{}catalog}\PYG{p}{(}\PYG{l+s+s2}{\PYGZdq{}}\PYG{l+s+s2}{local\PYGZhy{}test}\PYG{l+s+s2}{\PYGZdq{}}\PYG{p}{)}
\PYG{g+gp}{\PYGZgt{}\PYGZgt{}\PYGZgt{} }\PYG{n}{c}\PYG{o}{.}\PYG{n}{open\PYGZus{}repo}\PYG{p}{(}\PYG{l+s+s2}{\PYGZdq{}}\PYG{l+s+s2}{Prueba1}\PYG{l+s+s2}{\PYGZdq{}}\PYG{p}{)}
\PYG{g+gp}{\PYGZgt{}\PYGZgt{}\PYGZgt{} }\PYG{n+nb}{print}\PYG{p}{(}\PYG{n}{c}\PYG{o}{.}\PYG{n}{systems}\PYG{p}{(}\PYG{p}{)}\PYG{p}{)}
\end{sphinxVerbatim}

\end{description}

\end{fulllineitems}


\end{fulllineitems}

\phantomsection\label{\detokenize{openerm.Database:database}}\phantomsection\label{\detokenize{openerm.Database:module-openerm.Database}}\phantomsection\label{\detokenize{openerm.Database:database}}\index{openerm.Database (módulo)@\spxentry{openerm.Database}\spxextra{módulo}}

\subsubsection{Database}
\label{\detokenize{openerm.Database:id1}}\label{\detokenize{openerm.Database::doc}}
Esta clase representa un objeto \sphinxstylestrong{Database} de reportes OERM. Basicamente es
el objeto que permite salvar y recuperar reportes electrónicos desde sus
archivos físicos.


\sphinxstrong{Ver también:}

\begin{itemize}
\item {} 
{\hyperref[\detokenize{openerm.MetadataContainer:module-openerm.MetadataContainer}]{\sphinxcrossref{\sphinxcode{\sphinxupquote{openerm.MetadataContainer}}}}}

\item {} 
{\hyperref[\detokenize{openerm.PageContainer:module-openerm.PageContainer}]{\sphinxcrossref{\sphinxcode{\sphinxupquote{openerm.PageContainer}}}}}

\end{itemize}


\index{Database (clase en openerm.Database)@\spxentry{Database}\spxextra{clase en openerm.Database}}

\begin{fulllineitems}
\phantomsection\label{\detokenize{openerm.Database:openerm.Database.Database}}\pysiglinewithargsret{\sphinxbfcode{\sphinxupquote{class }}\sphinxcode{\sphinxupquote{openerm.Database.}}\sphinxbfcode{\sphinxupquote{Database}}}{\emph{file='prueba.oerm'}, \emph{mode='rb'}, \emph{default\_compress\_method=1}, \emph{default\_compress\_level=1}, \emph{default\_encription\_method=0}, \emph{pages\_in\_container=10}}{}
Bases: \sphinxcode{\sphinxupquote{object}}

Clase base para el manejo de un contenedor de reportes OERM

Esta clase representa un contenedor de reportes OERM se usa
para la lectura y escritura de este tipo de datos.
\begin{quote}\begin{description}
\item[{Parámetros}] \leavevmode\begin{itemize}
\item {} 
\sphinxstyleliteralstrong{\sphinxupquote{file}} (\sphinxstyleliteralemphasis{\sphinxupquote{string}}) \textendash{} Nombre del archivo físicos

\item {} 
\sphinxstyleliteralstrong{\sphinxupquote{mode}} (\sphinxstyleliteralemphasis{\sphinxupquote{string}}) \textendash{} Modo  “wb”, “ab” o “rb” (Default: «rb»)

\item {} 
\sphinxstyleliteralstrong{\sphinxupquote{default\_compress\_method}} (\sphinxstyleliteralemphasis{\sphinxupquote{int}}) \textendash{} Algoritmo default de compresion

\item {} 
\sphinxstyleliteralstrong{\sphinxupquote{default\_compress\_level}} (\sphinxstyleliteralemphasis{\sphinxupquote{int}}) \textendash{} Nivel de compresión 0=mínimo, 1=normal, 2=máximo. Por defecto: 1.

\item {} 
\sphinxstyleliteralstrong{\sphinxupquote{default\_encription\_method}} (\sphinxstyleliteralemphasis{\sphinxupquote{int}}) \textendash{} Algoritmo de encriptación

\item {} 
\sphinxstyleliteralstrong{\sphinxupquote{pages\_in\_container}} (\sphinxstyleliteralemphasis{\sphinxupquote{int}}) \textendash{} Cantidad de páginas por contenedor

\end{itemize}

\end{description}\end{quote}
\subsubsection*{Ejemplo}

\begin{sphinxVerbatim}[commandchars=\\\{\}]
\PYG{g+gp}{\PYGZgt{}\PYGZgt{}\PYGZgt{} }\PYG{k+kn}{from} \PYG{n+nn}{openerm}\PYG{n+nn}{.}\PYG{n+nn}{Database} \PYG{k}{import} \PYG{n}{Database}
\PYG{g+gp}{\PYGZgt{}\PYGZgt{}\PYGZgt{} }\PYG{c+c1}{\PYGZsh{} Apertura en modo lectura}
\PYG{g+gp}{\PYGZgt{}\PYGZgt{}\PYGZgt{} }\PYG{n}{dbin} \PYG{o}{=} \PYG{n}{Database}\PYG{p}{(}\PYG{n}{file} \PYG{o}{=} \PYG{l+s+s2}{\PYGZdq{}}\PYG{l+s+s2}{out/.sin\PYGZus{}compression\PYGZus{}sin\PYGZus{}encriptacion.oerm}\PYG{l+s+s2}{\PYGZdq{}}\PYG{p}{)}
\PYG{g+gp}{\PYGZgt{}\PYGZgt{}\PYGZgt{} }\PYG{c+c1}{\PYGZsh{} Apertura en modo escritura (NO append)}
\PYG{g+gp}{\PYGZgt{}\PYGZgt{}\PYGZgt{} }\PYG{n}{dbout} \PYG{o}{=} \PYG{n}{Database}\PYG{p}{(}\PYG{n}{file} \PYG{o}{=} \PYG{l+s+s2}{\PYGZdq{}}\PYG{l+s+s2}{out/.sin\PYGZus{}compression\PYGZus{}sin\PYGZus{}encriptacion.oerm}\PYG{l+s+s2}{\PYGZdq{}}\PYG{p}{,} \PYG{n}{mode}\PYG{o}{=}\PYG{l+s+s2}{\PYGZdq{}}\PYG{l+s+s2}{wb}\PYG{l+s+s2}{\PYGZdq{}}\PYG{p}{)}
\end{sphinxVerbatim}
\index{add\_page() (método de openerm.Database.Database)@\spxentry{add\_page()}\spxextra{método de openerm.Database.Database}}

\begin{fulllineitems}
\phantomsection\label{\detokenize{openerm.Database:openerm.Database.Database.add_page}}\pysiglinewithargsret{\sphinxbfcode{\sphinxupquote{add\_page}}}{\emph{page}}{}
Agregar una página al reporte

\begin{sphinxadmonition}{warning}{Advertencia:}
Documentación pendiente
\end{sphinxadmonition}

\end{fulllineitems}

\index{add\_report() (método de openerm.Database.Database)@\spxentry{add\_report()}\spxextra{método de openerm.Database.Database}}

\begin{fulllineitems}
\phantomsection\label{\detokenize{openerm.Database:openerm.Database.Database.add_report}}\pysiglinewithargsret{\sphinxbfcode{\sphinxupquote{add\_report}}}{\emph{reporte='n/a'}, \emph{sistema='n/a'}, \emph{aplicacion='n/a'}, \emph{departamento='n/a'}, \emph{fecha='20190504'}}{}
Agregar un reporte al contenedor

\begin{sphinxadmonition}{warning}{Advertencia:}
Documentación pendiente
\end{sphinxadmonition}

\end{fulllineitems}

\index{close() (método de openerm.Database.Database)@\spxentry{close()}\spxextra{método de openerm.Database.Database}}

\begin{fulllineitems}
\phantomsection\label{\detokenize{openerm.Database:openerm.Database.Database.close}}\pysiglinewithargsret{\sphinxbfcode{\sphinxupquote{close}}}{}{}
Cerrar el Database

\begin{sphinxadmonition}{warning}{Advertencia:}
Documentación pendiente
\end{sphinxadmonition}

\end{fulllineitems}

\index{find\_text() (método de openerm.Database.Database)@\spxentry{find\_text()}\spxextra{método de openerm.Database.Database}}

\begin{fulllineitems}
\phantomsection\label{\detokenize{openerm.Database:openerm.Database.Database.find_text}}\pysiglinewithargsret{\sphinxbfcode{\sphinxupquote{find\_text}}}{\emph{text}, \emph{reports=None}}{}
Búsqueda de un texto dentro de uno o más reportes
\begin{quote}\begin{description}
\item[{Parámetros}] \leavevmode\begin{itemize}
\item {} 
\sphinxstyleliteralstrong{\sphinxupquote{text}} (\sphinxstyleliteralemphasis{\sphinxupquote{string}}) \textendash{} Patrón de texto a buscar

\item {} 
\sphinxstyleliteralstrong{\sphinxupquote{reports}} (\sphinxstyleliteralemphasis{\sphinxupquote{list}}) \textendash{} Lista de reportes dónde buscar o None en todos

\end{itemize}

\end{description}\end{quote}
\subsubsection*{Ejemplo}

\begin{sphinxVerbatim}[commandchars=\\\{\}]
\PYG{g+gp}{\PYGZgt{}\PYGZgt{}\PYGZgt{} }\PYG{k+kn}{from} \PYG{n+nn}{openerm}\PYG{n+nn}{.}\PYG{n+nn}{Database} \PYG{k}{import} \PYG{n}{Database}
\PYG{g+gp}{\PYGZgt{}\PYGZgt{}\PYGZgt{} }\PYG{n}{db} \PYG{o}{=} \PYG{n}{Database}\PYG{p}{(}\PYG{n}{file} \PYG{o}{=} \PYG{l+s+s2}{\PYGZdq{}}\PYG{l+s+s2}{out/.sin\PYGZus{}compression\PYGZus{}sin\PYGZus{}encriptacion.oerm}\PYG{l+s+s2}{\PYGZdq{}}\PYG{p}{)}
\PYG{g+gp}{\PYGZgt{}\PYGZgt{}\PYGZgt{} }\PYG{n}{db}\PYG{o}{.}\PYG{n}{find\PYGZus{}text}\PYG{p}{(}\PYG{l+s+s2}{\PYGZdq{}}\PYG{l+s+s2}{IWY3}\PYG{l+s+s2}{\PYGZdq{}}\PYG{p}{)}
\PYG{g+go}{[(2, 10, 991, \PYGZsq{}AGH8B2NULTCTJ0L\PYGZhy{}[IWY3]\PYGZhy{}4K6D8RRBYCRQCH\PYGZsq{})]}
\end{sphinxVerbatim}
\begin{quote}\begin{description}
\item[{Devuelve}] \leavevmode
Lista de reportes y páginas

\end{description}\end{quote}

\end{fulllineitems}

\index{flush() (método de openerm.Database.Database)@\spxentry{flush()}\spxextra{método de openerm.Database.Database}}

\begin{fulllineitems}
\phantomsection\label{\detokenize{openerm.Database:openerm.Database.Database.flush}}\pysiglinewithargsret{\sphinxbfcode{\sphinxupquote{flush}}}{}{}
«Comitear» los cambios

\begin{sphinxadmonition}{warning}{Advertencia:}
Documentación pendiente
\end{sphinxadmonition}

\end{fulllineitems}

\index{get\_report() (método de openerm.Database.Database)@\spxentry{get\_report()}\spxextra{método de openerm.Database.Database}}

\begin{fulllineitems}
\phantomsection\label{\detokenize{openerm.Database:openerm.Database.Database.get_report}}\pysiglinewithargsret{\sphinxbfcode{\sphinxupquote{get\_report}}}{\emph{reporte}}{}
Retorna el id de un Reporte

\begin{sphinxadmonition}{warning}{Advertencia:}
Documentación pendiente
\end{sphinxadmonition}

\end{fulllineitems}

\index{reports() (método de openerm.Database.Database)@\spxentry{reports()}\spxextra{método de openerm.Database.Database}}

\begin{fulllineitems}
\phantomsection\label{\detokenize{openerm.Database:openerm.Database.Database.reports}}\pysiglinewithargsret{\sphinxbfcode{\sphinxupquote{reports}}}{}{}
Retorna la colección de Reportes del \sphinxstylestrong{Database}
\begin{quote}\begin{description}
\item[{Devuelve}] \leavevmode
\begin{itemize}
\item {} 
{\hyperref[\detokenize{openerm.Reports:module-openerm.Reports}]{\sphinxcrossref{\sphinxcode{\sphinxupquote{openerm.Reports}}}}}

\end{itemize}


\end{description}\end{quote}
\subsubsection*{Ejemplo}

\begin{sphinxVerbatim}[commandchars=\\\{\}]
\PYG{g+gp}{\PYGZgt{}\PYGZgt{}\PYGZgt{} }\PYG{k+kn}{from} \PYG{n+nn}{openerm}\PYG{n+nn}{.}\PYG{n+nn}{Database} \PYG{k}{import} \PYG{n}{Database}
\PYG{g+gp}{\PYGZgt{}\PYGZgt{}\PYGZgt{} }\PYG{n}{db} \PYG{o}{=} \PYG{n}{Database}\PYG{p}{(}\PYG{n}{file} \PYG{o}{=} \PYG{l+s+s2}{\PYGZdq{}}\PYG{l+s+s2}{out/.sin\PYGZus{}compression\PYGZus{}sin\PYGZus{}encriptacion.oerm}\PYG{l+s+s2}{\PYGZdq{}}\PYG{p}{)}
\PYG{g+gp}{\PYGZgt{}\PYGZgt{}\PYGZgt{} }\PYG{k}{for} \PYG{n}{report} \PYG{o+ow}{in} \PYG{n}{db}\PYG{o}{.}\PYG{n}{reports}\PYG{p}{(}\PYG{p}{)}\PYG{p}{:}
\PYG{g+gp}{... }    \PYG{n+nb}{print}\PYG{p}{(}\PYG{n}{report}\PYG{p}{)}
\PYG{g+go}{Report: Reporte 1}
\PYG{g+go}{Report: Reporte 2}
\PYG{g+go}{\PYGZgt{}\PYGZgt{}\PYGZgt{}}
\end{sphinxVerbatim}

\end{fulllineitems}

\index{set\_report() (método de openerm.Database.Database)@\spxentry{set\_report()}\spxextra{método de openerm.Database.Database}}

\begin{fulllineitems}
\phantomsection\label{\detokenize{openerm.Database:openerm.Database.Database.set_report}}\pysiglinewithargsret{\sphinxbfcode{\sphinxupquote{set\_report}}}{\emph{reporte}}{}
Establece el reporte actual

\begin{sphinxadmonition}{warning}{Advertencia:}
Documentación pendiente
\end{sphinxadmonition}

\end{fulllineitems}


\end{fulllineitems}



\subsubsection{SpoolHostReprint}
\label{\detokenize{openerm.SpoolHostReprint:module-openerm.SpoolHostReprint}}\label{\detokenize{openerm.SpoolHostReprint:spoolhostreprint}}\label{\detokenize{openerm.SpoolHostReprint::doc}}\index{openerm.SpoolHostReprint (módulo)@\spxentry{openerm.SpoolHostReprint}\spxextra{módulo}}
Un \sphinxstylestrong{SpoolHostReprint} es el objeto que permite la lectura de las salidas de
impresión del tipo FCFC que son aquellos en los que la primer columna
representa un «canal» de control para la impresora. Esta clase considera esta
columna  y particularmente el codigo «1» que representa el salto de pagina.
Esta columna podra ser quitada o no segun se requiera.


\sphinxstrong{Ver también:}

\begin{itemize}
\item {} 
{\hyperref[\detokenize{openerm.SpoolFixedRecordLength:module-openerm.SpoolFixedRecordLength}]{\sphinxcrossref{\sphinxcode{\sphinxupquote{openerm.SpoolFixedRecordLength}}}}}

\end{itemize}


\index{SpoolHostReprint (clase en openerm.SpoolHostReprint)@\spxentry{SpoolHostReprint}\spxextra{clase en openerm.SpoolHostReprint}}

\begin{fulllineitems}
\phantomsection\label{\detokenize{openerm.SpoolHostReprint:openerm.SpoolHostReprint.SpoolHostReprint}}\pysiglinewithargsret{\sphinxbfcode{\sphinxupquote{class }}\sphinxcode{\sphinxupquote{openerm.SpoolHostReprint.}}\sphinxbfcode{\sphinxupquote{SpoolHostReprint}}}{\emph{inputfile}, \emph{buffer\_size=102400}, \emph{encoding='Latin1'}}{}
Bases: \sphinxcode{\sphinxupquote{object}}

Clase base para lectura de archivos tipo «host reprint».
\begin{quote}\begin{description}
\item[{Parámetros}] \leavevmode\begin{itemize}
\item {} 
\sphinxstyleliteralstrong{\sphinxupquote{inputfile}} (\sphinxstyleliteralemphasis{\sphinxupquote{string}}) \textendash{} Path y nombre del archivo a leer

\item {} 
\sphinxstyleliteralstrong{\sphinxupquote{buffer\_size}} (\sphinxstyleliteralemphasis{\sphinxupquote{int}}) \textendash{} Opcional, tamaño del buffer de lectura.
Por defecto 102400 bytes.

\item {} 
\sphinxstyleliteralstrong{\sphinxupquote{encoding}} (\sphinxstyleliteralemphasis{\sphinxupquote{string}}) \textendash{} Opcional, Codificación de lectura. Por defecto \sphinxtitleref{Latin1}

\end{itemize}

\item[{Devuelve}] \leavevmode
None

\item[{Example}] \leavevmode
\begin{sphinxVerbatim}[commandchars=\\\{\}]
\PYG{g+gp}{\PYGZgt{}\PYGZgt{}\PYGZgt{} }\PYG{k+kn}{from} \PYG{n+nn}{openerm}\PYG{n+nn}{.}\PYG{n+nn}{SpoolHostReprint} \PYG{k}{import} \PYG{n}{SpoolHostReprint}
\PYG{g+gp}{\PYGZgt{}\PYGZgt{}\PYGZgt{} }\PYG{k}{with} \PYG{n}{SpoolHostReprint}\PYG{p}{(}\PYG{n}{test\PYGZus{}file}\PYG{p}{,} \PYG{l+m+mi}{102400}\PYG{p}{)} \PYG{k}{as} \PYG{n}{s}\PYG{p}{:}
\PYG{g+gp}{\PYGZgt{}\PYGZgt{}\PYGZgt{} }            \PYG{k}{for} \PYG{n}{page} \PYG{o+ow}{in} \PYG{n}{s}\PYG{p}{:}
\PYG{g+gp}{\PYGZgt{}\PYGZgt{}\PYGZgt{} }                    \PYG{n+nb}{print}\PYG{p}{(}\PYG{n}{page}\PYG{p}{)}
\end{sphinxVerbatim}

\end{description}\end{quote}

\end{fulllineitems}



\subsubsection{SpoolFixedRecordLength}
\label{\detokenize{openerm.SpoolFixedRecordLength:module-openerm.SpoolFixedRecordLength}}\label{\detokenize{openerm.SpoolFixedRecordLength:spoolfixedrecordlength}}\label{\detokenize{openerm.SpoolFixedRecordLength::doc}}\index{openerm.SpoolFixedRecordLength (módulo)@\spxentry{openerm.SpoolFixedRecordLength}\spxextra{módulo}}
Un \sphinxstylestrong{SpoolFixedRecordLength} es el objeto que permite la lectura de las salidas
de impresión del tipo Registro de longitud fija. Este formato es típico en
plataformas IBM, archivos en EBCDIC con una longitud de registro de 256 bytes
son habituales de ver. Cada registro representa una línea, puede eventualmente
ser del tipo FCFC, y contar con un canal de control.


\sphinxstrong{Ver también:}

\begin{itemize}
\item {} 
{\hyperref[\detokenize{openerm.SpoolHostReprint:module-openerm.SpoolHostReprint}]{\sphinxcrossref{\sphinxcode{\sphinxupquote{openerm.SpoolHostReprint}}}}}

\end{itemize}


\index{SpoolFixedRecordLength (clase en openerm.SpoolFixedRecordLength)@\spxentry{SpoolFixedRecordLength}\spxextra{clase en openerm.SpoolFixedRecordLength}}

\begin{fulllineitems}
\phantomsection\label{\detokenize{openerm.SpoolFixedRecordLength:openerm.SpoolFixedRecordLength.SpoolFixedRecordLength}}\pysiglinewithargsret{\sphinxbfcode{\sphinxupquote{class }}\sphinxcode{\sphinxupquote{openerm.SpoolFixedRecordLength.}}\sphinxbfcode{\sphinxupquote{SpoolFixedRecordLength}}}{\emph{inputfile}, \emph{buffer\_size=102400}, \emph{encoding='Latin1'}, \emph{record\_len=256}, \emph{newpage\_code='1'}}{}
Bases: \sphinxcode{\sphinxupquote{object}}

Clase base para lectura de archivos de tamaño de registro fijo.
\begin{quote}\begin{description}
\item[{Parámetros}] \leavevmode\begin{itemize}
\item {} 
\sphinxstyleliteralstrong{\sphinxupquote{inputfile}} (\sphinxstyleliteralemphasis{\sphinxupquote{string}}) \textendash{} Path y nombre del archivo a leer

\item {} 
\sphinxstyleliteralstrong{\sphinxupquote{buffer\_size}} (\sphinxstyleliteralemphasis{\sphinxupquote{int}}) \textendash{} Opcional, tamaño del buffer de lectura.
Por defecto 102400 bytes.

\item {} 
\sphinxstyleliteralstrong{\sphinxupquote{encoding}} (\sphinxstyleliteralemphasis{\sphinxupquote{string}}) \textendash{} Opcional, Codificación de lectura. Por defecto \sphinxtitleref{Latin1}

\item {} 
\sphinxstyleliteralstrong{\sphinxupquote{record\_len}} (\sphinxstyleliteralemphasis{\sphinxupquote{int}}) \textendash{} Opcional, Longitud de registro (por defecto 256)

\item {} 
\sphinxstyleliteralstrong{\sphinxupquote{newpage\_code}} (\sphinxstyleliteralemphasis{\sphinxupquote{string}}) \textendash{} Opcional, Cadena o carácter que determina el salto de página

\end{itemize}

\item[{Devuelve}] \leavevmode
None

\item[{Ejemplo}] \leavevmode
\begin{sphinxVerbatim}[commandchars=\\\{\}]
\PYG{g+gp}{\PYGZgt{}\PYGZgt{}\PYGZgt{} }\PYG{k+kn}{from} \PYG{n+nn}{openerm}\PYG{n+nn}{.}\PYG{n+nn}{SpoolFixedRecordLength} \PYG{k}{import} \PYG{n}{SpoolFixedRecordLength}
\PYG{g+gp}{\PYGZgt{}\PYGZgt{}\PYGZgt{} }\PYG{k}{with} \PYG{n}{SpoolFixedRecordLength}\PYG{p}{(}\PYG{n}{test\PYGZus{}file}\PYG{p}{,} \PYG{l+m+mi}{102400}\PYG{p}{)} \PYG{k}{as} \PYG{n}{s}\PYG{p}{:}
\PYG{g+gp}{\PYGZgt{}\PYGZgt{}\PYGZgt{} }            \PYG{k}{for} \PYG{n}{page} \PYG{o+ow}{in} \PYG{n}{s}\PYG{p}{:}
\PYG{g+gp}{\PYGZgt{}\PYGZgt{}\PYGZgt{} }                    \PYG{n+nb}{print}\PYG{p}{(}\PYG{n}{page}\PYG{p}{)}
\end{sphinxVerbatim}

\end{description}\end{quote}

\end{fulllineitems}



\subsubsection{ReportMatcher}
\label{\detokenize{openerm.ReportMatcher:module-openerm.ReportMatcher}}\label{\detokenize{openerm.ReportMatcher:reportmatcher}}\label{\detokenize{openerm.ReportMatcher::doc}}\index{openerm.ReportMatcher (módulo)@\spxentry{openerm.ReportMatcher}\spxextra{módulo}}
\sphinxstylestrong{ReportMatcher} es el objeto que identifica los reportes de una
determinada cola de impresión. Lo hace mediante una serie de reglas que se
definene en una archivo de configuración en formato YAML
\index{ReportMatcher (clase en openerm.ReportMatcher)@\spxentry{ReportMatcher}\spxextra{clase en openerm.ReportMatcher}}

\begin{fulllineitems}
\phantomsection\label{\detokenize{openerm.ReportMatcher:openerm.ReportMatcher.ReportMatcher}}\pysiglinewithargsret{\sphinxbfcode{\sphinxupquote{class }}\sphinxcode{\sphinxupquote{openerm.ReportMatcher.}}\sphinxbfcode{\sphinxupquote{ReportMatcher}}}{\emph{configfile='openerm.cfg'}, \emph{configbuffer=None}}{}
Bases: \sphinxcode{\sphinxupquote{object}}

Matcher de reportes

El matcher de reportes se configura mediante un archivo yaml. Ver ejemplos
en la carpeta \sphinxtitleref{tools}, aunque tambien puede configurarase mediante un «buffer»
o string yaml ya definido.
\begin{quote}\begin{description}
\item[{Parámetros}] \leavevmode\begin{itemize}
\item {} 
\sphinxstyleliteralstrong{\sphinxupquote{configfile}} (\sphinxstyleliteralemphasis{\sphinxupquote{string}}) \textendash{} Path absoluto al archivo de configuración de la clase

\item {} 
\sphinxstyleliteralstrong{\sphinxupquote{configbuffer}} (\sphinxstyleliteralemphasis{\sphinxupquote{string}}) \textendash{} Buffer con la configuración YAML de la clase.

\item {} 
\sphinxstyleliteralstrong{\sphinxupquote{parámatro tiene prioridad en caso de ingresar tambien configfile}} (\sphinxstyleliteralemphasis{\sphinxupquote{Este}}) \textendash{} 

\end{itemize}

\end{description}\end{quote}
\subsubsection*{Ejemplo}

\begin{sphinxVerbatim}[commandchars=\\\{\}]
\PYG{g+gp}{\PYGZgt{}\PYGZgt{}\PYGZgt{} }\PYG{k+kn}{from} \PYG{n+nn}{openerm}\PYG{n+nn}{.}\PYG{n+nn}{ReportMatcher} \PYG{k}{import} \PYG{n}{ReportMatcher}
\PYG{g+gp}{\PYGZgt{}\PYGZgt{}\PYGZgt{} }\PYG{n}{r} \PYG{o}{=} \PYG{n}{ReportMatcher}\PYG{p}{(}\PYG{l+s+s2}{\PYGZdq{}}\PYG{l+s+s2}{../var/reports.yaml}\PYG{l+s+s2}{\PYGZdq{}}\PYG{p}{)}
\PYG{g+gp}{\PYGZgt{}\PYGZgt{}\PYGZgt{} }\PYG{n+nb}{print}\PYG{p}{(}\PYG{n}{r}\PYG{o}{.}\PYG{n}{match}\PYG{p}{(}\PYG{l+s+s2}{\PYGZdq{}}\PYG{l+s+s2}{R8101611}\PYG{l+s+s2}{\PYGZdq{}}\PYG{p}{)}\PYG{p}{)}
\end{sphinxVerbatim}
\index{match() (método de openerm.ReportMatcher.ReportMatcher)@\spxentry{match()}\spxextra{método de openerm.ReportMatcher.ReportMatcher}}

\begin{fulllineitems}
\phantomsection\label{\detokenize{openerm.ReportMatcher:openerm.ReportMatcher.ReportMatcher.match}}\pysiglinewithargsret{\sphinxbfcode{\sphinxupquote{match}}}{\emph{page}}{}
Trata de determinar el reporte en función de los datos de una página
\begin{quote}\begin{description}
\item[{Parámetros}] \leavevmode
\sphinxstyleliteralstrong{\sphinxupquote{page}} (\sphinxstyleliteralemphasis{\sphinxupquote{string}}) \textendash{} Texto de la página completa a identificar

\item[{Devuelve}] \leavevmode
Datos del reporte

\item[{Tipo del valor devuelto}] \leavevmode
tuple

\end{description}\end{quote}
\begin{description}
\item[{\sphinxstylestrong{Datos de retorno}:}] \leavevmode

\begin{savenotes}\sphinxattablestart
\centering
\begin{tabulary}{\linewidth}[t]{|T|T|}
\hline
\sphinxstyletheadfamily 
Tipo
&\sphinxstyletheadfamily 
Detalle
\\
\hline
string
&
Id del reporte
\\
\hline
string
&
Sistema
\\
\hline
string
&
Departamento
\\
\hline
string
&
Fecha
\\
\hline
\end{tabulary}
\par
\sphinxattableend\end{savenotes}

\end{description}
\subsubsection*{Ejemplo}

\begin{sphinxVerbatim}[commandchars=\\\{\}]
\PYG{g+gp}{\PYGZgt{}\PYGZgt{}\PYGZgt{} }\PYG{k+kn}{from} \PYG{n+nn}{openerm}\PYG{n+nn}{.}\PYG{n+nn}{ReportMatcher} \PYG{k}{import} \PYG{n}{ReportMatcher}
\PYG{g+gp}{\PYGZgt{}\PYGZgt{}\PYGZgt{} }\PYG{n}{r} \PYG{o}{=} \PYG{n}{ReportMatcher}\PYG{p}{(}\PYG{l+s+s2}{\PYGZdq{}}\PYG{l+s+s2}{../var/reports.yaml}\PYG{l+s+s2}{\PYGZdq{}}\PYG{p}{)}
\PYG{g+gp}{\PYGZgt{}\PYGZgt{}\PYGZgt{} }\PYG{n+nb}{print}\PYG{p}{(}\PYG{n}{r}\PYG{o}{.}\PYG{n}{match}\PYG{p}{(}\PYG{l+s+s2}{\PYGZdq{}}\PYG{l+s+s2}{R8101611}\PYG{l+s+s2}{\PYGZdq{}}\PYG{p}{)}\PYG{p}{)}
\end{sphinxVerbatim}

\end{fulllineitems}


\end{fulllineitems}

\phantomsection\label{\detokenize{openerm.Utils:utils}}\phantomsection\label{\detokenize{openerm.Utils:module-openerm.Utils}}\phantomsection\label{\detokenize{openerm.Utils:utils}}\index{openerm.Utils (módulo)@\spxentry{openerm.Utils}\spxextra{módulo}}

\subsubsection{Utils}
\label{\detokenize{openerm.Utils:id1}}\label{\detokenize{openerm.Utils::doc}}
Este módulo contiene todo tipo de funciones de uso general para el
proyecto \sphinxstylestrong{OpenErm}
\index{AutoNum (clase en openerm.Utils)@\spxentry{AutoNum}\spxextra{clase en openerm.Utils}}

\begin{fulllineitems}
\phantomsection\label{\detokenize{openerm.Utils:openerm.Utils.AutoNum}}\pysigline{\sphinxbfcode{\sphinxupquote{class }}\sphinxcode{\sphinxupquote{openerm.Utils.}}\sphinxbfcode{\sphinxupquote{AutoNum}}}
Bases: \sphinxcode{\sphinxupquote{object}}

Clase autonumeradora de valores
\begin{description}
\item[{Ejemplo:}] \leavevmode
\begin{sphinxVerbatim}[commandchars=\\\{\}]
\PYG{g+gp}{\PYGZgt{}\PYGZgt{}\PYGZgt{} }\PYG{k+kn}{from} \PYG{n+nn}{openerm}\PYG{n+nn}{.}\PYG{n+nn}{Utils} \PYG{k}{import} \PYG{o}{*}
\PYG{g+gp}{\PYGZgt{}\PYGZgt{}\PYGZgt{} }\PYG{n}{my\PYGZus{}id} \PYG{o}{=} \PYG{n}{AutoNum}\PYG{p}{(}\PYG{p}{)}
\PYG{g+gp}{\PYGZgt{}\PYGZgt{}\PYGZgt{} }\PYG{n}{my\PYGZus{}id}\PYG{o}{.}\PYG{n}{get}\PYG{p}{(}\PYG{l+s+s2}{\PYGZdq{}}\PYG{l+s+s2}{Prueba}\PYG{l+s+s2}{\PYGZdq{}}\PYG{p}{)}
\PYG{g+go}{1}
\PYG{g+gp}{\PYGZgt{}\PYGZgt{}\PYGZgt{} }\PYG{n}{my\PYGZus{}id}\PYG{o}{.}\PYG{n}{get}\PYG{p}{(}\PYG{l+s+s2}{\PYGZdq{}}\PYG{l+s+s2}{Otra cosa}\PYG{l+s+s2}{\PYGZdq{}}\PYG{p}{)}
\PYG{g+go}{2}
\PYG{g+gp}{\PYGZgt{}\PYGZgt{}\PYGZgt{} }\PYG{n}{my\PYGZus{}id}\PYG{o}{.}\PYG{n}{get}\PYG{p}{(}\PYG{l+s+s2}{\PYGZdq{}}\PYG{l+s+s2}{Prueba}\PYG{l+s+s2}{\PYGZdq{}}\PYG{p}{)}
\PYG{g+go}{1}
\end{sphinxVerbatim}

\end{description}
\index{get() (método de openerm.Utils.AutoNum)@\spxentry{get()}\spxextra{método de openerm.Utils.AutoNum}}

\begin{fulllineitems}
\phantomsection\label{\detokenize{openerm.Utils:openerm.Utils.AutoNum.get}}\pysiglinewithargsret{\sphinxbfcode{\sphinxupquote{get}}}{\emph{value}}{}
Retorna el numerador de un determinado valor
\begin{quote}\begin{description}
\item[{Parámetros}] \leavevmode
\sphinxstyleliteralstrong{\sphinxupquote{value}} (\sphinxstyleliteralemphasis{\sphinxupquote{any}}) \textendash{} valor a numerar

\item[{Devuelve}] \leavevmode
Número único del valor

\item[{Tipo del valor devuelto}] \leavevmode
int

\end{description}\end{quote}

\end{fulllineitems}

\index{list() (método de openerm.Utils.AutoNum)@\spxentry{list()}\spxextra{método de openerm.Utils.AutoNum}}

\begin{fulllineitems}
\phantomsection\label{\detokenize{openerm.Utils:openerm.Utils.AutoNum.list}}\pysiglinewithargsret{\sphinxbfcode{\sphinxupquote{list}}}{}{}
Retorna la lista completa de valores, numeradores
\begin{description}
\item[{Ejemplo:}] \leavevmode
\begin{sphinxVerbatim}[commandchars=\\\{\}]
\PYG{g+gp}{\PYGZgt{}\PYGZgt{}\PYGZgt{} }\PYG{k+kn}{from} \PYG{n+nn}{openerm}\PYG{n+nn}{.}\PYG{n+nn}{Utils} \PYG{k}{import} \PYG{o}{*}
\PYG{g+gp}{\PYGZgt{}\PYGZgt{}\PYGZgt{} }\PYG{n}{my\PYGZus{}id} \PYG{o}{=} \PYG{n}{AutoNum}\PYG{p}{(}\PYG{p}{)}
\PYG{g+gp}{\PYGZgt{}\PYGZgt{}\PYGZgt{} }\PYG{n}{my\PYGZus{}id}\PYG{o}{.}\PYG{n}{get}\PYG{p}{(}\PYG{l+s+s2}{\PYGZdq{}}\PYG{l+s+s2}{Prueba}\PYG{l+s+s2}{\PYGZdq{}}\PYG{p}{)}
\PYG{g+go}{1}
\PYG{g+gp}{\PYGZgt{}\PYGZgt{}\PYGZgt{} }\PYG{n}{my\PYGZus{}id}\PYG{o}{.}\PYG{n}{get}\PYG{p}{(}\PYG{l+s+s2}{\PYGZdq{}}\PYG{l+s+s2}{Otra cosa}\PYG{l+s+s2}{\PYGZdq{}}\PYG{p}{)}
\PYG{g+go}{2}
\PYG{g+gp}{\PYGZgt{}\PYGZgt{}\PYGZgt{} }\PYG{n}{my\PYGZus{}id}\PYG{o}{.}\PYG{n}{get}\PYG{p}{(}\PYG{l+s+s2}{\PYGZdq{}}\PYG{l+s+s2}{Prueba}\PYG{l+s+s2}{\PYGZdq{}}\PYG{p}{)}
\PYG{g+go}{1}
\PYG{g+gp}{\PYGZgt{}\PYGZgt{}\PYGZgt{} }\PYG{n}{my\PYGZus{}id}\PYG{o}{.}\PYG{n}{list}\PYG{p}{(}\PYG{p}{)}
\PYG{g+go}{[(\PYGZsq{}Otra cosa\PYGZsq{}, 2), (\PYGZsq{}Prueba\PYGZsq{}, 1)]}
\end{sphinxVerbatim}

\end{description}
\begin{quote}\begin{description}
\item[{Devuelve}] \leavevmode
list

\end{description}\end{quote}

\end{fulllineitems}


\end{fulllineitems}

\index{file\_accessible() (en el módulo openerm.Utils)@\spxentry{file\_accessible()}\spxextra{en el módulo openerm.Utils}}

\begin{fulllineitems}
\phantomsection\label{\detokenize{openerm.Utils:openerm.Utils.file_accessible}}\pysiglinewithargsret{\sphinxcode{\sphinxupquote{openerm.Utils.}}\sphinxbfcode{\sphinxupquote{file\_accessible}}}{\emph{filepath}, \emph{mode}}{}
Verifica la accesibilidad de un archivo en un determinado modo de apertura.
\begin{quote}\begin{description}
\item[{Parámetros}] \leavevmode
\sphinxstyleliteralstrong{\sphinxupquote{filepath}} (\sphinxstyleliteralemphasis{\sphinxupquote{string}}) \textendash{} 

\end{description}\end{quote}

\end{fulllineitems}

\index{filesInPath() (en el módulo openerm.Utils)@\spxentry{filesInPath()}\spxextra{en el módulo openerm.Utils}}

\begin{fulllineitems}
\phantomsection\label{\detokenize{openerm.Utils:openerm.Utils.filesInPath}}\pysiglinewithargsret{\sphinxcode{\sphinxupquote{openerm.Utils.}}\sphinxbfcode{\sphinxupquote{filesInPath}}}{\emph{path}, \emph{pattern='*.*'}}{}
Retorna de forma recursiva los archivos que respetan un patrón
\begin{quote}\begin{description}
\item[{Parámetros}] \leavevmode\begin{itemize}
\item {} 
\sphinxstyleliteralstrong{\sphinxupquote{path}} (\sphinxstyleliteralemphasis{\sphinxupquote{string}}) \textendash{} Path principal

\item {} 
\sphinxstyleliteralstrong{\sphinxupquote{pattern}} (\sphinxstyleliteralemphasis{\sphinxupquote{string}}) \textendash{} (Opcional) patrón a buscar, por defecto “\sphinxstyleemphasis{.}”

\end{itemize}

\end{description}\end{quote}
\begin{description}
\item[{Ejemplo:}] \leavevmode
\begin{sphinxVerbatim}[commandchars=\\\{\}]
\PYG{g+gp}{\PYGZgt{}\PYGZgt{}\PYGZgt{} }\PYG{k}{for} \PYG{n}{f} \PYG{o+ow}{in} \PYG{n}{filesInPath}\PYG{p}{(}\PYG{l+s+s2}{\PYGZdq{}}\PYG{l+s+s2}{c:}\PYG{l+s+s2}{\PYGZdq{}}\PYG{p}{,} \PYG{l+s+s2}{\PYGZdq{}}\PYG{l+s+s2}{*.txt}\PYG{l+s+s2}{\PYGZdq{}}\PYG{p}{)}\PYG{p}{:}
\PYG{g+gp}{\PYGZgt{}\PYGZgt{}\PYGZgt{} }    \PYG{n+nb}{print}\PYG{p}{(}\PYG{n}{f}\PYG{p}{)}
\end{sphinxVerbatim}

\end{description}

\end{fulllineitems}

\index{generate\_filename() (en el módulo openerm.Utils)@\spxentry{generate\_filename()}\spxextra{en el módulo openerm.Utils}}

\begin{fulllineitems}
\phantomsection\label{\detokenize{openerm.Utils:openerm.Utils.generate_filename}}\pysiglinewithargsret{\sphinxcode{\sphinxupquote{openerm.Utils.}}\sphinxbfcode{\sphinxupquote{generate\_filename}}}{\emph{mask}}{}
Genera un nombre de archivo en función a una máscara
\begin{quote}\begin{description}
\item[{Parámetros}] \leavevmode
\sphinxstyleliteralstrong{\sphinxupquote{mask}} (\sphinxstyleliteralemphasis{\sphinxupquote{string}}) \textendash{} Mascara usada

\end{description}\end{quote}

\end{fulllineitems}

\index{get\_values\_from\_byte() (en el módulo openerm.Utils)@\spxentry{get\_values\_from\_byte()}\spxextra{en el módulo openerm.Utils}}

\begin{fulllineitems}
\phantomsection\label{\detokenize{openerm.Utils:openerm.Utils.get_values_from_byte}}\pysiglinewithargsret{\sphinxcode{\sphinxupquote{openerm.Utils.}}\sphinxbfcode{\sphinxupquote{get\_values\_from\_byte}}}{\emph{byte}}{}
Retorna dos valores de un byte empaquetado
\begin{quote}\begin{description}
\item[{Parámetros}] \leavevmode
\sphinxstyleliteralstrong{\sphinxupquote{byte}} \textendash{} Entero que represta un byte

\end{description}\end{quote}
\begin{description}
\item[{Return}] \leavevmode
(v1, v2) Tupla con los dos valores enteros

\end{description}

\end{fulllineitems}

\index{set\_byte\_from\_values() (en el módulo openerm.Utils)@\spxentry{set\_byte\_from\_values()}\spxextra{en el módulo openerm.Utils}}

\begin{fulllineitems}
\phantomsection\label{\detokenize{openerm.Utils:openerm.Utils.set_byte_from_values}}\pysiglinewithargsret{\sphinxcode{\sphinxupquote{openerm.Utils.}}\sphinxbfcode{\sphinxupquote{set\_byte\_from\_values}}}{\emph{value1}, \emph{value2}}{}
Retorna un byte empaquetado a partir de dos valores
\begin{quote}\begin{description}
\item[{Parámetros}] \leavevmode\begin{itemize}
\item {} 
\sphinxstyleliteralstrong{\sphinxupquote{value1}} (\sphinxstyleliteralemphasis{\sphinxupquote{int}}) \textendash{} Entero 0 a 127

\item {} 
\sphinxstyleliteralstrong{\sphinxupquote{value2}} (\sphinxstyleliteralemphasis{\sphinxupquote{int}}) \textendash{} Entero 0 a 127

\end{itemize}

\item[{Devuelve}] \leavevmode
byte

\end{description}\end{quote}

\end{fulllineitems}

\index{slugify() (en el módulo openerm.Utils)@\spxentry{slugify()}\spxextra{en el módulo openerm.Utils}}

\begin{fulllineitems}
\phantomsection\label{\detokenize{openerm.Utils:openerm.Utils.slugify}}\pysiglinewithargsret{\sphinxcode{\sphinxupquote{openerm.Utils.}}\sphinxbfcode{\sphinxupquote{slugify}}}{\emph{text}, \emph{delim='-'}}{}
Normaliza una cadena para ser usada como nombre de archivo.
\begin{quote}\begin{description}
\item[{Parámetros}] \leavevmode\begin{itemize}
\item {} 
\sphinxstyleliteralstrong{\sphinxupquote{text}} (\sphinxstyleliteralemphasis{\sphinxupquote{str}}) \textendash{} String a normalizar

\item {} 
\sphinxstyleliteralstrong{\sphinxupquote{delim}} (\sphinxstyleliteralemphasis{\sphinxupquote{str}}) \textendash{} Caracter de reemplazo de aquellos no válidos

\end{itemize}

\end{description}\end{quote}
\begin{description}
\item[{Ejemplo:}] \leavevmode
\begin{sphinxVerbatim}[commandchars=\\\{\}]
\PYG{g+gp}{\PYGZgt{}\PYGZgt{}\PYGZgt{} }\PYG{k+kn}{from} \PYG{n+nn}{openerm}\PYG{n+nn}{.}\PYG{n+nn}{Utils} \PYG{k}{import} \PYG{o}{*}
\PYG{g+gp}{\PYGZgt{}\PYGZgt{}\PYGZgt{} }\PYG{n}{slugify}\PYG{p}{(}\PYG{l+s+s2}{\PYGZdq{}}\PYG{l+s+s2}{Esto, no es válido como nombre de Archivo!}\PYG{l+s+s2}{\PYGZdq{}}\PYG{p}{,} \PYG{l+s+s2}{\PYGZdq{}}\PYG{l+s+s2}{\PYGZhy{}}\PYG{l+s+s2}{\PYGZdq{}}\PYG{p}{)}
\PYG{g+go}{\PYGZsq{}esto\PYGZhy{}no\PYGZhy{}es\PYGZhy{}valido\PYGZhy{}como\PYGZhy{}nombre\PYGZhy{}de\PYGZhy{}archivo\PYGZsq{}}
\end{sphinxVerbatim}

\end{description}

\end{fulllineitems}

\index{str\_to\_list() (en el módulo openerm.Utils)@\spxentry{str\_to\_list()}\spxextra{en el módulo openerm.Utils}}

\begin{fulllineitems}
\phantomsection\label{\detokenize{openerm.Utils:openerm.Utils.str_to_list}}\pysiglinewithargsret{\sphinxcode{\sphinxupquote{openerm.Utils.}}\sphinxbfcode{\sphinxupquote{str\_to\_list}}}{\emph{str\_value}, \emph{maxvalue}}{}
Devuelve una lista de enteros a partir de un string
\begin{quote}\begin{description}
\item[{Parámetros}] \leavevmode\begin{itemize}
\item {} 
\sphinxstyleliteralstrong{\sphinxupquote{str\_value}} (\sphinxstyleliteralemphasis{\sphinxupquote{string}}) \textendash{} Cadena de números separados por , o -

\item {} 
\sphinxstyleliteralstrong{\sphinxupquote{maxvalue}} (\sphinxstyleliteralemphasis{\sphinxupquote{int}}) \textendash{} Máximo valor que puede tener la lista

\end{itemize}

\end{description}\end{quote}
\begin{description}
\item[{Ejemplo:}] \leavevmode
\begin{sphinxVerbatim}[commandchars=\\\{\}]
\PYG{g+gp}{\PYGZgt{}\PYGZgt{}\PYGZgt{} }\PYG{k+kn}{from} \PYG{n+nn}{openerm}\PYG{n+nn}{.}\PYG{n+nn}{Utils} \PYG{k}{import} \PYG{o}{*}
\PYG{g+gp}{\PYGZgt{}\PYGZgt{}\PYGZgt{} }\PYG{n}{str\PYGZus{}to\PYGZus{}list}\PYG{p}{(}\PYG{l+s+s2}{\PYGZdq{}}\PYG{l+s+s2}{1,2,3,4}\PYG{l+s+s2}{\PYGZdq{}}\PYG{p}{,} \PYG{l+m+mi}{10}\PYG{p}{)}
\PYG{g+go}{[1, 2, 3, 4]}
\PYG{g+gp}{\PYGZgt{}\PYGZgt{}\PYGZgt{} }\PYG{n}{str\PYGZus{}to\PYGZus{}list}\PYG{p}{(}\PYG{l+s+s2}{\PYGZdq{}}\PYG{l+s+s2}{1\PYGZhy{}6,9, 12\PYGZhy{}14}\PYG{l+s+s2}{\PYGZdq{}}\PYG{p}{,} \PYG{l+m+mi}{15}\PYG{p}{)}
\PYG{g+go}{[1, 2, 3, 4, 5, 6, 9, 12, 13, 14]}
\end{sphinxVerbatim}

\end{description}

\end{fulllineitems}

\phantomsection\label{\detokenize{openerm.Config:config}}\phantomsection\label{\detokenize{openerm.Config:module-openerm.Config}}\phantomsection\label{\detokenize{openerm.Config:config}}\index{openerm.Config (módulo)@\spxentry{openerm.Config}\spxextra{módulo}}

\subsubsection{Config}
\label{\detokenize{openerm.Config:id1}}\label{\detokenize{openerm.Config::doc}}
Esta clase gestiona los archivos de configuración del proyecto Oerm. Se encarga
de la carga de los archivos .cfg (formato yaml), realiza la validación de cada
uno y devuelve un diccionario de datos para ser aprovechado luego.
\index{Config (clase en openerm.Config)@\spxentry{Config}\spxextra{clase en openerm.Config}}

\begin{fulllineitems}
\phantomsection\label{\detokenize{openerm.Config:openerm.Config.Config}}\pysiglinewithargsret{\sphinxbfcode{\sphinxupquote{class }}\sphinxcode{\sphinxupquote{openerm.Config.}}\sphinxbfcode{\sphinxupquote{Config}}}{\emph{configfile}, \emph{schema=None}}{}
Bases: \sphinxcode{\sphinxupquote{object}}

Clase base para el manejo de configuraciones «jerarquicas».
:param configfile: Nombre del archivo físico de configuración a cargar
:type configfile: string
:param schema: Esquema de validación en formato yaml
:type schema: string
\subsubsection*{Ejemplo}

\begin{sphinxVerbatim}[commandchars=\\\{\}]
\PYG{g+gp}{\PYGZgt{}\PYGZgt{}\PYGZgt{} }\PYG{k+kn}{from} \PYG{n+nn}{openerm}\PYG{n+nn}{.}\PYG{n+nn}{Database} \PYG{k}{import} \PYG{n}{Config}
\PYG{g+gp}{\PYGZgt{}\PYGZgt{}\PYGZgt{} }\PYG{n}{cfg} \PYG{o}{=} \PYG{n}{Config}\PYG{p}{(}\PYG{l+s+s2}{\PYGZdq{}}\PYG{l+s+s2}{file.cfg}\PYG{l+s+s2}{\PYGZdq{}}\PYG{p}{,} \PYG{n}{schema\PYGZus{}yaml}\PYG{p}{)}
\end{sphinxVerbatim}

\end{fulllineitems}

\index{ConfigLoadingException@\spxentry{ConfigLoadingException}}

\begin{fulllineitems}
\phantomsection\label{\detokenize{openerm.Config:openerm.Config.ConfigLoadingException}}\pysiglinewithargsret{\sphinxbfcode{\sphinxupquote{exception }}\sphinxcode{\sphinxupquote{openerm.Config.}}\sphinxbfcode{\sphinxupquote{ConfigLoadingException}}}{\emph{*args}}{}
Bases: \sphinxcode{\sphinxupquote{Exception}}

Excepciones especiales para los cargadores de configuraciones

\end{fulllineitems}

\index{LoadConfig (clase en openerm.Config)@\spxentry{LoadConfig}\spxextra{clase en openerm.Config}}

\begin{fulllineitems}
\phantomsection\label{\detokenize{openerm.Config:openerm.Config.LoadConfig}}\pysiglinewithargsret{\sphinxbfcode{\sphinxupquote{class }}\sphinxcode{\sphinxupquote{openerm.Config.}}\sphinxbfcode{\sphinxupquote{LoadConfig}}}{\emph{configfile}}{}
Bases: {\hyperref[\detokenize{openerm.Config:openerm.Config.Config}]{\sphinxcrossref{\sphinxcode{\sphinxupquote{openerm.Config.Config}}}}}

Clase base para el manejo de la configuración del proceso de carga.

\end{fulllineitems}

\index{ProcessorConfig (clase en openerm.Config)@\spxentry{ProcessorConfig}\spxextra{clase en openerm.Config}}

\begin{fulllineitems}
\phantomsection\label{\detokenize{openerm.Config:openerm.Config.ProcessorConfig}}\pysiglinewithargsret{\sphinxbfcode{\sphinxupquote{class }}\sphinxcode{\sphinxupquote{openerm.Config.}}\sphinxbfcode{\sphinxupquote{ProcessorConfig}}}{\emph{configfile}}{}
Bases: {\hyperref[\detokenize{openerm.Config:openerm.Config.Config}]{\sphinxcrossref{\sphinxcode{\sphinxupquote{openerm.Config.Config}}}}}

Clase base para el manejo de la configuración del proceso de carga.

\end{fulllineitems}



\subsection{Internas - Contenedores}
\label{\detokenize{openerm:internas-contenedores}}\phantomsection\label{\detokenize{openerm.MetadataContainer:metadatacontainer}}\phantomsection\label{\detokenize{openerm.MetadataContainer:module-openerm.MetadataContainer}}\phantomsection\label{\detokenize{openerm.MetadataContainer:metadatacontainer}}\index{openerm.MetadataContainer (módulo)@\spxentry{openerm.MetadataContainer}\spxextra{módulo}}

\subsubsection{MetadataContainer}
\label{\detokenize{openerm.MetadataContainer:id1}}\label{\detokenize{openerm.MetadataContainer::doc}}
Un contenedor de metadatos es un tipo de «bloque» de un Database OpenErm, representa
los atributos que describen el reporte. Cada reporte tiene un conjunto básico y fijo
de atributos y es posible agregar los que deseemos.


\sphinxstrong{Ver también:}

\begin{itemize}
\item {} 
{\hyperref[\detokenize{openerm.PageContainer:module-openerm.PageContainer}]{\sphinxcrossref{\sphinxcode{\sphinxupquote{openerm.PageContainer}}}}}

\item {} 
{\hyperref[\detokenize{openerm.MetadataContainer:module-openerm.MetadataContainer}]{\sphinxcrossref{\sphinxcode{\sphinxupquote{openerm.MetadataContainer}}}}}

\end{itemize}


\index{MetadataContainer (clase en openerm.MetadataContainer)@\spxentry{MetadataContainer}\spxextra{clase en openerm.MetadataContainer}}

\begin{fulllineitems}
\phantomsection\label{\detokenize{openerm.MetadataContainer:openerm.MetadataContainer.MetadataContainer}}\pysiglinewithargsret{\sphinxbfcode{\sphinxupquote{class }}\sphinxcode{\sphinxupquote{openerm.MetadataContainer.}}\sphinxbfcode{\sphinxupquote{MetadataContainer}}}{\emph{metadata=None}}{}
Bases: \sphinxcode{\sphinxupquote{object}}

Contenedeor de los «metadatos» del reporte
\begin{quote}\begin{description}
\item[{Parámetros}] \leavevmode
\sphinxstyleliteralstrong{\sphinxupquote{metadata}} (\sphinxstyleliteralemphasis{\sphinxupquote{dict}}) \textendash{} Diccionario de los datos fundamentales de un reporte

\end{description}\end{quote}
\index{add() (método de openerm.MetadataContainer.MetadataContainer)@\spxentry{add()}\spxextra{método de openerm.MetadataContainer.MetadataContainer}}

\begin{fulllineitems}
\phantomsection\label{\detokenize{openerm.MetadataContainer:openerm.MetadataContainer.MetadataContainer.add}}\pysiglinewithargsret{\sphinxbfcode{\sphinxupquote{add}}}{\emph{extradata}}{}
Agrega un diccionario de datos extra al contenedor
\begin{quote}\begin{description}
\item[{Parámetros}] \leavevmode
\sphinxstyleliteralstrong{\sphinxupquote{extradata}} (\sphinxstyleliteralemphasis{\sphinxupquote{dict}}) \textendash{} Datos adicionales

\end{description}\end{quote}
\begin{description}
\item[{Ejemplo:}] \leavevmode
\begin{sphinxVerbatim}[commandchars=\\\{\}]
\PYG{g+gp}{\PYGZgt{}\PYGZgt{}\PYGZgt{} }\PYG{k+kn}{import} \PYG{n+nn}{datetime}
\PYG{g+gp}{\PYGZgt{}\PYGZgt{}\PYGZgt{} }\PYG{k+kn}{from} \PYG{n+nn}{openerm}\PYG{n+nn}{.}\PYG{n+nn}{MetadataContainer} \PYG{k}{import} \PYG{n}{MetadataContainer}
\PYG{g+gp}{\PYGZgt{}\PYGZgt{}\PYGZgt{} }\PYG{n}{now} \PYG{o}{=} \PYG{n}{datetime}\PYG{o}{.}\PYG{n}{datetime}\PYG{o}{.}\PYG{n}{now}\PYG{p}{(}\PYG{p}{)}
\PYG{g+gp}{\PYGZgt{}\PYGZgt{}\PYGZgt{} }\PYG{n}{m} \PYG{o}{=} \PYG{n}{MetadataContainer}\PYG{p}{(}\PYG{l+s+s2}{\PYGZdq{}}\PYG{l+s+s2}{Reporte sin identificar}\PYG{l+s+s2}{\PYGZdq{}}\PYG{p}{,} \PYG{l+s+s2}{\PYGZdq{}}\PYG{l+s+s2}{n/a}\PYG{l+s+s2}{\PYGZdq{}}\PYG{p}{,} \PYG{l+s+s2}{\PYGZdq{}}\PYG{l+s+s2}{n/a}\PYG{l+s+s2}{\PYGZdq{}}\PYG{p}{,} \PYG{l+s+s2}{\PYGZdq{}}\PYG{l+s+s2}{n/a}\PYG{l+s+s2}{\PYGZdq{}}\PYG{p}{,} \PYG{n}{now}\PYG{p}{)}
\PYG{g+gp}{\PYGZgt{}\PYGZgt{}\PYGZgt{} }\PYG{n}{extra\PYGZus{}data} \PYG{o}{=} \PYG{p}{\PYGZob{}}\PYG{l+s+s2}{\PYGZdq{}}\PYG{l+s+s2}{autor}\PYG{l+s+s2}{\PYGZdq{}}\PYG{p}{:} \PYG{l+s+s2}{\PYGZdq{}}\PYG{l+s+s2}{Ernesto Guevara}\PYG{l+s+s2}{\PYGZdq{}}\PYG{p}{,} \PYG{l+s+s2}{\PYGZdq{}}\PYG{l+s+s2}{Estado}\PYG{l+s+s2}{\PYGZdq{}}\PYG{p}{:} \PYG{l+s+s2}{\PYGZdq{}}\PYG{l+s+s2}{Draft}\PYG{l+s+s2}{\PYGZdq{}}\PYG{p}{\PYGZcb{}}
\PYG{g+gp}{\PYGZgt{}\PYGZgt{}\PYGZgt{} }\PYG{n}{m}\PYG{o}{.}\PYG{n}{add}\PYG{p}{(}\PYG{n}{extra\PYGZus{}data}\PYG{p}{)}
\end{sphinxVerbatim}

\end{description}

\end{fulllineitems}

\index{dump() (método de openerm.MetadataContainer.MetadataContainer)@\spxentry{dump()}\spxextra{método de openerm.MetadataContainer.MetadataContainer}}

\begin{fulllineitems}
\phantomsection\label{\detokenize{openerm.MetadataContainer:openerm.MetadataContainer.MetadataContainer.dump}}\pysiglinewithargsret{\sphinxbfcode{\sphinxupquote{dump}}}{}{}
Retorna en bytes el contenido de los metadatos ya listos para comprimir y
eventualmentre salvar físicamente en el Database Oerm. El «dump» termina siendo
una representación plana de un diccionario de datos en formato JSON. Los datos
pueden ser variables.
\begin{description}
\item[{Ejemplo:}] \leavevmode
\begin{sphinxVerbatim}[commandchars=\\\{\}]
\PYG{g+gp}{\PYGZgt{}\PYGZgt{}\PYGZgt{} }\PYG{k+kn}{import} \PYG{n+nn}{datetime}
\PYG{g+gp}{\PYGZgt{}\PYGZgt{}\PYGZgt{} }\PYG{k+kn}{from} \PYG{n+nn}{openerm}\PYG{n+nn}{.}\PYG{n+nn}{MetadataContainer} \PYG{k}{import} \PYG{n}{MetadataContainer}
\PYG{g+gp}{\PYGZgt{}\PYGZgt{}\PYGZgt{} }\PYG{n}{now} \PYG{o}{=} \PYG{n}{datetime}\PYG{o}{.}\PYG{n}{datetime}\PYG{o}{.}\PYG{n}{now}\PYG{p}{(}\PYG{p}{)}
\PYG{g+gp}{\PYGZgt{}\PYGZgt{}\PYGZgt{} }\PYG{n}{m} \PYG{o}{=} \PYG{n}{MetadataContainer}\PYG{p}{(}\PYG{l+s+s2}{\PYGZdq{}}\PYG{l+s+s2}{Reporte sin identificar}\PYG{l+s+s2}{\PYGZdq{}}\PYG{p}{,} \PYG{l+s+s2}{\PYGZdq{}}\PYG{l+s+s2}{n/a}\PYG{l+s+s2}{\PYGZdq{}}\PYG{p}{,} \PYG{l+s+s2}{\PYGZdq{}}\PYG{l+s+s2}{n/a}\PYG{l+s+s2}{\PYGZdq{}}\PYG{p}{,} \PYG{l+s+s2}{\PYGZdq{}}\PYG{l+s+s2}{n/a}\PYG{l+s+s2}{\PYGZdq{}}\PYG{p}{,} \PYG{n}{now}\PYG{p}{)}
\PYG{g+gp}{\PYGZgt{}\PYGZgt{}\PYGZgt{} }\PYG{n}{data} \PYG{o}{=} \PYG{n}{m}\PYG{o}{.}\PYG{n}{dump}\PYG{p}{(}\PYG{p}{)}
\PYG{g+gp}{\PYGZgt{}\PYGZgt{}\PYGZgt{} }\PYG{n+nb}{print}\PYG{p}{(}\PYG{n}{data}\PYG{p}{)}
\PYG{g+go}{b\PYGZsq{}\PYGZob{}\PYGZdq{}departamento\PYGZdq{}: \PYGZdq{}n/a\PYGZdq{}, \PYGZdq{}fecha\PYGZdq{}: \PYGZdq{}20160914\PYGZdq{}, \PYGZdq{}aplicacion\PYGZdq{}: \PYGZdq{}n/a\PYGZdq{}, \PYGZdq{}sistema\PYGZdq{}: \PYGZdq{}n/a\PYGZdq{}, \PYGZdq{}reporte\PYGZdq{}: \PYGZdq{}Reporte sin identificar\PYGZdq{}\PYGZcb{}\PYGZsq{}}
\end{sphinxVerbatim}

\end{description}

\begin{sphinxadmonition}{note}{Nota:}
El método \sphinxstyleemphasis{dump} entrega una estructura como el siguiente ejemplo:

\begin{sphinxVerbatim}[commandchars=\\\{\}]
\PYG{o}{+}\PYG{o}{==}\PYG{o}{==}\PYG{o}{==}\PYG{o}{==}\PYG{o}{==}\PYG{o}{==}\PYG{o}{=}\PYG{o}{+}
\PYG{o}{\textbar{}}    \PYG{n}{Datos}    \PYG{o}{\textbar{}} \PYG{o}{\PYGZhy{}}\PYG{o}{\PYGZhy{}}\PYG{o}{\PYGZgt{}} \PYG{n}{Comprimibles}
\PYG{o}{\textbar{}}  \PYG{n}{Json} \PYG{n}{dump}  \PYG{o}{\textbar{}}
\PYG{o}{+}\PYG{o}{==}\PYG{o}{==}\PYG{o}{==}\PYG{o}{==}\PYG{o}{==}\PYG{o}{==}\PYG{o}{=}\PYG{o}{+}
\end{sphinxVerbatim}
\end{sphinxadmonition}

\end{fulllineitems}

\index{load() (método de openerm.MetadataContainer.MetadataContainer)@\spxentry{load()}\spxextra{método de openerm.MetadataContainer.MetadataContainer}}

\begin{fulllineitems}
\phantomsection\label{\detokenize{openerm.MetadataContainer:openerm.MetadataContainer.MetadataContainer.load}}\pysiglinewithargsret{\sphinxbfcode{\sphinxupquote{load}}}{\emph{data}}{}
Recupera de un conjunto de bytes los metadatos
\begin{quote}\begin{description}
\item[{Parámetros}] \leavevmode
\sphinxstyleliteralstrong{\sphinxupquote{data}} \textendash{} bytes

\end{description}\end{quote}
\begin{description}
\item[{Ejemplo:}] \leavevmode
\begin{sphinxVerbatim}[commandchars=\\\{\}]
\PYG{g+gp}{\PYGZgt{}\PYGZgt{}\PYGZgt{} }\PYG{k+kn}{import} \PYG{n+nn}{datetime}
\PYG{g+gp}{\PYGZgt{}\PYGZgt{}\PYGZgt{} }\PYG{k+kn}{from} \PYG{n+nn}{openerm}\PYG{n+nn}{.}\PYG{n+nn}{MetadataContainer} \PYG{k}{import} \PYG{n}{MetadataContainer}
\PYG{g+gp}{\PYGZgt{}\PYGZgt{}\PYGZgt{} }\PYG{n}{now} \PYG{o}{=} \PYG{n}{datetime}\PYG{o}{.}\PYG{n}{datetime}\PYG{o}{.}\PYG{n}{now}\PYG{p}{(}\PYG{p}{)}
\PYG{g+gp}{\PYGZgt{}\PYGZgt{}\PYGZgt{} }\PYG{n}{m} \PYG{o}{=} \PYG{n}{MetadataContainer}\PYG{p}{(}\PYG{l+s+s2}{\PYGZdq{}}\PYG{l+s+s2}{Reporte sin identificar}\PYG{l+s+s2}{\PYGZdq{}}\PYG{p}{,} \PYG{l+s+s2}{\PYGZdq{}}\PYG{l+s+s2}{n/a}\PYG{l+s+s2}{\PYGZdq{}}\PYG{p}{,} \PYG{l+s+s2}{\PYGZdq{}}\PYG{l+s+s2}{n/a}\PYG{l+s+s2}{\PYGZdq{}}\PYG{p}{,} \PYG{l+s+s2}{\PYGZdq{}}\PYG{l+s+s2}{n/a}\PYG{l+s+s2}{\PYGZdq{}}\PYG{p}{,} \PYG{n}{now}\PYG{p}{)}
\PYG{g+gp}{\PYGZgt{}\PYGZgt{}\PYGZgt{} }\PYG{n}{data} \PYG{o}{=} \PYG{n}{m}\PYG{o}{.}\PYG{n}{dump}\PYG{p}{(}\PYG{p}{)}
\PYG{g+gp}{\PYGZgt{}\PYGZgt{}\PYGZgt{} }\PYG{n+nb}{print}\PYG{p}{(}\PYG{n}{data}\PYG{p}{)}
\PYG{g+go}{b\PYGZsq{}\PYGZob{}\PYGZdq{}departamento\PYGZdq{}: \PYGZdq{}n/a\PYGZdq{}, \PYGZdq{}fecha\PYGZdq{}: \PYGZdq{}20160914\PYGZdq{}, \PYGZdq{}aplicacion\PYGZdq{}: \PYGZdq{}n/a\PYGZdq{}, \PYGZdq{}sistema\PYGZdq{}: \PYGZdq{}n/a\PYGZdq{}, \PYGZdq{}reporte\PYGZdq{}: \PYGZdq{}Reporte sin identificar\PYGZdq{}\PYGZcb{}\PYGZsq{}}
\PYG{g+gp}{\PYGZgt{}\PYGZgt{}\PYGZgt{} }\PYG{n}{m2} \PYG{o}{=} \PYG{n}{MetadataContainer}\PYG{p}{(}\PYG{p}{)}
\PYG{g+gp}{\PYGZgt{}\PYGZgt{}\PYGZgt{} }\PYG{n}{m2}\PYG{o}{.}\PYG{n}{load}\PYG{p}{(}\PYG{n}{data}\PYG{p}{)}
\PYG{g+gp}{\PYGZgt{}\PYGZgt{}\PYGZgt{} }\PYG{n+nb}{print}\PYG{p}{(}\PYG{n}{m2}\PYG{p}{)}
\PYG{g+go}{\PYGZob{}\PYGZsq{}departamento\PYGZsq{}: \PYGZsq{}n/a\PYGZsq{}, \PYGZsq{}fecha\PYGZsq{}: \PYGZsq{}20160914\PYGZsq{}, \PYGZsq{}sistema\PYGZsq{}: \PYGZsq{}n/a\PYGZsq{}, \PYGZsq{}aplicacion\PYGZsq{}: \PYGZsq{}n/a\PYGZsq{}, \PYGZsq{}reporte\PYGZsq{}: \PYGZsq{}Reporte sin identificar\PYGZsq{}\PYGZcb{}}
\end{sphinxVerbatim}

\end{description}

\end{fulllineitems}


\end{fulllineitems}

\phantomsection\label{\detokenize{openerm.PageContainer:pagecontainer}}\phantomsection\label{\detokenize{openerm.PageContainer:module-openerm.PageContainer}}\phantomsection\label{\detokenize{openerm.PageContainer:pagecontainer}}\index{openerm.PageContainer (módulo)@\spxentry{openerm.PageContainer}\spxextra{módulo}}

\subsubsection{PageContainer}
\label{\detokenize{openerm.PageContainer:id1}}\label{\detokenize{openerm.PageContainer::doc}}
Un contenedor de páginas es un objeto que alberga de 1 a N páginas de un reporte. Es
uno de los posibles tipos de Bloque que pueden salvarse en un Database \sphinxstylestrong{OpenErm}. La
idea de agrupar varias páginas permite mejorar la performance de los compresores usados
logrando mejores ratios de compresión y por ende reducir el tamaño de los reportes y
el trafico de red o de disco.


\sphinxstrong{Ver también:}

\begin{itemize}
\item {} 
{\hyperref[\detokenize{openerm.MetadataContainer:module-openerm.MetadataContainer}]{\sphinxcrossref{\sphinxcode{\sphinxupquote{openerm.MetadataContainer}}}}}

\item {} 
{\hyperref[\detokenize{openerm.PageContainer:module-openerm.PageContainer}]{\sphinxcrossref{\sphinxcode{\sphinxupquote{openerm.PageContainer}}}}}

\end{itemize}


\index{PageContainer (clase en openerm.PageContainer)@\spxentry{PageContainer}\spxextra{clase en openerm.PageContainer}}

\begin{fulllineitems}
\phantomsection\label{\detokenize{openerm.PageContainer:openerm.PageContainer.PageContainer}}\pysiglinewithargsret{\sphinxbfcode{\sphinxupquote{class }}\sphinxcode{\sphinxupquote{openerm.PageContainer.}}\sphinxbfcode{\sphinxupquote{PageContainer}}}{\emph{max\_page\_count=1}}{}
Bases: \sphinxcode{\sphinxupquote{object}}

Clase base para el manejo de un contenedor de páginas de reportes.Esta clase
representa un contenedor de páginas, se agrupan las páginas en grupos para mejorar
los niveles de compresión y de lectura/escritura.
\begin{quote}\begin{description}
\item[{Parámetros}] \leavevmode
\sphinxstyleliteralstrong{\sphinxupquote{max\_page\_count}} (\sphinxstyleliteralemphasis{\sphinxupquote{int}}) \textendash{} Cantidad máxima de página en el contenedor

\end{description}\end{quote}

Ejemplo:

\begin{sphinxVerbatim}[commandchars=\\\{\}]
\PYG{g+gp}{\PYGZgt{}\PYGZgt{}\PYGZgt{} }\PYG{k+kn}{from} \PYG{n+nn}{openerm}\PYG{n+nn}{.}\PYG{n+nn}{PageContainer} \PYG{k}{import} \PYG{n}{PageContainer}
\PYG{g+gp}{\PYGZgt{}\PYGZgt{}\PYGZgt{} }\PYG{n}{p} \PYG{o}{=} \PYG{n}{PageContainer}\PYG{p}{(}\PYG{l+m+mi}{10}\PYG{p}{)}
\end{sphinxVerbatim}
\index{add() (método de openerm.PageContainer.PageContainer)@\spxentry{add()}\spxextra{método de openerm.PageContainer.PageContainer}}

\begin{fulllineitems}
\phantomsection\label{\detokenize{openerm.PageContainer:openerm.PageContainer.PageContainer.add}}\pysiglinewithargsret{\sphinxbfcode{\sphinxupquote{add}}}{\emph{page}}{}
Agrega una página al grupo.     Esta rutina agrega un «string» que representa
la página en el contenedor. La página queda registrada en una lista interna.
\begin{quote}\begin{description}
\item[{Parámetros}] \leavevmode
\sphinxstyleliteralstrong{\sphinxupquote{page}} (\sphinxstyleliteralemphasis{\sphinxupquote{string}}) \textendash{} Texto completo de la página a incorporar

\end{description}\end{quote}

\begin{sphinxadmonition}{error}{Error:}
\sphinxstylestrong{ValueError}: Cuando se superó la máxima cantidad de páginas del contenedor
\end{sphinxadmonition}

Ejemplo:

\begin{sphinxVerbatim}[commandchars=\\\{\}]
\PYG{g+gp}{\PYGZgt{}\PYGZgt{}\PYGZgt{} }\PYG{k+kn}{from} \PYG{n+nn}{openerm}\PYG{n+nn}{.}\PYG{n+nn}{PageContainer} \PYG{k}{import} \PYG{n}{PageContainer}
\PYG{g+gp}{\PYGZgt{}\PYGZgt{}\PYGZgt{} }\PYG{n}{p} \PYG{o}{=} \PYG{n}{PageContainer}\PYG{p}{(}\PYG{l+m+mi}{10}\PYG{p}{)}
\PYG{g+gp}{\PYGZgt{}\PYGZgt{}\PYGZgt{} }\PYG{n}{sample\PYGZus{}page} \PYG{o}{=} \PYG{l+s+s2}{\PYGZdq{}}\PYG{l+s+s2}{Pagina de una sola linea}\PYG{l+s+s2}{\PYGZdq{}}
\PYG{g+gp}{\PYGZgt{}\PYGZgt{}\PYGZgt{} }\PYG{n}{p}\PYG{o}{.}\PYG{n}{add}\PYG{p}{(}\PYG{n}{sample\PYGZus{}page}\PYG{p}{)}
\PYG{g+gp}{\PYGZgt{}\PYGZgt{}\PYGZgt{} }\PYG{n+nb}{print}\PYG{p}{(}\PYG{n}{p}\PYG{p}{)}
\PYG{g+go}{[PageContainer]\PYGZhy{}\PYGZhy{}\PYGZhy{}\PYGZgt{}}
\PYG{g+go}{Cantidad máxima de página: 10}
\PYG{g+go}{Total de páginas                 : 1}
\PYG{g+go}{Pagina: 1:Pagina de una sola linea}
\PYG{g+go}{[PageContainer]\PYGZhy{}\PYGZhy{}\PYGZhy{}\PYGZgt{}}
\end{sphinxVerbatim}

\end{fulllineitems}

\index{clear() (método de openerm.PageContainer.PageContainer)@\spxentry{clear()}\spxextra{método de openerm.PageContainer.PageContainer}}

\begin{fulllineitems}
\phantomsection\label{\detokenize{openerm.PageContainer:openerm.PageContainer.PageContainer.clear}}\pysiglinewithargsret{\sphinxbfcode{\sphinxupquote{clear}}}{}{}
Limpia la lista interna de páginas

Ejemplo:

\begin{sphinxVerbatim}[commandchars=\\\{\}]
\PYG{g+gp}{\PYGZgt{}\PYGZgt{}\PYGZgt{} }\PYG{k+kn}{from} \PYG{n+nn}{openerm}\PYG{n+nn}{.}\PYG{n+nn}{PageContainer} \PYG{k}{import} \PYG{n}{PageContainer}
\PYG{g+gp}{\PYGZgt{}\PYGZgt{}\PYGZgt{} }\PYG{n}{p} \PYG{o}{=} \PYG{n}{PageContainer}\PYG{p}{(}\PYG{l+m+mi}{10}\PYG{p}{)}
\PYG{g+gp}{\PYGZgt{}\PYGZgt{}\PYGZgt{} }\PYG{n}{sample\PYGZus{}page} \PYG{o}{=} \PYG{l+s+s2}{\PYGZdq{}}\PYG{l+s+s2}{Pagina de una sola linea}\PYG{l+s+s2}{\PYGZdq{}}
\PYG{g+gp}{\PYGZgt{}\PYGZgt{}\PYGZgt{} }\PYG{n}{p}\PYG{o}{.}\PYG{n}{add}\PYG{p}{(}\PYG{n}{sample\PYGZus{}page}\PYG{p}{)}
\PYG{g+gp}{\PYGZgt{}\PYGZgt{}\PYGZgt{} }\PYG{n+nb}{print}\PYG{p}{(}\PYG{n}{p}\PYG{p}{)}
\PYG{g+go}{[PageContainer]\PYGZhy{}\PYGZhy{}\PYGZhy{}\PYGZgt{}}
\PYG{g+go}{Cantidad máxima de página: 10}
\PYG{g+go}{Total de páginas         : 1}
\PYG{g+go}{Pagina: 1:Pagina de una sola linea}
\PYG{g+go}{[PageContainer]\PYGZhy{}\PYGZhy{}\PYGZhy{}\PYGZgt{}}
\PYG{g+gp}{\PYGZgt{}\PYGZgt{}\PYGZgt{} }\PYG{n}{p}\PYG{o}{.}\PYG{n}{clear}\PYG{p}{(}\PYG{p}{)}
\PYG{g+gp}{\PYGZgt{}\PYGZgt{}\PYGZgt{} }\PYG{n+nb}{print}\PYG{p}{(}\PYG{n}{p}\PYG{p}{)}
\PYG{g+go}{[PageContainer]\PYGZhy{}\PYGZhy{}\PYGZhy{}\PYGZgt{}}
\PYG{g+go}{Cantidad máxima de página: 10}
\PYG{g+go}{Total de páginas         : 0}
\PYG{g+go}{[PageContainer]\PYGZhy{}\PYGZhy{}\PYGZhy{}\PYGZgt{}}
\end{sphinxVerbatim}

\end{fulllineitems}

\index{dump() (método de openerm.PageContainer.PageContainer)@\spxentry{dump()}\spxextra{método de openerm.PageContainer.PageContainer}}

\begin{fulllineitems}
\phantomsection\label{\detokenize{openerm.PageContainer:openerm.PageContainer.PageContainer.dump}}\pysiglinewithargsret{\sphinxbfcode{\sphinxupquote{dump}}}{}{}
Retorna en bytes el contenido de la lista de páginas del grupo. Este método
es utilizado cuando construímos el «Bloque» que posteriormente salvaremos en
un «Database»
\begin{quote}\begin{description}
\item[{Devuelve}] \leavevmode
\begin{itemize}
\item {} 
data (bytes)          : Bytes completos de todas las páginas del grupo

\item {} 
var\_data (bytes)      : Bytes estructurados con los datos para «desarmar» el grupo

\end{itemize}


\item[{Tipo del valor devuelto}] \leavevmode
tuple

\end{description}\end{quote}

Ejemplo:

\begin{sphinxVerbatim}[commandchars=\\\{\}]
\PYG{g+gp}{\PYGZgt{}\PYGZgt{}\PYGZgt{} }\PYG{k+kn}{from} \PYG{n+nn}{openerm} \PYG{k}{import} \PYG{n}{PageContainer}
\PYG{g+gp}{\PYGZgt{}\PYGZgt{}\PYGZgt{} }\PYG{n}{p} \PYG{o}{=} \PYG{n}{PageContainer}\PYG{p}{(}\PYG{l+m+mi}{10}\PYG{p}{)}
\PYG{g+gp}{\PYGZgt{}\PYGZgt{}\PYGZgt{} }\PYG{n}{sample\PYGZus{}page} \PYG{o}{=} \PYG{l+s+s2}{\PYGZdq{}}\PYG{l+s+s2}{Pagina de una sola linea}\PYG{l+s+s2}{\PYGZdq{}}
\PYG{g+gp}{\PYGZgt{}\PYGZgt{}\PYGZgt{} }\PYG{n}{p}\PYG{o}{.}\PYG{n}{add}\PYG{p}{(}\PYG{n}{sample\PYGZus{}page}\PYG{p}{)}
\PYG{g+gp}{\PYGZgt{}\PYGZgt{}\PYGZgt{} }\PYG{p}{(}\PYG{l+s+sa}{b}\PYG{l+s+s1}{\PYGZsq{}}\PYG{l+s+s1}{Pagina de una sola linea}\PYG{l+s+s1}{\PYGZsq{}}\PYG{p}{,} \PYG{l+s+sa}{b}\PYG{l+s+s1}{\PYGZsq{}}\PYG{l+s+s1}{\PYGZbs{}\PYGZob{}\PYGZbs{}\PYGZbs{}\PYGZbs{}}\PYG{l+s+s1}{\PYGZsq{}}\PYG{p}{)}
\end{sphinxVerbatim}

\begin{sphinxadmonition}{note}{Nota:}
El método \sphinxstyleemphasis{dump} entrega una estructura como el siguiente ejemplo:

\begin{sphinxVerbatim}[commandchars=\\\{\}]
\PYG{o}{+}\PYG{o}{==}\PYG{o}{==}\PYG{o}{==}\PYG{o}{==}\PYG{o}{==}\PYG{o}{==}\PYG{o}{=}\PYG{o}{+}
\PYG{o}{\textbar{}}    \PYG{n}{Datos}    \PYG{o}{\textbar{}} \PYG{o}{\PYGZhy{}}\PYG{o}{\PYGZhy{}}\PYG{o}{\PYGZgt{}} \PYG{n}{Comprimibles}
\PYG{o}{+}\PYG{o}{==}\PYG{o}{==}\PYG{o}{==}\PYG{o}{==}\PYG{o}{==}\PYG{o}{==}\PYG{o}{=}\PYG{o}{+}
        \PYG{o}{+}\PYG{o}{==}\PYG{o}{==}\PYG{o}{==}\PYG{o}{==}\PYG{o}{==}\PYG{o}{==}\PYG{o}{=}\PYG{o}{+}
        \PYG{o}{\textbar{}} \PYG{n}{Pagina} \PYG{l+m+mi}{1}    \PYG{o}{\textbar{}}
        \PYG{o}{+}\PYG{o}{==}\PYG{o}{==}\PYG{o}{==}\PYG{o}{==}\PYG{o}{==}\PYG{o}{==}\PYG{o}{=}\PYG{o}{+}
        \PYG{o}{+}\PYG{o}{==}\PYG{o}{==}\PYG{o}{==}\PYG{o}{==}\PYG{o}{==}\PYG{o}{==}\PYG{o}{=}\PYG{o}{+}
        \PYG{o}{\textbar{}} \PYG{n}{Pagina} \PYG{l+s+s2}{\PYGZdq{}}\PYG{l+s+s2}{N}\PYG{l+s+s2}{\PYGZdq{}}  \PYG{o}{\textbar{}}
        \PYG{o}{+}\PYG{o}{==}\PYG{o}{==}\PYG{o}{==}\PYG{o}{==}\PYG{o}{==}\PYG{o}{==}\PYG{o}{=}\PYG{o}{+}
\PYG{o}{+}\PYG{o}{==}\PYG{o}{==}\PYG{o}{==}\PYG{o}{==}\PYG{o}{==}\PYG{o}{==}\PYG{o}{=}\PYG{o}{+}
\PYG{o}{\textbar{}}    \PYG{n}{Datos}    \PYG{o}{\textbar{}} \PYG{o}{\PYGZhy{}}\PYG{o}{\PYGZhy{}}\PYG{o}{\PYGZgt{}} \PYG{n}{No} \PYG{n}{comprimibles}
\PYG{o}{\textbar{}} \PYG{n}{Adicionales} \PYG{o}{\textbar{}}
\PYG{o}{+}\PYG{o}{==}\PYG{o}{==}\PYG{o}{==}\PYG{o}{==}\PYG{o}{==}\PYG{o}{==}\PYG{o}{=}\PYG{o}{+}
        \PYG{o}{+}\PYG{o}{==}\PYG{o}{==}\PYG{o}{==}\PYG{o}{==}\PYG{o}{==}\PYG{o}{==}\PYG{o}{==}\PYG{o}{+}\PYG{o}{==}\PYG{o}{==}\PYG{o}{==}\PYG{o}{==}\PYG{o}{==}\PYG{o}{==}\PYG{o}{==}\PYG{o}{==}\PYG{o}{+}    \PYG{o}{+}\PYG{o}{==}\PYG{o}{==}\PYG{o}{==}\PYG{o}{==}\PYG{o}{==}\PYG{o}{==}\PYG{o}{==}\PYG{o}{==}\PYG{o}{+}
        \PYG{o}{\textbar{}} \PYG{n}{Cant}\PYG{o}{.}\PYG{n}{Paginas} \PYG{o}{\textbar{}} \PYG{n}{Long}\PYG{o}{.} \PYG{n}{Pagina} \PYG{l+m+mi}{1} \PYG{o}{\textbar{}} \PYG{o}{.}\PYG{o}{.} \PYG{o}{\textbar{}} \PYG{n}{Long}\PYG{o}{.} \PYG{n}{Pagina} \PYG{n}{N} \PYG{o}{\textbar{}}
        \PYG{o}{+}\PYG{o}{==}\PYG{o}{==}\PYG{o}{==}\PYG{o}{==}\PYG{o}{==}\PYG{o}{==}\PYG{o}{==}\PYG{o}{+}\PYG{o}{==}\PYG{o}{==}\PYG{o}{==}\PYG{o}{==}\PYG{o}{==}\PYG{o}{==}\PYG{o}{==}\PYG{o}{==}\PYG{o}{+}    \PYG{o}{+}\PYG{o}{==}\PYG{o}{==}\PYG{o}{==}\PYG{o}{==}\PYG{o}{==}\PYG{o}{==}\PYG{o}{==}\PYG{o}{==}\PYG{o}{+}
\end{sphinxVerbatim}
\end{sphinxadmonition}

\end{fulllineitems}

\index{get\_page() (método de openerm.PageContainer.PageContainer)@\spxentry{get\_page()}\spxextra{método de openerm.PageContainer.PageContainer}}

\begin{fulllineitems}
\phantomsection\label{\detokenize{openerm.PageContainer:openerm.PageContainer.PageContainer.get_page}}\pysiglinewithargsret{\sphinxbfcode{\sphinxupquote{get\_page}}}{\emph{pagenum}}{}
Retorna una página determinada del grupo.
\begin{quote}\begin{description}
\item[{Parámetros}] \leavevmode
\sphinxstyleliteralstrong{\sphinxupquote{pagenum}} (\sphinxstyleliteralemphasis{\sphinxupquote{int}}) \textendash{} Número de página a recupear

\item[{Devuelve}] \leavevmode
Pagina o \sphinxtitleref{None}

\item[{Tipo del valor devuelto}] \leavevmode
(string)

\end{description}\end{quote}
\begin{description}
\item[{Ejemplo}] \leavevmode
\begin{sphinxVerbatim}[commandchars=\\\{\}]
\PYG{g+gp}{\PYGZgt{}\PYGZgt{}\PYGZgt{} }\PYG{k+kn}{import} \PYG{n+nn}{string}
\PYG{g+gp}{\PYGZgt{}\PYGZgt{}\PYGZgt{} }\PYG{k+kn}{import} \PYG{n+nn}{random}
\PYG{g+gp}{\PYGZgt{}\PYGZgt{}\PYGZgt{} }\PYG{k+kn}{from} \PYG{n+nn}{openerm}\PYG{n+nn}{.}\PYG{n+nn}{PageContainer} \PYG{k}{import} \PYG{n}{PageContainer}
\PYG{g+gp}{\PYGZgt{}\PYGZgt{}\PYGZgt{} }\PYG{k}{def} \PYG{n+nf}{rnd\PYGZus{}generator}\PYG{p}{(}\PYG{n}{size}\PYG{o}{=}\PYG{l+m+mi}{1024}\PYG{p}{,} \PYG{n}{chars}\PYG{o}{=}\PYG{n}{string}\PYG{o}{.}\PYG{n}{ascii\PYGZus{}uppercase} \PYG{o}{+} \PYG{n}{string}\PYG{o}{.}\PYG{n}{digits}\PYG{p}{)}\PYG{p}{:}
\PYG{g+go}{                return \PYGZsq{}\PYGZsq{}.join(random.choice(chars) for \PYGZus{} in range(size))}
\PYG{g+gp}{\PYGZgt{}\PYGZgt{}\PYGZgt{} }\PYG{n}{pgroup} \PYG{o}{=} \PYG{n}{PageContainer}\PYG{p}{(}\PYG{l+m+mi}{10}\PYG{p}{)}
\PYG{g+gp}{\PYGZgt{}\PYGZgt{}\PYGZgt{} }\PYG{k}{for} \PYG{n}{i} \PYG{o+ow}{in} \PYG{n+nb}{range}\PYG{p}{(}\PYG{l+m+mi}{1}\PYG{p}{,}\PYG{l+m+mi}{11}\PYG{p}{)}\PYG{p}{:}
\PYG{g+go}{                random\PYGZus{}text = rnd\PYGZus{}generator(size=200*60)}
\PYG{g+go}{                pgroup.add(random\PYGZus{}text)}
\PYG{g+gp}{\PYGZgt{}\PYGZgt{}\PYGZgt{} }\PYG{n+nb}{print}\PYG{p}{(}\PYG{n}{pgroup}\PYG{o}{.}\PYG{n}{get\PYGZus{}page}\PYG{p}{(}\PYG{l+m+mi}{5}\PYG{p}{)}\PYG{p}{)}
\end{sphinxVerbatim}

\end{description}

\end{fulllineitems}

\index{load() (método de openerm.PageContainer.PageContainer)@\spxentry{load()}\spxextra{método de openerm.PageContainer.PageContainer}}

\begin{fulllineitems}
\phantomsection\label{\detokenize{openerm.PageContainer:openerm.PageContainer.PageContainer.load}}\pysiglinewithargsret{\sphinxbfcode{\sphinxupquote{load}}}{\emph{container\_data}}{}
Recupera de un conjunto de bytes cada una de las páginas y construye la lista de las mismas
\begin{quote}\begin{description}
\item[{Parámetros}] \leavevmode
\sphinxstyleliteralstrong{\sphinxupquote{container\_data}} (\sphinxstyleliteralemphasis{\sphinxupquote{bytes}}) \textendash{} Bytes completos de todas las páginas del grupo. Ver {\hyperref[\detokenize{openerm.PageContainer:openerm.PageContainer.PageContainer.dump}]{\sphinxcrossref{\sphinxcode{\sphinxupquote{dump()}}}}}

\end{description}\end{quote}

Ejemplo:

\begin{sphinxVerbatim}[commandchars=\\\{\}]
\PYG{g+gp}{\PYGZgt{}\PYGZgt{}\PYGZgt{} }\PYG{k+kn}{from} \PYG{n+nn}{openerm} \PYG{k}{import} \PYG{n}{PageContainer}
\PYG{g+gp}{\PYGZgt{}\PYGZgt{}\PYGZgt{} }\PYG{n}{p} \PYG{o}{=} \PYG{n}{PageContainer}\PYG{p}{(}\PYG{l+m+mi}{10}\PYG{p}{)}
\PYG{g+gp}{\PYGZgt{}\PYGZgt{}\PYGZgt{} }\PYG{n}{data} \PYG{o}{=} \PYG{p}{(}\PYG{l+s+sa}{b}\PYG{l+s+s1}{\PYGZsq{}}\PYG{l+s+s1}{Pagina de una sola linea}\PYG{l+s+s1}{\PYGZsq{}}\PYG{p}{,} \PYG{l+s+sa}{b}\PYG{l+s+s1}{\PYGZsq{}}\PYG{l+s+s1}{\PYGZbs{}\PYGZob{}\PYGZbs{}\PYGZbs{}\PYGZbs{}}\PYG{l+s+s1}{\PYGZsq{}}\PYG{p}{)}
\PYG{g+gp}{\PYGZgt{}\PYGZgt{}\PYGZgt{} }\PYG{n}{p}\PYG{o}{.}\PYG{n}{load}\PYG{p}{(}\PYG{n}{data}\PYG{p}{)}
\end{sphinxVerbatim}

\end{fulllineitems}


\end{fulllineitems}



\subsection{Internas - Core}
\label{\detokenize{openerm:internas-core}}\phantomsection\label{\detokenize{openerm.Block:block}}\phantomsection\label{\detokenize{openerm.Block:module-openerm.Block}}\phantomsection\label{\detokenize{openerm.Block:block}}\index{openerm.Block (módulo)@\spxentry{openerm.Block}\spxextra{módulo}}

\subsubsection{Block}
\label{\detokenize{openerm.Block:id1}}\label{\detokenize{openerm.Block::doc}}
Un «bloque» de datos. Un Block es la unidad miníma de un Database OERM. Hay dos
tipos de bloque básico:
\begin{itemize}
\item {} \begin{description}
\item[{{\hyperref[\detokenize{openerm.MetadataContainer:module-openerm.MetadataContainer}]{\sphinxcrossref{\sphinxcode{\sphinxupquote{openerm.MetadataContainer}}}}} para guardar los atributos de un}] \leavevmode
determinado reporte.

\end{description}

\item {} 
{\hyperref[\detokenize{openerm.PageContainer:module-openerm.PageContainer}]{\sphinxcrossref{\sphinxcode{\sphinxupquote{openerm.PageContainer}}}}} para guardar conjuntos de páginas

\end{itemize}


\sphinxstrong{Ver también:}

\begin{itemize}
\item {} 
{\hyperref[\detokenize{openerm.Database:module-openerm.Database}]{\sphinxcrossref{\sphinxcode{\sphinxupquote{openerm.Database}}}}}

\item {} 
{\hyperref[\detokenize{openerm.Compressor:module-openerm.Compressor}]{\sphinxcrossref{\sphinxcode{\sphinxupquote{openerm.Compressor}}}}}

\item {} 
{\hyperref[\detokenize{openerm.Cipher:module-openerm.Cipher}]{\sphinxcrossref{\sphinxcode{\sphinxupquote{openerm.Cipher}}}}}

\end{itemize}


\index{Block (clase en openerm.Block)@\spxentry{Block}\spxextra{clase en openerm.Block}}

\begin{fulllineitems}
\phantomsection\label{\detokenize{openerm.Block:openerm.Block.Block}}\pysiglinewithargsret{\sphinxbfcode{\sphinxupquote{class }}\sphinxcode{\sphinxupquote{openerm.Block.}}\sphinxbfcode{\sphinxupquote{Block}}}{\emph{default\_compress\_method=1}, \emph{default\_compress\_level=1}, \emph{default\_encription\_method=0}}{}
Bases: \sphinxcode{\sphinxupquote{object}}

Bloque de un archivo OpenERM. El bloque es la unidad mínima de
información. Existen en está versión dos tipo de bloques básicos:
\begin{itemize}
\item {} 
Contenedores de metadatos

\item {} 
Contenedores de páginas

\end{itemize}

Cada bloque puede o no ser comprimido y encriptado con algún algoritmo que
dependerá de la implememntación OpenERM particular.
\begin{quote}\begin{description}
\item[{Parámetros}] \leavevmode\begin{itemize}
\item {} 
\sphinxstyleliteralstrong{\sphinxupquote{default\_compress\_method}} (\sphinxstyleliteralemphasis{\sphinxupquote{int}}) \textendash{} Compresión por defecto del bloque (default = 1 - Gzip)

\item {} 
\sphinxstyleliteralstrong{\sphinxupquote{default\_compress\_level}} (\sphinxstyleliteralemphasis{\sphinxupquote{int}}) \textendash{} Nivel de compresión por defecto del bloque (default = 1 Medium)

\item {} 
\sphinxstyleliteralstrong{\sphinxupquote{default\_encription\_method}} (\sphinxstyleliteralemphasis{\sphinxupquote{int}}) \textendash{} Algoritmo de cifrado (0 = Ninguno)

\end{itemize}

\end{description}\end{quote}
\index{block\_types (atributo de openerm.Block.Block)@\spxentry{block\_types}\spxextra{atributo de openerm.Block.Block}}

\begin{fulllineitems}
\phantomsection\label{\detokenize{openerm.Block:openerm.Block.Block.block_types}}\pysigline{\sphinxbfcode{\sphinxupquote{block\_types}}\sphinxbfcode{\sphinxupquote{ = None}}}
Tipos de bloque
1 - Report metadata
2 - Pages container

\end{fulllineitems}

\index{dump() (método de openerm.Block.Block)@\spxentry{dump()}\spxextra{método de openerm.Block.Block}}

\begin{fulllineitems}
\phantomsection\label{\detokenize{openerm.Block:openerm.Block.Block.dump}}\pysiglinewithargsret{\sphinxbfcode{\sphinxupquote{dump}}}{\emph{tipo\_bloque}, \emph{data}, \emph{variable\_data=None}}{}
Convierte los datos en un bloque OpenErm que puede ser «salvable» en un archivo
\begin{quote}\begin{description}
\item[{Parámetros}] \leavevmode\begin{itemize}
\item {} 
\sphinxstyleliteralstrong{\sphinxupquote{tipo\_bloque}} (\sphinxstyleliteralemphasis{\sphinxupquote{int}}) \textendash{} Tipo de bloqe (1: metadatos, 2: páginas)

\item {} 
\sphinxstyleliteralstrong{\sphinxupquote{data}} (\sphinxstyleliteralemphasis{\sphinxupquote{bytes}}) \textendash{} Bytes de los datos a salvar

\item {} 
\sphinxstyleliteralstrong{\sphinxupquote{variable\_data}} (\sphinxstyleliteralemphasis{\sphinxupquote{bytes}}) \textendash{} (opcional) Datos adicionales, es información que se salva en el bloque pero no se comprime

\end{itemize}

\end{description}\end{quote}
\subsubsection*{Ejemplo}

\begin{sphinxVerbatim}[commandchars=\\\{\}]
\PYG{g+gp}{\PYGZgt{}\PYGZgt{}\PYGZgt{} }\PYG{k+kn}{from} \PYG{n+nn}{openerm}\PYG{n+nn}{.}\PYG{n+nn}{Block} \PYG{k}{import} \PYG{n}{Block}
\PYG{g+gp}{\PYGZgt{}\PYGZgt{}\PYGZgt{} }\PYG{n}{b} \PYG{o}{=} \PYG{n}{Block}\PYG{p}{(}\PYG{n}{default\PYGZus{}compress\PYGZus{}method}\PYG{o}{=}\PYG{l+m+mi}{1}\PYG{p}{,} \PYG{n}{default\PYGZus{}compress\PYGZus{}level}\PYG{o}{=}\PYG{l+m+mi}{1}\PYG{p}{,} \PYG{n}{default\PYGZus{}encription\PYGZus{}method}\PYG{o}{=}\PYG{l+m+mi}{0}\PYG{p}{)}
\PYG{g+gp}{\PYGZgt{}\PYGZgt{}\PYGZgt{} }\PYG{n}{data} \PYG{o}{=} \PYG{n}{b}\PYG{o}{.}\PYG{n}{dump}\PYG{p}{(}\PYG{n}{tipo\PYGZus{}bloque}\PYG{o}{=}\PYG{l+m+mi}{2}\PYG{p}{,} \PYG{n}{data}\PYG{o}{=}\PYG{l+s+sa}{b}\PYG{l+s+s2}{\PYGZdq{}}\PYG{l+s+s2}{Esto hace las veces de una pagina de reporte}\PYG{l+s+s2}{\PYGZdq{}}\PYG{p}{,}\PYG{n}{variable\PYGZus{}data}\PYG{o}{=}\PYG{k+kc}{None}\PYG{p}{)}
\PYG{g+gp}{\PYGZgt{}\PYGZgt{}\PYGZgt{} }\PYG{n+nb}{print}\PYG{p}{(}\PYG{n}{data}\PYG{p}{)}
\PYG{g+go}{b\PYGZsq{}\PYGZbs{}\PYGZbs{}\PYGZbs{}\PYGZlt{}\PYGZcb{}\PYGZob{}\PYGZbs{}\PYGZbs{}\PYGZbs{}\PYGZbs{}1xœs\PYGZhy{}.ÉWÈHLNUÈI,V(KMN\PYGZhy{}VHIU(ÍKT(HLÏR@\PYGZca{}QjA\PYGZti{}QI*\PYGZbs{}]Ê\PYGZsq{}}
\end{sphinxVerbatim}

\begin{sphinxadmonition}{note}{Nota:}
El método \sphinxstyleemphasis{dump} entrega una estructura como el siguiente ejemplo:

\begin{sphinxVerbatim}[commandchars=\\\{\}]
\PYG{o}{+}\PYG{o}{==}\PYG{o}{==}\PYG{o}{==}\PYG{o}{==}\PYG{o}{==}\PYG{o}{==}\PYG{o}{==}\PYG{o}{==}\PYG{o}{==}\PYG{o}{==}\PYG{o}{==}\PYG{o}{=}\PYG{o}{+}
\PYG{o}{\textbar{}} \PYG{n}{Long}\PYG{o}{.}\PYG{n}{Total} \PYG{k}{del} \PYG{n}{Bloque} \PYG{o}{\textbar{}}   \PYG{o}{\PYGZhy{}}\PYG{o}{\PYGZhy{}}\PYG{o}{\PYGZgt{}} \PYG{n+nb}{long} \PYG{p}{(}\PYG{l+m+mi}{4} \PYG{n+nb}{bytes}\PYG{p}{)}
\PYG{o}{+}\PYG{o}{==}\PYG{o}{==}\PYG{o}{==}\PYG{o}{==}\PYG{o}{==}\PYG{o}{==}\PYG{o}{==}\PYG{o}{==}\PYG{o}{==}\PYG{o}{==}\PYG{o}{==}\PYG{o}{=}\PYG{o}{+}
\PYG{o}{\textbar{}} \PYG{n}{Tipo} \PYG{n}{de} \PYG{n}{Bloque}  \PYG{o}{\textbar{}}         \PYG{o}{\PYGZhy{}}\PYG{o}{\PYGZhy{}}\PYG{o}{\PYGZgt{}} \PYG{n+nb}{int} \PYG{p}{(}\PYG{l+m+mi}{1} \PYG{n+nb}{bytes}\PYG{p}{)}
\PYG{o}{+}\PYG{o}{==}\PYG{o}{==}\PYG{o}{==}\PYG{o}{==}\PYG{o}{==}\PYG{o}{==}\PYG{o}{==}\PYG{o}{==}\PYG{o}{=}\PYG{o}{+}
\PYG{o}{\textbar{}} \PYG{n}{Alg}\PYG{o}{.} \PYG{n}{Compresión} \PYG{o}{\textbar{}}         \PYG{o}{\PYGZhy{}}\PYG{o}{\PYGZhy{}}\PYG{o}{\PYGZgt{}} \PYG{n+nb}{int} \PYG{p}{(}\PYG{l+m+mi}{1} \PYG{n+nb}{bytes}\PYG{p}{)}
\PYG{o}{+}\PYG{o}{==}\PYG{o}{==}\PYG{o}{==}\PYG{o}{==}\PYG{o}{==}\PYG{o}{==}\PYG{o}{==}\PYG{o}{==}\PYG{o}{=}\PYG{o}{+}
\PYG{o}{\textbar{}} \PYG{n}{Alg}\PYG{o}{.} \PYG{n}{Cifrado}    \PYG{o}{\textbar{}}         \PYG{o}{\PYGZhy{}}\PYG{o}{\PYGZhy{}}\PYG{o}{\PYGZgt{}} \PYG{n+nb}{int} \PYG{p}{(}\PYG{l+m+mi}{1} \PYG{n+nb}{bytes}\PYG{p}{)}
\PYG{o}{+}\PYG{o}{==}\PYG{o}{==}\PYG{o}{==}\PYG{o}{==}\PYG{o}{==}\PYG{o}{==}\PYG{o}{==}\PYG{o}{==}\PYG{o}{==}\PYG{o}{==}\PYG{o}{==}\PYG{o}{=}\PYG{o}{+}
\PYG{o}{\textbar{}} \PYG{n}{Long}\PYG{o}{.} \PYG{n}{de} \PYG{n}{los} \PYG{n}{Datos}    \PYG{o}{\textbar{}}   \PYG{o}{\PYGZhy{}}\PYG{o}{\PYGZhy{}}\PYG{o}{\PYGZgt{}} \PYG{n+nb}{long} \PYG{p}{(}\PYG{l+m+mi}{4} \PYG{n+nb}{bytes}\PYG{p}{)}
\PYG{o}{+}\PYG{o}{==}\PYG{o}{==}\PYG{o}{==}\PYG{o}{==}\PYG{o}{==}\PYG{o}{==}\PYG{o}{==}\PYG{o}{==}\PYG{o}{==}\PYG{o}{==}\PYG{o}{==}\PYG{o}{=}\PYG{o}{+}
\PYG{o}{\textbar{}}                       \PYG{o}{\textbar{}}
\PYG{o}{\textbar{}}        \PYG{n}{Datos}          \PYG{o}{\textbar{}}   \PYG{o}{\PYGZhy{}}\PYG{o}{\PYGZhy{}}\PYG{o}{\PYGZgt{}} \PYG{n}{Longitud} \PYG{n}{variable} \PYG{p}{(}\PYG{n}{datos} \PYG{n}{comprimibles}\PYG{p}{)}
\PYG{o}{\textbar{}}                       \PYG{o}{\textbar{}}
\PYG{o}{+}\PYG{o}{==}\PYG{o}{==}\PYG{o}{==}\PYG{o}{==}\PYG{o}{==}\PYG{o}{==}\PYG{o}{==}\PYG{o}{==}\PYG{o}{==}\PYG{o}{==}\PYG{o}{==}\PYG{o}{=}\PYG{o}{+}
\PYG{o}{\textbar{}}                       \PYG{o}{\textbar{}}
\PYG{o}{\textbar{}}    \PYG{n}{Datos} \PYG{n}{variables}    \PYG{o}{\textbar{}}   \PYG{o}{\PYGZhy{}}\PYG{o}{\PYGZhy{}}\PYG{o}{\PYGZgt{}} \PYG{p}{(}\PYG{n}{Opcional}\PYG{p}{)} \PYG{n}{Longitud} \PYG{n}{variable} \PYG{p}{(}\PYG{n}{datos} \PYG{n}{NO} \PYG{n}{comprimibles}\PYG{p}{)}
\PYG{o}{\textbar{}}                       \PYG{o}{\textbar{}}
\PYG{o}{+}\PYG{o}{==}\PYG{o}{==}\PYG{o}{==}\PYG{o}{==}\PYG{o}{==}\PYG{o}{==}\PYG{o}{==}\PYG{o}{==}\PYG{o}{==}\PYG{o}{==}\PYG{o}{==}\PYG{o}{=}\PYG{o}{+}
\end{sphinxVerbatim}
\end{sphinxadmonition}

\end{fulllineitems}

\index{load() (método de openerm.Block.Block)@\spxentry{load()}\spxextra{método de openerm.Block.Block}}

\begin{fulllineitems}
\phantomsection\label{\detokenize{openerm.Block:openerm.Block.Block.load}}\pysiglinewithargsret{\sphinxbfcode{\sphinxupquote{load}}}{\emph{data}}{}
Convierte un bloque OpenErm en datos lógicos
\begin{quote}\begin{description}
\item[{Parámetros}] \leavevmode
\sphinxstyleliteralstrong{\sphinxupquote{data}} (\sphinxstyleliteralemphasis{\sphinxupquote{bytes}}) \textendash{} Bytes del bloque completo

\item[{Devuelve}] \leavevmode
tuple

\end{description}\end{quote}

Se devuelve una «tupla» de 7 elementos:


\begin{savenotes}\sphinxattablestart
\centering
\begin{tabulary}{\linewidth}[t]{|T|T|}
\hline
\sphinxstyletheadfamily 
Tipo
&\sphinxstyletheadfamily 
Detalle
\\
\hline
long
&
Longitud total del bloque incluído este dato
\\
\hline
int
&
Tipo de bloque
\\
\hline
int
&
Tipo de compresión
\\
\hline
int
&
Tipo de encriptado
\\
\hline
long
&
longitud de los datos comprimidos
\\
\hline
bytes
&
Datos
\\
\hline
bytes
&
Datos adicionales
\\
\hline
\end{tabulary}
\par
\sphinxattableend\end{savenotes}
\subsubsection*{Ejemplo}

\begin{sphinxVerbatim}[commandchars=\\\{\}]
\PYG{g+gp}{\PYGZgt{}\PYGZgt{}\PYGZgt{} }\PYG{k+kn}{from} \PYG{n+nn}{openerm}\PYG{n+nn}{.}\PYG{n+nn}{Block} \PYG{k}{import} \PYG{n}{Block}
\PYG{g+gp}{\PYGZgt{}\PYGZgt{}\PYGZgt{} }\PYG{n}{b} \PYG{o}{=} \PYG{n}{Block}\PYG{p}{(}\PYG{n}{default\PYGZus{}compress\PYGZus{}method}\PYG{o}{=}\PYG{l+m+mi}{1}\PYG{p}{,} \PYG{n}{default\PYGZus{}compress\PYGZus{}level}\PYG{o}{=}\PYG{l+m+mi}{1}\PYG{p}{,} \PYG{n}{default\PYGZus{}encription\PYGZus{}method}\PYG{o}{=}\PYG{l+m+mi}{0}\PYG{p}{)}
\PYG{g+gp}{\PYGZgt{}\PYGZgt{}\PYGZgt{} }\PYG{n}{data} \PYG{o}{=} \PYG{n}{b}\PYG{o}{.}\PYG{n}{dump}\PYG{p}{(}\PYG{n}{tipo\PYGZus{}bloque}\PYG{o}{=}\PYG{l+m+mi}{2}\PYG{p}{,} \PYG{n}{data}\PYG{o}{=}\PYG{l+s+sa}{b}\PYG{l+s+s2}{\PYGZdq{}}\PYG{l+s+s2}{Esto hace las veces de una pagina de reporte}\PYG{l+s+s2}{\PYGZdq{}}\PYG{p}{,}\PYG{n}{variable\PYGZus{}data}\PYG{o}{=}\PYG{k+kc}{None}\PYG{p}{)}
\PYG{g+gp}{\PYGZgt{}\PYGZgt{}\PYGZgt{} }\PYG{n+nb}{print}\PYG{p}{(}\PYG{n}{data}\PYG{p}{)}
\PYG{g+go}{b\PYGZsq{}\PYGZbs{}\PYGZbs{}\PYGZbs{}\PYGZlt{}\PYGZcb{}\PYGZob{}\PYGZbs{}\PYGZbs{}\PYGZbs{}\PYGZbs{}1xœs\PYGZhy{}.ÉWÈHLNUÈI,V(KMN\PYGZhy{}VHIU(ÍKT(HLÏR@\PYGZca{}QjA\PYGZti{}QI*\PYGZbs{}]Ê\PYGZsq{}}
\PYG{g+gp}{\PYGZgt{}\PYGZgt{}\PYGZgt{} }\PYG{n+nb}{print}\PYG{p}{(}\PYG{n}{b}\PYG{o}{.}\PYG{n}{load}\PYG{p}{(}\PYG{n}{data}\PYG{p}{)}\PYG{p}{)}
\PYG{g+go}{(60, 2, 1, 0, 49, b\PYGZsq{}Esto hace las veces de una pagina de reporte\PYGZsq{}, None)}
\end{sphinxVerbatim}

\end{fulllineitems}


\end{fulllineitems}



\subsubsection{Cipher}
\label{\detokenize{openerm.Cipher:module-openerm.Cipher}}\label{\detokenize{openerm.Cipher:cipher}}\label{\detokenize{openerm.Cipher::doc}}\index{openerm.Cipher (módulo)@\spxentry{openerm.Cipher}\spxextra{módulo}}\index{Cipher (clase en openerm.Cipher)@\spxentry{Cipher}\spxextra{clase en openerm.Cipher}}

\begin{fulllineitems}
\phantomsection\label{\detokenize{openerm.Cipher:openerm.Cipher.Cipher}}\pysiglinewithargsret{\sphinxbfcode{\sphinxupquote{class }}\sphinxcode{\sphinxupquote{openerm.Cipher.}}\sphinxbfcode{\sphinxupquote{Cipher}}}{\emph{cipher\_type=0}}{}
Bases: \sphinxcode{\sphinxupquote{object}}

Clase base para el manejo de cifrado de «bytes».

Esta es una clase base para poder configurar distintos algoritmos de
cifrado, pero que a la vez ofrezca una interfaz común para cifrar
y descifrar.
\begin{quote}\begin{description}
\item[{Parámetros}] \leavevmode
\sphinxstyleliteralstrong{\sphinxupquote{cipher\_type}} (\sphinxstyleliteralemphasis{\sphinxupquote{int}}) \textendash{} Opcional, Tipo de cifrado por defecto (default=0-Ninguno)

\end{description}\end{quote}
\subsubsection*{Ejemplo}

\begin{sphinxVerbatim}[commandchars=\\\{\}]
\PYG{g+gp}{\PYGZgt{}\PYGZgt{}\PYGZgt{} }\PYG{k+kn}{from} \PYG{n+nn}{openerm}\PYG{n+nn}{.}\PYG{n+nn}{Cipher} \PYG{k}{import} \PYG{n}{Cipher}
\PYG{g+gp}{\PYGZgt{}\PYGZgt{}\PYGZgt{} }\PYG{n}{c} \PYG{o}{=} \PYG{n}{Cipher}\PYG{p}{(}\PYG{n+nb}{type}\PYG{o}{=}\PYG{l+m+mi}{2}\PYG{p}{)}
\PYG{g+gp}{\PYGZgt{}\PYGZgt{}\PYGZgt{} }\PYG{n}{tmp} \PYG{o}{=} \PYG{n}{c}\PYG{o}{.}\PYG{n}{encode}\PYG{p}{(}\PYG{l+s+sa}{b}\PYG{l+s+s1}{\PYGZsq{}}\PYG{l+s+s1}{esto es una prueba}\PYG{l+s+s1}{\PYGZsq{}}\PYG{p}{)}
\PYG{g+gp}{\PYGZgt{}\PYGZgt{}\PYGZgt{} }\PYG{n+nb}{print}\PYG{p}{(}\PYG{n}{tmp}\PYG{p}{)}
\PYG{g+go}{b\PYGZsq{}gAAAAABX2dYtQhe7Th3sA24606o\PYGZus{}747bts4n11Jm37gHW5SIRanP105loH4jPBJzEPrlBmhb9ai5FcXIBhTtUswHA\PYGZus{}H6yrzgVK0CPDig0iSKQjyfaWJryJI=\PYGZsq{}}
\end{sphinxVerbatim}
\index{available\_types (atributo de openerm.Cipher.Cipher)@\spxentry{available\_types}\spxextra{atributo de openerm.Cipher.Cipher}}

\begin{fulllineitems}
\phantomsection\label{\detokenize{openerm.Cipher:openerm.Cipher.Cipher.available_types}}\pysigline{\sphinxbfcode{\sphinxupquote{available\_types}}}
Retorna los tipos de cifrados disponibles.
\begin{quote}\begin{description}
\item[{Devuelve}] \leavevmode
Lista de algoritmos cifrados.

\item[{Tipo del valor devuelto}] \leavevmode
list

\end{description}\end{quote}
\subsubsection*{Ejemplo}

\begin{sphinxVerbatim}[commandchars=\\\{\}]
\PYG{g+gp}{\PYGZgt{}\PYGZgt{}\PYGZgt{} }\PYG{k+kn}{from} \PYG{n+nn}{openerm}\PYG{n+nn}{.}\PYG{n+nn}{Cipher} \PYG{k}{import} \PYG{n}{Cipher}
\PYG{g+gp}{\PYGZgt{}\PYGZgt{}\PYGZgt{} }\PYG{n}{c} \PYG{o}{=} \PYG{n}{Cipher}\PYG{p}{(}\PYG{p}{)}
\PYG{g+gp}{\PYGZgt{}\PYGZgt{}\PYGZgt{} }\PYG{n}{c}\PYG{o}{.}\PYG{n}{available\PYGZus{}types}
\PYG{g+go}{[(0, \PYGZsq{}Sin encriptación\PYGZsq{}), (1, \PYGZsq{}Spritz\PYGZsq{}), (2, \PYGZsq{}Fernet\PYGZsq{})]}
\end{sphinxVerbatim}

\end{fulllineitems}

\index{decode() (método de openerm.Cipher.Cipher)@\spxentry{decode()}\spxextra{método de openerm.Cipher.Cipher}}

\begin{fulllineitems}
\phantomsection\label{\detokenize{openerm.Cipher:openerm.Cipher.Cipher.decode}}\pysiglinewithargsret{\sphinxbfcode{\sphinxupquote{decode}}}{\emph{data}}{}
Descifra un conjunto de bytes
\begin{quote}\begin{description}
\item[{Parámetros}] \leavevmode
\sphinxstyleliteralstrong{\sphinxupquote{data}} (\sphinxstyleliteralemphasis{\sphinxupquote{bytes}}) \textendash{} conjunto de bytes cifrados

\item[{Devuelve}] \leavevmode
datos descifrados.

\item[{Tipo del valor devuelto}] \leavevmode
bytes

\end{description}\end{quote}
\subsubsection*{Ejemplo}

\begin{sphinxVerbatim}[commandchars=\\\{\}]
\PYG{g+gp}{\PYGZgt{}\PYGZgt{}\PYGZgt{} }\PYG{k+kn}{from} \PYG{n+nn}{openerm}\PYG{n+nn}{.}\PYG{n+nn}{Cipher} \PYG{k}{import} \PYG{n}{Cipher}
\PYG{g+gp}{\PYGZgt{}\PYGZgt{}\PYGZgt{} }\PYG{n}{c} \PYG{o}{=} \PYG{n}{Cipher}\PYG{p}{(}\PYG{n+nb}{type}\PYG{o}{=}\PYG{l+m+mi}{2}\PYG{p}{)}
\PYG{g+gp}{\PYGZgt{}\PYGZgt{}\PYGZgt{} }\PYG{n}{tmp} \PYG{o}{=} \PYG{n}{c}\PYG{o}{.}\PYG{n}{encode}\PYG{p}{(}\PYG{l+s+sa}{b}\PYG{l+s+s1}{\PYGZsq{}}\PYG{l+s+s1}{esto es una prueba}\PYG{l+s+s1}{\PYGZsq{}}\PYG{p}{)}
\PYG{g+gp}{\PYGZgt{}\PYGZgt{}\PYGZgt{} }\PYG{n+nb}{print}\PYG{p}{(}\PYG{n}{tmp}\PYG{p}{)}
\PYG{g+go}{b\PYGZsq{}gAAAAABX2dYtQhe7Th3sA24606o\PYGZus{}747bts4n11Jm37gHW5SIRanP105loH4jPBJzEPrlBmhb9ai5FcXIBhTtUswHA\PYGZus{}H6yrzgVK0CPDig0iSKQjyfaWJryJI=\PYGZsq{}}
\PYG{g+gp}{\PYGZgt{}\PYGZgt{}\PYGZgt{} }\PYG{n+nb}{print}\PYG{p}{(}\PYG{n}{c}\PYG{o}{.}\PYG{n}{decode}\PYG{p}{(}\PYG{n}{tmp}\PYG{p}{)}\PYG{p}{)}
\PYG{g+go}{b\PYGZsq{}esto es una prueba\PYGZsq{}}
\end{sphinxVerbatim}

\end{fulllineitems}

\index{encode() (método de openerm.Cipher.Cipher)@\spxentry{encode()}\spxextra{método de openerm.Cipher.Cipher}}

\begin{fulllineitems}
\phantomsection\label{\detokenize{openerm.Cipher:openerm.Cipher.Cipher.encode}}\pysiglinewithargsret{\sphinxbfcode{\sphinxupquote{encode}}}{\emph{data}}{}
Cifra conjunto de bytes
\begin{quote}\begin{description}
\item[{Parámetros}] \leavevmode
\sphinxstyleliteralstrong{\sphinxupquote{data}} (\sphinxstyleliteralemphasis{\sphinxupquote{bytes}}) \textendash{} conjunto de bytes a cifrar

\item[{Devuelve}] \leavevmode
datos cifrados.

\item[{Tipo del valor devuelto}] \leavevmode
bytes

\end{description}\end{quote}
\subsubsection*{Ejemplo}

\begin{sphinxVerbatim}[commandchars=\\\{\}]
\PYG{g+gp}{\PYGZgt{}\PYGZgt{}\PYGZgt{} }\PYG{k+kn}{from} \PYG{n+nn}{openerm}\PYG{n+nn}{.}\PYG{n+nn}{Cipher} \PYG{k}{import} \PYG{n}{Cipher}
\PYG{g+gp}{\PYGZgt{}\PYGZgt{}\PYGZgt{} }\PYG{n}{c} \PYG{o}{=} \PYG{n}{Cipher}\PYG{p}{(}\PYG{n+nb}{type}\PYG{o}{=}\PYG{l+m+mi}{2}\PYG{p}{)}
\PYG{g+gp}{\PYGZgt{}\PYGZgt{}\PYGZgt{} }\PYG{n}{tmp} \PYG{o}{=} \PYG{n}{c}\PYG{o}{.}\PYG{n}{encode}\PYG{p}{(}\PYG{l+s+sa}{b}\PYG{l+s+s1}{\PYGZsq{}}\PYG{l+s+s1}{esto es una prueba}\PYG{l+s+s1}{\PYGZsq{}}\PYG{p}{)}
\PYG{g+gp}{\PYGZgt{}\PYGZgt{}\PYGZgt{} }\PYG{n+nb}{print}\PYG{p}{(}\PYG{n}{tmp}\PYG{p}{)}
\PYG{g+go}{b\PYGZsq{}gAAAAABX2dYtQhe7Th3sA24606o\PYGZus{}747bts4n11Jm37gHW5SIRanP105loH4jPBJzEPrlBmhb9ai5FcXIBhTtUswHA\PYGZus{}H6yrzgVK0CPDig0iSKQjyfaWJryJI=\PYGZsq{}}
\end{sphinxVerbatim}

\end{fulllineitems}

\index{type (atributo de openerm.Cipher.Cipher)@\spxentry{type}\spxextra{atributo de openerm.Cipher.Cipher}}

\begin{fulllineitems}
\phantomsection\label{\detokenize{openerm.Cipher:openerm.Cipher.Cipher.type}}\pysigline{\sphinxbfcode{\sphinxupquote{type}}}
Tipo de cifrado.

\end{fulllineitems}

\index{type\_info() (método de openerm.Cipher.Cipher)@\spxentry{type\_info()}\spxextra{método de openerm.Cipher.Cipher}}

\begin{fulllineitems}
\phantomsection\label{\detokenize{openerm.Cipher:openerm.Cipher.Cipher.type_info}}\pysiglinewithargsret{\sphinxbfcode{\sphinxupquote{type\_info}}}{\emph{cipher\_type}}{}
Retorna la información de un determinado algoritmo de cifrado disponible.

Returns:
(tupla).

\end{fulllineitems}


\end{fulllineitems}

\phantomsection\label{\detokenize{openerm.Compressor:compressor}}\phantomsection\label{\detokenize{openerm.Compressor:module-openerm.Compressor}}\phantomsection\label{\detokenize{openerm.Compressor:compressor}}\index{openerm.Compressor (módulo)@\spxentry{openerm.Compressor}\spxextra{módulo}}

\subsubsection{Compressor}
\label{\detokenize{openerm.Compressor:id1}}\label{\detokenize{openerm.Compressor::doc}}
Esta clase gestiona los algoritmos de compresión que se pueden utilizarse en un
{\hyperref[\detokenize{openerm.Database:module-openerm.Database}]{\sphinxcrossref{\sphinxcode{\sphinxupquote{openerm.Database}}}}} OERM. Este objeto ofrece dos rutinas básicas:
\begin{itemize}
\item {} 
\sphinxcode{\sphinxupquote{compress()}}

\item {} 
\sphinxcode{\sphinxupquote{decompress()}}

\end{itemize}

y dos posibles formas de configurar estos metódos en la inicialización del
objeto:
\begin{itemize}
\item {} 
\sphinxstylestrong{type}: Tipo de compresión

\item {} 
\sphinxstylestrong{level}: Nivel de compresión  0=mínimo/rápido, 1=normal/estándar,2=máximo/lento

\end{itemize}

\#\# Tipos/Algoritmos de compresión


\begin{savenotes}\sphinxattablestart
\centering
\begin{tabulary}{\linewidth}[t]{|T|T|}
\hline
\sphinxstyletheadfamily 
Tipo
&\sphinxstyletheadfamily 
Detalle
\\
\hline
0
&
Sin compresión
\\
\hline
1
&
\sphinxstylestrong{Gzip} es una abreviatura de GNU ZIP, un software libre GNU que reemplaza al programa compress de UNIX. gzip fue creado por Jean-loup Gailly y Mark Adler. Apareció el 31 de octubre de 1992 (versión 0.1). La versión 1.0 apareció en febrero de 1993. Gzip se basa en el algoritmo Deflate, que es una combinación del LZ77 y la codificación Huffman. Deflate se desarrolló como respuesta a las patentes que cubrieron LZW y otros algoritmos de compresión y limitaba el uso del compress.
\\
\hline
2
&
\sphinxstylestrong{Bzip2} es un programa libre desarrollado bajo licencia BSD que comprime y descomprime ficheros usando los algoritmos de compresión de Burrows-Wheeler y de codificación de Huffman. El porcentaje de compresión alcanzado depende del contenido del fichero a comprimir, pero por lo general es bastante mejor al de los compresores basados en el algoritmo LZ77/LZ78 (gzip, compress, WinZip, pkzip,…). Como contrapartida, bzip2 emplea más memoria y más tiempo en su ejecución.
\\
\hline
3
&
\sphinxstylestrong{LZMA} El Algoritmo de cadena de Lempel-Ziv-Markov o LZMA es un algoritmo de compresión de datos en desarrollo desde 1998. Se utiliza un esquema de compresión diccionario algo similar a LZ77, cuenta con una alta relación de compresión y una compresión de tamaño variable diccionario (de hasta 4 GB). Se utiliza en el formato 7z del archivador 7-Zip.
\\
\hline
4
&
\sphinxstylestrong{LZ4} is lossless compression algorithm, providing compression speed at 400 MB/s per core (0.16 Bytes/cycle). It features an extremely fast decoder, with speed in multiple GB/s per core (0.71 Bytes/cycle). A high compression derivative, called LZ4\_HC, is available, trading customizable CPU time for compression ratio. LZ4 library is provided as open source software using a BSD license.
\\
\hline
5
&
\sphinxstylestrong{pyLZMA}, otra implementación de \sphinxstylestrong{LZMA}
\\
\hline
6
&
\sphinxstylestrong{BLOSC} an extremely fast, multi-threaded, meta-compressor library
\\
\hline
7
&
\sphinxstylestrong{Snappy} (previously known as Zippy) is a fast data compression and decompression library written in C++ by Google based on ideas from LZ77 and open-sourced in 2011. It does not aim for maximum compression, or compatibility with any other compression library; instead, it aims for very high speeds and reasonable compression
\\
\hline
8
&
\sphinxstylestrong{Lzo} or Lempel\textendash{}Ziv\textendash{}Oberhumer is a lossless data compression algorithm that is focused on decompression speed
\\
\hline
9
&
\sphinxstylestrong{Brotli} es una librería de compresión de datos de código abierto desarrollada por Jyrki Alakuijala y Zoltán Szabadka.1 2 Brotli está basado en una variante moderna del algoritmo LZ77, codificación Huffman y modelado de contexto de segundo orden.
\\
\hline
10
&
\sphinxstylestrong{Zstandard}, or zstd as short version, is a fast lossless compression algorithm, targeting real-time compression scenarios at zlib-level and better compression ratios.
\\
\hline
\end{tabulary}
\par
\sphinxattableend\end{savenotes}
\index{Compressor (clase en openerm.Compressor)@\spxentry{Compressor}\spxextra{clase en openerm.Compressor}}

\begin{fulllineitems}
\phantomsection\label{\detokenize{openerm.Compressor:openerm.Compressor.Compressor}}\pysiglinewithargsret{\sphinxbfcode{\sphinxupquote{class }}\sphinxcode{\sphinxupquote{openerm.Compressor.}}\sphinxbfcode{\sphinxupquote{Compressor}}}{\emph{compress\_type=1}, \emph{level=1}}{}
Bases: \sphinxcode{\sphinxupquote{object}}

Clase base para el manejo de compresión/descompresión de «bytes».

Esta es una clase base para poder configurar distintos algoritmos de
compresión, pero que a la vez ofrezca una interfaz común para comprimir
y descomprimir.
\begin{quote}\begin{description}
\item[{Parámetros}] \leavevmode\begin{itemize}
\item {} 
\sphinxstyleliteralstrong{\sphinxupquote{compress\_type}} (\sphinxstyleliteralemphasis{\sphinxupquote{int}}) \textendash{} Tipo de compresión

\item {} 
\sphinxstyleliteralstrong{\sphinxupquote{level}} (\sphinxstyleliteralemphasis{\sphinxupquote{int}}) \textendash{} Nivel de compresión 0=mínimo, 1=normal, 2=máximo

\end{itemize}

\end{description}\end{quote}
\subsubsection*{Ejemplo}

\begin{sphinxVerbatim}[commandchars=\\\{\}]
\PYG{g+gp}{\PYGZgt{}\PYGZgt{}\PYGZgt{} }\PYG{k+kn}{from} \PYG{n+nn}{openerm}\PYG{n+nn}{.}\PYG{n+nn}{Compressor} \PYG{k}{import} \PYG{n}{Compressor}
\PYG{g+gp}{\PYGZgt{}\PYGZgt{}\PYGZgt{} }\PYG{n}{c} \PYG{o}{=} \PYG{n}{Compressor}\PYG{p}{(}\PYG{n}{compress\PYGZus{}type}\PYG{o}{=}\PYG{l+m+mi}{1}\PYG{p}{,} \PYG{n}{level}\PYG{o}{=}\PYG{l+m+mi}{1}\PYG{p}{)}
\PYG{g+gp}{\PYGZgt{}\PYGZgt{}\PYGZgt{} }\PYG{n}{tmp} \PYG{o}{=} \PYG{n}{c}\PYG{o}{.}\PYG{n}{compress}\PYG{p}{(}\PYG{l+s+sa}{b}\PYG{l+s+s2}{\PYGZdq{}}\PYG{l+s+s2}{Esta es una prueba}\PYG{l+s+s2}{\PYGZdq{}}\PYG{p}{)}
\PYG{g+gp}{\PYGZgt{}\PYGZgt{}\PYGZgt{} }\PYG{n+nb}{print}\PYG{p}{(}\PYG{n}{tmp}\PYG{p}{)}
\PYG{g+go}{b\PYGZsq{}x\PYGZca{}s\PYGZhy{}.ITH\PYGZhy{}V(ÍKT((*MMJ\PYGZbs{}\PYGZlt{}\PYGZcb{}‰\PYGZsq{}}
\end{sphinxVerbatim}
\index{available\_types (atributo de openerm.Compressor.Compressor)@\spxentry{available\_types}\spxextra{atributo de openerm.Compressor.Compressor}}

\begin{fulllineitems}
\phantomsection\label{\detokenize{openerm.Compressor:openerm.Compressor.Compressor.available_types}}\pysigline{\sphinxbfcode{\sphinxupquote{available\_types}}}
Retorna los tipos de compresión disponibles.
\begin{quote}\begin{description}
\item[{Devuelve}] \leavevmode
Lista de algoritmos disponibles.

\end{description}\end{quote}
\subsubsection*{Ejemplo}

\begin{sphinxVerbatim}[commandchars=\\\{\}]
\PYG{g+gp}{\PYGZgt{}\PYGZgt{}\PYGZgt{} }\PYG{k+kn}{from} \PYG{n+nn}{openerm}\PYG{n+nn}{.}\PYG{n+nn}{Compressor} \PYG{k}{import} \PYG{n}{Compressor}
\PYG{g+gp}{\PYGZgt{}\PYGZgt{}\PYGZgt{} }\PYG{n}{c} \PYG{o}{=} \PYG{n}{Compressor}\PYG{p}{(}\PYG{p}{)}
\PYG{g+gp}{\PYGZgt{}\PYGZgt{}\PYGZgt{} }\PYG{k}{for} \PYG{n}{c} \PYG{o+ow}{in} \PYG{n}{c}\PYG{o}{.}\PYG{n}{available\PYGZus{}types}\PYG{p}{:}
\PYG{g+gp}{... }    \PYG{n+nb}{print}\PYG{p}{(}\PYG{n}{c}\PYG{p}{)}
\PYG{g+gp}{...}
\PYG{g+go}{(0, \PYGZsq{}Sin compression\PYGZsq{})}
\PYG{g+go}{(1, \PYGZsq{}GZIP level=6 (1\PYGZhy{}9)\PYGZsq{})}
\PYG{g+go}{(2, \PYGZsq{}BZIP level=6 (1\PYGZhy{}9)\PYGZsq{})}
\PYG{g+go}{(3, \PYGZsq{}LZMA preset=6 (0\PYGZhy{}9) \PYGZsq{})}
\PYG{g+go}{(4, \PYGZsq{}LZ4 nivel estándar\PYGZsq{})}
\PYG{g+go}{(5, \PYGZsq{}pyLZMA quality=1 (0\PYGZhy{}2)\PYGZsq{})}
\PYG{g+go}{(6, \PYGZsq{}BLOSC blosclz clevel=6 (1\PYGZhy{}9)\PYGZsq{})}
\PYG{g+go}{(7, \PYGZsq{}Snappy\PYGZsq{})}
\PYG{g+go}{(8, \PYGZsq{}Lzo level=2 (1\PYGZhy{}9)\PYGZsq{})}
\PYG{g+go}{(9, \PYGZsq{}Brotli quality=2 (1\PYGZhy{}11)\PYGZsq{})}
\PYG{g+go}{(10, \PYGZsq{}zstd level=3 (1\PYGZhy{}22)\PYGZsq{})}
\PYG{g+go}{\PYGZgt{}\PYGZgt{}\PYGZgt{}}
\end{sphinxVerbatim}

\end{fulllineitems}

\index{compress() (método de openerm.Compressor.Compressor)@\spxentry{compress()}\spxextra{método de openerm.Compressor.Compressor}}

\begin{fulllineitems}
\phantomsection\label{\detokenize{openerm.Compressor:openerm.Compressor.Compressor.compress}}\pysiglinewithargsret{\sphinxbfcode{\sphinxupquote{compress}}}{\emph{data}}{}
Comprime un conjunto de bytes
\begin{quote}\begin{description}
\item[{Parámetros}] \leavevmode
\sphinxstyleliteralstrong{\sphinxupquote{data}} \textendash{} (bytes) conjunto de bytes a comprimir

\item[{Devuelve}] \leavevmode
bytes comprimidos.

\end{description}\end{quote}
\subsubsection*{Ejemplo}

\begin{sphinxVerbatim}[commandchars=\\\{\}]
\PYG{g+gp}{\PYGZgt{}\PYGZgt{}\PYGZgt{} }\PYG{k+kn}{from} \PYG{n+nn}{openerm}\PYG{n+nn}{.}\PYG{n+nn}{Compressor} \PYG{k}{import} \PYG{n}{Compressor}
\PYG{g+gp}{\PYGZgt{}\PYGZgt{}\PYGZgt{} }\PYG{n}{c} \PYG{o}{=} \PYG{n}{Compressor}\PYG{p}{(}\PYG{n}{compress\PYGZus{}type}\PYG{o}{=}\PYG{l+m+mi}{1}\PYG{p}{,} \PYG{n}{level}\PYG{o}{=}\PYG{l+m+mi}{1}\PYG{p}{)}
\PYG{g+gp}{\PYGZgt{}\PYGZgt{}\PYGZgt{} }\PYG{n}{tmp} \PYG{o}{=} \PYG{n}{c}\PYG{o}{.}\PYG{n}{compress}\PYG{p}{(}\PYG{l+s+sa}{b}\PYG{l+s+s2}{\PYGZdq{}}\PYG{l+s+s2}{Esta es una prueba}\PYG{l+s+s2}{\PYGZdq{}}\PYG{p}{)}
\PYG{g+gp}{\PYGZgt{}\PYGZgt{}\PYGZgt{} }\PYG{n+nb}{print}\PYG{p}{(}\PYG{n}{tmp}\PYG{p}{)}
\PYG{g+go}{b\PYGZsq{}x\PYGZca{}s\PYGZhy{}.ITH\PYGZhy{}V(ÍKT((*MMJ\PYGZbs{}\PYGZlt{}\PYGZcb{}‰\PYGZsq{}}
\end{sphinxVerbatim}

\end{fulllineitems}

\index{compression\_type\_info() (método de openerm.Compressor.Compressor)@\spxentry{compression\_type\_info()}\spxextra{método de openerm.Compressor.Compressor}}

\begin{fulllineitems}
\phantomsection\label{\detokenize{openerm.Compressor:openerm.Compressor.Compressor.compression_type_info}}\pysiglinewithargsret{\sphinxbfcode{\sphinxupquote{compression\_type\_info}}}{\emph{compress\_type}}{}
Retorna la información de un determinado
algoritmo de compressión disponible.
\begin{quote}\begin{description}
\item[{Parámetros}] \leavevmode
\sphinxstyleliteralstrong{\sphinxupquote{compress\_type}} (\sphinxstyleliteralemphasis{\sphinxupquote{int}}) \textendash{} Id del algoritmo de compresión

\item[{Devuelve}] \leavevmode
Información del algoritmo de compresión

\item[{Tipo del valor devuelto}] \leavevmode
string

\end{description}\end{quote}
\subsubsection*{Ejemplo}

\begin{sphinxVerbatim}[commandchars=\\\{\}]
\PYG{g+gp}{\PYGZgt{}\PYGZgt{}\PYGZgt{} }\PYG{k+kn}{from} \PYG{n+nn}{openerm}\PYG{n+nn}{.}\PYG{n+nn}{Compressor} \PYG{k}{import} \PYG{n}{Compressor}
\PYG{g+gp}{\PYGZgt{}\PYGZgt{}\PYGZgt{} }\PYG{n}{c} \PYG{o}{=} \PYG{n}{Compressor}\PYG{p}{(}\PYG{p}{)}
\PYG{g+gp}{\PYGZgt{}\PYGZgt{}\PYGZgt{} }\PYG{n+nb}{print}\PYG{p}{(}\PYG{n}{c}\PYG{o}{.}\PYG{n}{compression\PYGZus{}type\PYGZus{}info}\PYG{p}{(}\PYG{l+m+mi}{10}\PYG{p}{)}\PYG{p}{)}
\PYG{g+go}{zstd level=3 (1\PYGZhy{}22)}
\end{sphinxVerbatim}

\end{fulllineitems}

\index{decompress() (método de openerm.Compressor.Compressor)@\spxentry{decompress()}\spxextra{método de openerm.Compressor.Compressor}}

\begin{fulllineitems}
\phantomsection\label{\detokenize{openerm.Compressor:openerm.Compressor.Compressor.decompress}}\pysiglinewithargsret{\sphinxbfcode{\sphinxupquote{decompress}}}{\emph{data}}{}
Descomprime un conjunto de bytes
\begin{quote}\begin{description}
\item[{Parámetros}] \leavevmode
\sphinxstyleliteralstrong{\sphinxupquote{data}} \textendash{} (bytes) conjunto de bytes comprimidos

\item[{Devuelve}] \leavevmode
bytes descomprimidos.

\end{description}\end{quote}
\subsubsection*{Ejemplo}

\begin{sphinxVerbatim}[commandchars=\\\{\}]
\PYG{g+gp}{\PYGZgt{}\PYGZgt{}\PYGZgt{} }\PYG{k+kn}{from} \PYG{n+nn}{openerm}\PYG{n+nn}{.}\PYG{n+nn}{Compressor} \PYG{k}{import} \PYG{n}{Compressor}
\PYG{g+gp}{\PYGZgt{}\PYGZgt{}\PYGZgt{} }\PYG{n}{c} \PYG{o}{=} \PYG{n}{Compressor}\PYG{p}{(}\PYG{n}{compress\PYGZus{}type}\PYG{o}{=}\PYG{l+m+mi}{1}\PYG{p}{,} \PYG{n}{level}\PYG{o}{=}\PYG{l+m+mi}{1}\PYG{p}{)}
\PYG{g+gp}{\PYGZgt{}\PYGZgt{}\PYGZgt{} }\PYG{n}{tmp} \PYG{o}{=} \PYG{n}{c}\PYG{o}{.}\PYG{n}{compress}\PYG{p}{(}\PYG{l+s+sa}{b}\PYG{l+s+s2}{\PYGZdq{}}\PYG{l+s+s2}{Esta es una prueba}\PYG{l+s+s2}{\PYGZdq{}}\PYG{p}{)}
\PYG{g+gp}{\PYGZgt{}\PYGZgt{}\PYGZgt{} }\PYG{n+nb}{print}\PYG{p}{(}\PYG{n}{tmp}\PYG{p}{)}
\PYG{g+go}{b\PYGZsq{}x\PYGZca{}s\PYGZhy{}.ITH\PYGZhy{}V(ÍKT((*MMJ\PYGZbs{}\PYGZlt{}\PYGZcb{}‰\PYGZsq{}}
\PYG{g+gp}{\PYGZgt{}\PYGZgt{}\PYGZgt{} }\PYG{n+nb}{print}\PYG{p}{(}\PYG{n}{c}\PYG{o}{.}\PYG{n}{decompress}\PYG{p}{(}\PYG{n}{tmp}\PYG{p}{)}\PYG{p}{)}
\PYG{g+go}{b\PYGZsq{}Esta es una prueba\PYGZsq{}}
\end{sphinxVerbatim}

\end{fulllineitems}

\index{level (atributo de openerm.Compressor.Compressor)@\spxentry{level}\spxextra{atributo de openerm.Compressor.Compressor}}

\begin{fulllineitems}
\phantomsection\label{\detokenize{openerm.Compressor:openerm.Compressor.Compressor.level}}\pysigline{\sphinxbfcode{\sphinxupquote{level}}}
Nivel de compresión a utilizar
\subsubsection*{Ejemplo}

\begin{sphinxVerbatim}[commandchars=\\\{\}]
\PYG{g+gp}{\PYGZgt{}\PYGZgt{}\PYGZgt{} }\PYG{k+kn}{from} \PYG{n+nn}{openerm}\PYG{n+nn}{.}\PYG{n+nn}{Compressor} \PYG{k}{import} \PYG{n}{Compressor}
\PYG{g+gp}{\PYGZgt{}\PYGZgt{}\PYGZgt{} }\PYG{n}{c} \PYG{o}{=} \PYG{n}{Compressor}\PYG{p}{(}\PYG{p}{)}
\PYG{g+gp}{\PYGZgt{}\PYGZgt{}\PYGZgt{} }\PYG{n+nb}{print}\PYG{p}{(}\PYG{n}{c}\PYG{o}{.}\PYG{n}{level}\PYG{p}{)}
\PYG{g+go}{1}
\end{sphinxVerbatim}

\end{fulllineitems}

\index{type (atributo de openerm.Compressor.Compressor)@\spxentry{type}\spxextra{atributo de openerm.Compressor.Compressor}}

\begin{fulllineitems}
\phantomsection\label{\detokenize{openerm.Compressor:openerm.Compressor.Compressor.type}}\pysigline{\sphinxbfcode{\sphinxupquote{type}}}
Tipo de compresión (algorimo) a utilizar.
\subsubsection*{Ejemplo}

\begin{sphinxVerbatim}[commandchars=\\\{\}]
\PYG{g+gp}{\PYGZgt{}\PYGZgt{}\PYGZgt{} }\PYG{k+kn}{from} \PYG{n+nn}{openerm}\PYG{n+nn}{.}\PYG{n+nn}{Compressor} \PYG{k}{import} \PYG{n}{Compressor}
\PYG{g+gp}{\PYGZgt{}\PYGZgt{}\PYGZgt{} }\PYG{n}{c} \PYG{o}{=} \PYG{n}{Compressor}\PYG{p}{(}\PYG{p}{)}
\PYG{g+gp}{\PYGZgt{}\PYGZgt{}\PYGZgt{} }\PYG{n+nb}{print}\PYG{p}{(}\PYG{n}{c}\PYG{o}{.}\PYG{n}{type}\PYG{p}{)}
\PYG{g+go}{1}
\end{sphinxVerbatim}

\end{fulllineitems}


\end{fulllineitems}



\subsubsection{Index}
\label{\detokenize{openerm.Index:module-openerm.Index}}\label{\detokenize{openerm.Index:index}}\label{\detokenize{openerm.Index::doc}}\index{openerm.Index (módulo)@\spxentry{openerm.Index}\spxextra{módulo}}
Copyright (c) 2014 Patricio Moracho \textless{}\sphinxhref{mailto:pmoracho@gmail.com}{pmoracho@gmail.com}\textgreater{}

Index.py

This program is free software; you can redistribute it and/or
modify it under the terms of version 3 of the GNU General Public License
as published by the Free Software Foundation. A copy of this license should
be included in the file GPL-3.

This program is distributed in the hope that it will be useful,
but WITHOUT ANY WARRANTY; without even the implied warranty of
MERCHANTABILITY or FITNESS FOR A PARTICULAR PURPOSE.    See the
GNU Library General Public License for more details.

You should have received a copy of the GNU General Public License
along with this program; if not, write to the Free Software
Foundation, Inc., 59 Temple Place - Suite 330, Boston, MA 02111-1307, USA.
\index{Index (clase en openerm.Index)@\spxentry{Index}\spxextra{clase en openerm.Index}}

\begin{fulllineitems}
\phantomsection\label{\detokenize{openerm.Index:openerm.Index.Index}}\pysiglinewithargsret{\sphinxbfcode{\sphinxupquote{class }}\sphinxcode{\sphinxupquote{openerm.Index.}}\sphinxbfcode{\sphinxupquote{Index}}}{\emph{oermdb\_file}}{}
Bases: \sphinxcode{\sphinxupquote{object}}
\index{add\_container() (método de openerm.Index.Index)@\spxentry{add\_container()}\spxextra{método de openerm.Index.Index}}

\begin{fulllineitems}
\phantomsection\label{\detokenize{openerm.Index:openerm.Index.Index.add_container}}\pysiglinewithargsret{\sphinxbfcode{\sphinxupquote{add\_container}}}{\emph{reporte\_id}, \emph{container\_offset}}{}
\end{fulllineitems}

\index{add\_report() (método de openerm.Index.Index)@\spxentry{add\_report()}\spxextra{método de openerm.Index.Index}}

\begin{fulllineitems}
\phantomsection\label{\detokenize{openerm.Index:openerm.Index.Index.add_report}}\pysiglinewithargsret{\sphinxbfcode{\sphinxupquote{add\_report}}}{\emph{reporte}, \emph{report\_offset}, \emph{pages\_in\_container}}{}
\end{fulllineitems}

\index{get\_report() (método de openerm.Index.Index)@\spxentry{get\_report()}\spxextra{método de openerm.Index.Index}}

\begin{fulllineitems}
\phantomsection\label{\detokenize{openerm.Index:openerm.Index.Index.get_report}}\pysiglinewithargsret{\sphinxbfcode{\sphinxupquote{get\_report}}}{\emph{reporte}}{}
Obtiene el id de un reporte en el índice

\end{fulllineitems}

\index{read() (método de openerm.Index.Index)@\spxentry{read()}\spxextra{método de openerm.Index.Index}}

\begin{fulllineitems}
\phantomsection\label{\detokenize{openerm.Index:openerm.Index.Index.read}}\pysiglinewithargsret{\sphinxbfcode{\sphinxupquote{read}}}{}{}
\end{fulllineitems}

\index{write() (método de openerm.Index.Index)@\spxentry{write()}\spxextra{método de openerm.Index.Index}}

\begin{fulllineitems}
\phantomsection\label{\detokenize{openerm.Index:openerm.Index.Index.write}}\pysiglinewithargsret{\sphinxbfcode{\sphinxupquote{write}}}{}{}
\end{fulllineitems}


\end{fulllineitems}

\phantomsection\label{\detokenize{openerm.Pages:pages}}\phantomsection\label{\detokenize{openerm.Pages:module-openerm.Pages}}\phantomsection\label{\detokenize{openerm.Pages:pages}}\index{openerm.Pages (módulo)@\spxentry{openerm.Pages}\spxextra{módulo}}

\subsubsection{Pages}
\label{\detokenize{openerm.Pages:id1}}\label{\detokenize{openerm.Pages::doc}}
Clase «dummy», pensada eventualmente para el manejo de una colección de páginas.
\index{Pages (clase en openerm.Pages)@\spxentry{Pages}\spxextra{clase en openerm.Pages}}

\begin{fulllineitems}
\phantomsection\label{\detokenize{openerm.Pages:openerm.Pages.Pages}}\pysiglinewithargsret{\sphinxbfcode{\sphinxupquote{class }}\sphinxcode{\sphinxupquote{openerm.Pages.}}\sphinxbfcode{\sphinxupquote{Pages}}}{\emph{file}, \emph{index}}{}
Bases: \sphinxcode{\sphinxupquote{object}}

Clase «Dummy» para el manejo de páginas

\end{fulllineitems}

\phantomsection\label{\detokenize{openerm.Report:report}}\phantomsection\label{\detokenize{openerm.Report:module-openerm.Report}}\phantomsection\label{\detokenize{openerm.Report:report}}\index{openerm.Report (módulo)@\spxentry{openerm.Report}\spxextra{módulo}}

\subsubsection{Report}
\label{\detokenize{openerm.Report:id1}}\label{\detokenize{openerm.Report::doc}}
Este objeto representa un Reporte almacenado en un {\hyperref[\detokenize{openerm.Database:module-openerm.Database}]{\sphinxcrossref{\sphinxcode{\sphinxupquote{openerm.Database}}}}}.

Ejemplo:

Un reporte por su parte posee:
\begin{itemize}
\item {} 
Páginas

\item {} 
Metadatos o atributos

\end{itemize}


\sphinxstrong{Ver también:}

\begin{itemize}
\item {} 
{\hyperref[\detokenize{openerm.Reports:module-openerm.Reports}]{\sphinxcrossref{\sphinxcode{\sphinxupquote{openerm.Reports}}}}}

\item {} 
{\hyperref[\detokenize{openerm.Database:module-openerm.Database}]{\sphinxcrossref{\sphinxcode{\sphinxupquote{openerm.Database}}}}}

\end{itemize}


\index{Report (clase en openerm.Report)@\spxentry{Report}\spxextra{clase en openerm.Report}}

\begin{fulllineitems}
\phantomsection\label{\detokenize{openerm.Report:openerm.Report.Report}}\pysiglinewithargsret{\sphinxbfcode{\sphinxupquote{class }}\sphinxcode{\sphinxupquote{openerm.Report.}}\sphinxbfcode{\sphinxupquote{Report}}}{\emph{database}, \emph{idrpt}}{}
Bases: \sphinxcode{\sphinxupquote{object}}

Clase para el manejo de un Reporte OERM.
\begin{quote}\begin{description}
\item[{Parámetros}] \leavevmode\begin{itemize}
\item {} 
\sphinxstyleliteralstrong{\sphinxupquote{database}} \textendash{} Objeto {\hyperref[\detokenize{openerm.Database:module-openerm.Database}]{\sphinxcrossref{\sphinxcode{\sphinxupquote{openerm.Database}}}}}

\item {} 
\sphinxstyleliteralstrong{\sphinxupquote{idrpt}} (\sphinxstyleliteralemphasis{\sphinxupquote{int}}) \textendash{} Identificador único del reporte en el Database

\end{itemize}

\end{description}\end{quote}
\subsubsection*{Ejemplo}

\begin{sphinxVerbatim}[commandchars=\\\{\}]
\PYG{g+gp}{\PYGZgt{}\PYGZgt{}\PYGZgt{} }\PYG{n}{fuerom} \PYG{n}{openerm}\PYG{o}{.}\PYG{n}{Database} \PYG{k+kn}{import} \PYG{n+nn}{Database}
\PYG{g+gp}{\PYGZgt{}\PYGZgt{}\PYGZgt{} }\PYG{k+kn}{from} \PYG{n+nn}{openerm}\PYG{n+nn}{.}\PYG{n+nn}{Report} \PYG{k}{import} \PYG{n}{Report}
\PYG{g+gp}{\PYGZgt{}\PYGZgt{}\PYGZgt{} }\PYG{n}{db} \PYG{o}{=} \PYG{n}{Database}\PYG{p}{(}\PYG{n}{file} \PYG{o}{=} \PYG{l+s+s2}{\PYGZdq{}}\PYG{l+s+s2}{out/zstd\PYGZhy{}level\PYGZhy{}3\PYGZhy{}1\PYGZhy{}22.test.oerm}\PYG{l+s+s2}{\PYGZdq{}}\PYG{p}{,} \PYG{n}{mode}\PYG{o}{=}\PYG{l+s+s2}{\PYGZdq{}}\PYG{l+s+s2}{rb}\PYG{l+s+s2}{\PYGZdq{}}\PYG{p}{)}
\PYG{g+gp}{\PYGZgt{}\PYGZgt{}\PYGZgt{} }\PYG{n}{r} \PYG{o}{=} \PYG{n}{Report}\PYG{p}{(}\PYG{n}{db}\PYG{p}{,} \PYG{l+m+mi}{1}\PYG{p}{)}
\PYG{g+gp}{\PYGZgt{}\PYGZgt{}\PYGZgt{} }\PYG{k}{for} \PYG{n}{page} \PYG{o+ow}{in} \PYG{n}{r}\PYG{p}{:}
\PYG{g+gp}{... }    \PYG{n+nb}{print}\PYG{p}{(}\PYG{n}{page}\PYG{p}{[}\PYG{l+m+mi}{0}\PYG{p}{:}\PYG{l+m+mi}{10}\PYG{p}{]}\PYG{p}{)}
\PYG{g+gp}{...}
\PYG{g+go}{Pagina 1 \PYGZhy{}}
\PYG{g+go}{Pagina 2 \PYGZhy{}}
\PYG{g+go}{Pagina 3 \PYGZhy{}}
\PYG{g+go}{Pagina 4 \PYGZhy{}}
\PYG{g+go}{Pagina 5 \PYGZhy{}}
\PYG{g+go}{Pagina 6 \PYGZhy{}}
\PYG{g+go}{Pagina 7 \PYGZhy{}}
\PYG{g+go}{Pagina 8 \PYGZhy{}}
\PYG{g+go}{Pagina 9 \PYGZhy{}}
\PYG{g+go}{Pagina 10}
\PYG{g+go}{Pagina 11}
\end{sphinxVerbatim}
\begin{description}
\item[{\sphinxstylestrong{data}:}] \leavevmode

\begin{savenotes}\sphinxattablestart
\centering
\begin{tabulary}{\linewidth}[t]{|T|T|}
\hline
\sphinxstyletheadfamily 
Tipo
&\sphinxstyletheadfamily 
Detalle
\\
\hline
int
&
Id del reporte
\\
\hline
string
&
Nombre del reporte
\\
\hline
long
&
Offset al contenedor de metadatos
\\
\hline
long
&
Max cantidad de páginas en los PageContainers
\\
\hline
long
&
Offset al primer PageContainer
\\
\hline
list
&
Lista de Offsets a los PageContainers
\\
\hline
\end{tabulary}
\par
\sphinxattableend\end{savenotes}

\end{description}
\index{find\_text() (método de openerm.Report.Report)@\spxentry{find\_text()}\spxextra{método de openerm.Report.Report}}

\begin{fulllineitems}
\phantomsection\label{\detokenize{openerm.Report:openerm.Report.Report.find_text}}\pysiglinewithargsret{\sphinxbfcode{\sphinxupquote{find\_text}}}{\emph{text}}{}
Búsqueda de un texto dentro del reporte
\begin{quote}\begin{description}
\item[{Parámetros}] \leavevmode
\sphinxstyleliteralstrong{\sphinxupquote{text}} (\sphinxstyleliteralemphasis{\sphinxupquote{string}}) \textendash{} Patrón de texto a buscar

\end{description}\end{quote}
\subsubsection*{Ejemplo}

\begin{sphinxVerbatim}[commandchars=\\\{\}]
\PYG{g+gp}{\PYGZgt{}\PYGZgt{}\PYGZgt{} }\PYG{k+kn}{from} \PYG{n+nn}{openerm}\PYG{n+nn}{.}\PYG{n+nn}{Database} \PYG{k}{import} \PYG{n}{Database}
\PYG{g+gp}{\PYGZgt{}\PYGZgt{}\PYGZgt{} }\PYG{k+kn}{from} \PYG{n+nn}{openerm}\PYG{n+nn}{.}\PYG{n+nn}{Report} \PYG{k}{import} \PYG{n}{Report}
\PYG{g+gp}{\PYGZgt{}\PYGZgt{}\PYGZgt{} }\PYG{n}{db} \PYG{o}{=} \PYG{n}{Database}\PYG{p}{(}\PYG{n}{file} \PYG{o}{=} \PYG{l+s+s2}{\PYGZdq{}}\PYG{l+s+s2}{out/.sin\PYGZus{}compression\PYGZus{}sin\PYGZus{}encriptacion.oerm}\PYG{l+s+s2}{\PYGZdq{}}\PYG{p}{)}
\PYG{g+gp}{\PYGZgt{}\PYGZgt{}\PYGZgt{} }\PYG{n}{r} \PYG{o}{=} \PYG{n}{Report}\PYG{p}{(}\PYG{n}{db}\PYG{p}{,} \PYG{l+m+mi}{1}\PYG{p}{)}
\PYG{g+gp}{\PYGZgt{}\PYGZgt{}\PYGZgt{} }\PYG{n}{report}\PYG{o}{.}\PYG{n}{find\PYGZus{}text}\PYG{p}{(}\PYG{l+s+s2}{\PYGZdq{}}\PYG{l+s+s2}{IWY3}\PYG{l+s+s2}{\PYGZdq{}}\PYG{p}{)}
\PYG{g+go}{[(2, 10, 991, \PYGZsq{}AGH8B2NULTCTJ0L\PYGZhy{}[IWY3]\PYGZhy{}4K6D8RRBYCRQCH\PYGZsq{})]}
\end{sphinxVerbatim}
\begin{quote}\begin{description}
\item[{Devuelve}] \leavevmode
\begin{description}
\item[{Lista de reportes y páginas}] \leavevmode\begin{itemize}
\item {} 
Reporte id

\item {} 
Página

\item {} 
Posición en la página

\item {} 
Extracto de la ocurrencia a modo de ejemplo

\end{itemize}

\end{description}


\end{description}\end{quote}

\end{fulllineitems}

\index{get\_page() (método de openerm.Report.Report)@\spxentry{get\_page()}\spxextra{método de openerm.Report.Report}}

\begin{fulllineitems}
\phantomsection\label{\detokenize{openerm.Report:openerm.Report.Report.get_page}}\pysiglinewithargsret{\sphinxbfcode{\sphinxupquote{get\_page}}}{\emph{pagenum}}{}
Retorna una pagina del reporte
\begin{quote}\begin{description}
\item[{Parámetros}] \leavevmode
\sphinxstyleliteralstrong{\sphinxupquote{pagenum}} (\sphinxstyleliteralemphasis{\sphinxupquote{int}}) \textendash{} Número de página

\end{description}\end{quote}
\subsubsection*{Ejemplo}

\begin{sphinxVerbatim}[commandchars=\\\{\}]
\PYG{g+gp}{\PYGZgt{}\PYGZgt{}\PYGZgt{} }\PYG{k+kn}{from} \PYG{n+nn}{openerm}\PYG{n+nn}{.}\PYG{n+nn}{Database} \PYG{k}{import} \PYG{n}{Database}
\PYG{g+gp}{\PYGZgt{}\PYGZgt{}\PYGZgt{} }\PYG{k+kn}{from} \PYG{n+nn}{openerm}\PYG{n+nn}{.}\PYG{n+nn}{Report} \PYG{k}{import} \PYG{n}{Report}
\PYG{g+gp}{\PYGZgt{}\PYGZgt{}\PYGZgt{} }\PYG{n}{db} \PYG{o}{=} \PYG{n}{Database}\PYG{p}{(}\PYG{n}{file} \PYG{o}{=} \PYG{l+s+s2}{\PYGZdq{}}\PYG{l+s+s2}{out/zstd\PYGZhy{}level\PYGZhy{}3\PYGZhy{}1\PYGZhy{}22.test.oerm}\PYG{l+s+s2}{\PYGZdq{}}\PYG{p}{,} \PYG{n}{mode}\PYG{o}{=}\PYG{l+s+s2}{\PYGZdq{}}\PYG{l+s+s2}{rb}\PYG{l+s+s2}{\PYGZdq{}}\PYG{p}{)}
\PYG{g+gp}{\PYGZgt{}\PYGZgt{}\PYGZgt{} }\PYG{n}{r} \PYG{o}{=} \PYG{n}{Report}\PYG{p}{(}\PYG{n}{db}\PYG{p}{,} \PYG{l+m+mi}{1}\PYG{p}{)}
\PYG{g+gp}{\PYGZgt{}\PYGZgt{}\PYGZgt{} }\PYG{n}{p} \PYG{o}{=} \PYG{n}{r}\PYG{o}{.}\PYG{n}{get\PYGZus{}page}\PYG{p}{(}\PYG{l+m+mi}{5}\PYG{p}{)}
\PYG{g+gp}{\PYGZgt{}\PYGZgt{}\PYGZgt{} }\PYG{n+nb}{print}\PYG{p}{(}\PYG{n}{p}\PYG{p}{[}\PYG{l+m+mi}{0}\PYG{p}{:}\PYG{l+m+mi}{30}\PYG{p}{]}\PYG{p}{)}
\PYG{g+go}{Pagina 5 \PYGZhy{}\PYGZhy{}\PYGZhy{}\PYGZhy{}\PYGZhy{}\PYGZhy{}\PYGZhy{}\PYGZhy{}\PYGZhy{}\PYGZhy{}\PYGZhy{}\PYGZhy{}\PYGZhy{}\PYGZhy{}\PYGZhy{}\PYGZhy{}\PYGZhy{}}
\PYG{g+go}{ZSV}
\PYG{g+go}{\PYGZgt{}\PYGZgt{}\PYGZgt{}}
\end{sphinxVerbatim}
\begin{quote}\begin{description}
\item[{Devuelve}] \leavevmode
Texto completo de la página

\item[{Tipo del valor devuelto}] \leavevmode
string

\end{description}\end{quote}

\end{fulllineitems}

\index{id (atributo de openerm.Report.Report)@\spxentry{id}\spxextra{atributo de openerm.Report.Report}}

\begin{fulllineitems}
\phantomsection\label{\detokenize{openerm.Report:openerm.Report.Report.id}}\pysigline{\sphinxbfcode{\sphinxupquote{id}}\sphinxbfcode{\sphinxupquote{ = None}}}
id del reporte

\end{fulllineitems}

\index{metadata (atributo de openerm.Report.Report)@\spxentry{metadata}\spxextra{atributo de openerm.Report.Report}}

\begin{fulllineitems}
\phantomsection\label{\detokenize{openerm.Report:openerm.Report.Report.metadata}}\pysigline{\sphinxbfcode{\sphinxupquote{metadata}}\sphinxbfcode{\sphinxupquote{ = None}}}
Metadatos del reporte

\end{fulllineitems}

\index{nombre (atributo de openerm.Report.Report)@\spxentry{nombre}\spxextra{atributo de openerm.Report.Report}}

\begin{fulllineitems}
\phantomsection\label{\detokenize{openerm.Report:openerm.Report.Report.nombre}}\pysigline{\sphinxbfcode{\sphinxupquote{nombre}}\sphinxbfcode{\sphinxupquote{ = None}}}
Nombre del reporte

\end{fulllineitems}

\index{total\_pages (atributo de openerm.Report.Report)@\spxentry{total\_pages}\spxextra{atributo de openerm.Report.Report}}

\begin{fulllineitems}
\phantomsection\label{\detokenize{openerm.Report:openerm.Report.Report.total_pages}}\pysigline{\sphinxbfcode{\sphinxupquote{total\_pages}}\sphinxbfcode{\sphinxupquote{ = None}}}
Cantidad total de páginas del reporte

\end{fulllineitems}


\end{fulllineitems}

\phantomsection\label{\detokenize{openerm.Reports:module-openerm.Reports}}\index{openerm.Reports (módulo)@\spxentry{openerm.Reports}\spxextra{módulo}}

\subsubsection{Reports}
\label{\detokenize{openerm.Reports:reports}}\label{\detokenize{openerm.Reports::doc}}
Clase base para manejar una colección de reportes correspondientes a
un Database OERM.


\sphinxstrong{Ver también:}

\begin{itemize}
\item {} 
{\hyperref[\detokenize{openerm.Report:module-openerm.Report}]{\sphinxcrossref{\sphinxcode{\sphinxupquote{openerm.Report}}}}}

\end{itemize}



\begin{sphinxadmonition}{note}{Nota:}\begin{itemize}
\item {} 
Esta clase no debiera usarse directamente

\item {} 
En un futuro modificar la inicialización, para que reciba el archivo.oerm y los archivos de los

\end{itemize}
\end{sphinxadmonition}
\index{Reports (clase en openerm.Reports)@\spxentry{Reports}\spxextra{clase en openerm.Reports}}

\begin{fulllineitems}
\phantomsection\label{\detokenize{openerm.Reports:openerm.Reports.Reports}}\pysiglinewithargsret{\sphinxbfcode{\sphinxupquote{class }}\sphinxcode{\sphinxupquote{openerm.Reports.}}\sphinxbfcode{\sphinxupquote{Reports}}}{\emph{database}}{}
Bases: \sphinxcode{\sphinxupquote{object}}

Clase base para manejar una colección de reportes correspondientes a
un Database OERM.
\begin{quote}\begin{description}
\item[{Parámetros}] \leavevmode
\sphinxstyleliteralstrong{\sphinxupquote{database}} ({\hyperref[\detokenize{openerm.Database:module-openerm.Database}]{\sphinxcrossref{\sphinxcode{\sphinxupquote{openerm.Database}}}}}) \textendash{} Objeto Database

\item[{Devuelve}] \leavevmode
Iterador por la lista de Reportes

\item[{Tipo del valor devuelto}] \leavevmode
iterable

\end{description}\end{quote}
\subsubsection*{Ejemplo}

\begin{sphinxVerbatim}[commandchars=\\\{\}]
\PYG{g+gp}{\PYGZgt{}\PYGZgt{}\PYGZgt{} }\PYG{k+kn}{from} \PYG{n+nn}{openerm}\PYG{n+nn}{.}\PYG{n+nn}{Database} \PYG{k}{import} \PYG{n}{Database}
\PYG{g+gp}{\PYGZgt{}\PYGZgt{}\PYGZgt{} }\PYG{k+kn}{from} \PYG{n+nn}{openerm}\PYG{n+nn}{.}\PYG{n+nn}{Reports} \PYG{k}{import} \PYG{n}{Reports}
\PYG{g+gp}{\PYGZgt{}\PYGZgt{}\PYGZgt{} }\PYG{n}{db} \PYG{o}{=} \PYG{n}{Database}\PYG{p}{(}\PYG{n}{file} \PYG{o}{=} \PYG{l+s+s2}{\PYGZdq{}}\PYG{l+s+s2}{out/zstd\PYGZhy{}level\PYGZhy{}3\PYGZhy{}1\PYGZhy{}22.test.oerm}\PYG{l+s+s2}{\PYGZdq{}}\PYG{p}{,} \PYG{n}{mode}\PYG{o}{=}\PYG{l+s+s2}{\PYGZdq{}}\PYG{l+s+s2}{rb}\PYG{l+s+s2}{\PYGZdq{}}\PYG{p}{)}
\PYG{g+gp}{\PYGZgt{}\PYGZgt{}\PYGZgt{} }\PYG{k}{for} \PYG{n}{report} \PYG{o+ow}{in} \PYG{n}{Reports}\PYG{p}{(}\PYG{n}{db}\PYG{p}{)}\PYG{p}{:}
\PYG{g+gp}{... }    \PYG{n+nb}{print}\PYG{p}{(}\PYG{n}{report}\PYG{p}{)}
\PYG{g+go}{Report: Reporte 1}
\PYG{g+go}{Report: Reporte 2}
\PYG{g+go}{\PYGZgt{}\PYGZgt{}\PYGZgt{}}
\end{sphinxVerbatim}
\index{find\_text() (método de openerm.Reports.Reports)@\spxentry{find\_text()}\spxextra{método de openerm.Reports.Reports}}

\begin{fulllineitems}
\phantomsection\label{\detokenize{openerm.Reports:openerm.Reports.Reports.find_text}}\pysiglinewithargsret{\sphinxbfcode{\sphinxupquote{find\_text}}}{\emph{text}, \emph{search\_in\_reports=None}}{}
Búsqueda de un texto dentro de uno o más reportes
\begin{quote}\begin{description}
\item[{Parámetros}] \leavevmode\begin{itemize}
\item {} 
\sphinxstyleliteralstrong{\sphinxupquote{text}} (\sphinxstyleliteralemphasis{\sphinxupquote{string}}) \textendash{} Patrón de texto a buscar

\item {} 
\sphinxstyleliteralstrong{\sphinxupquote{search\_in\_reports}} (\sphinxstyleliteralemphasis{\sphinxupquote{list}}) \textendash{} Lista de id´s de reportes dónde buscar o None en todos

\end{itemize}

\end{description}\end{quote}
\subsubsection*{Ejemplo}

\begin{sphinxVerbatim}[commandchars=\\\{\}]
\PYG{g+gp}{\PYGZgt{}\PYGZgt{}\PYGZgt{} }\PYG{k+kn}{from} \PYG{n+nn}{openerm}\PYG{n+nn}{.}\PYG{n+nn}{Database} \PYG{k}{import} \PYG{n}{Database}
\PYG{g+gp}{\PYGZgt{}\PYGZgt{}\PYGZgt{} }\PYG{k+kn}{from} \PYG{n+nn}{openerm}\PYG{n+nn}{.}\PYG{n+nn}{Reports} \PYG{k}{import} \PYG{n}{Reports}
\PYG{g+gp}{\PYGZgt{}\PYGZgt{}\PYGZgt{} }\PYG{n}{db} \PYG{o}{=} \PYG{n}{Database}\PYG{p}{(}\PYG{n}{file} \PYG{o}{=} \PYG{l+s+s2}{\PYGZdq{}}\PYG{l+s+s2}{out/.sin\PYGZus{}compression\PYGZus{}sin\PYGZus{}encriptacion.oerm}\PYG{l+s+s2}{\PYGZdq{}}\PYG{p}{)}
\PYG{g+gp}{\PYGZgt{}\PYGZgt{}\PYGZgt{} }\PYG{n}{reports} \PYG{o}{=} \PYG{n}{Reports}\PYG{p}{(}\PYG{n}{db}\PYG{p}{)}
\PYG{g+gp}{\PYGZgt{}\PYGZgt{}\PYGZgt{} }\PYG{n}{reports}\PYG{o}{.}\PYG{n}{find\PYGZus{}text}\PYG{p}{(}\PYG{l+s+s2}{\PYGZdq{}}\PYG{l+s+s2}{IWY3}\PYG{l+s+s2}{\PYGZdq{}}\PYG{p}{)}
\PYG{g+go}{[(2, 10, 991, \PYGZsq{}AGH8B2NULTCTJ0L\PYGZhy{}[IWY3]\PYGZhy{}4K6D8RRBYCRQCH\PYGZsq{})]}
\end{sphinxVerbatim}
\begin{quote}\begin{description}
\item[{Devuelve}] \leavevmode
\begin{description}
\item[{Lista de reportes y páginas}] \leavevmode\begin{itemize}
\item {} 
Reporte id

\item {} 
Página

\item {} 
Posición en la página

\item {} 
Extracto de la ocurrencia a modo de ejemplo

\end{itemize}

\end{description}


\end{description}\end{quote}

\end{fulllineitems}

\index{get\_report() (método de openerm.Reports.Reports)@\spxentry{get\_report()}\spxextra{método de openerm.Reports.Reports}}

\begin{fulllineitems}
\phantomsection\label{\detokenize{openerm.Reports:openerm.Reports.Reports.get_report}}\pysiglinewithargsret{\sphinxbfcode{\sphinxupquote{get\_report}}}{\emph{rptid}}{}
Retorna un objeto de la clase {\hyperref[\detokenize{openerm.Report:module-openerm.Report}]{\sphinxcrossref{\sphinxcode{\sphinxupquote{openerm.Report}}}}} según el \sphinxstylestrong{id} del mismo
\begin{quote}\begin{description}
\item[{Parámetros}] \leavevmode
\sphinxstyleliteralstrong{\sphinxupquote{id}} (\sphinxstyleliteralemphasis{\sphinxupquote{int}}) \textendash{} Número de reporte de la base Oerm

\end{description}\end{quote}
\subsubsection*{Ejemplo}

\begin{sphinxVerbatim}[commandchars=\\\{\}]
\PYG{g+gp}{\PYGZgt{}\PYGZgt{}\PYGZgt{} }\PYG{k+kn}{from} \PYG{n+nn}{openerm}\PYG{n+nn}{.}\PYG{n+nn}{Database} \PYG{k}{import} \PYG{n}{Database}
\PYG{g+gp}{\PYGZgt{}\PYGZgt{}\PYGZgt{} }\PYG{k+kn}{from} \PYG{n+nn}{openerm}\PYG{n+nn}{.}\PYG{n+nn}{Reports} \PYG{k}{import} \PYG{n}{Reports}
\PYG{g+gp}{\PYGZgt{}\PYGZgt{}\PYGZgt{} }\PYG{n}{db}\PYG{o}{=} \PYG{n}{Database}\PYG{p}{(}\PYG{n}{file} \PYG{o}{=} \PYG{l+s+s2}{\PYGZdq{}}\PYG{l+s+s2}{out/zstd\PYGZhy{}level\PYGZhy{}3\PYGZhy{}1\PYGZhy{}22.test.oerm}\PYG{l+s+s2}{\PYGZdq{}}\PYG{p}{,} \PYG{n}{mode}\PYG{o}{=}\PYG{l+s+s2}{\PYGZdq{}}\PYG{l+s+s2}{rb}\PYG{l+s+s2}{\PYGZdq{}}\PYG{p}{)}
\PYG{g+gp}{\PYGZgt{}\PYGZgt{}\PYGZgt{} }\PYG{n}{reports} \PYG{o}{=} \PYG{n}{Reports}\PYG{p}{(}\PYG{n}{db}\PYG{p}{)}
\PYG{g+gp}{\PYGZgt{}\PYGZgt{}\PYGZgt{} }\PYG{n}{reports}\PYG{o}{.}\PYG{n}{get\PYGZus{}report}\PYG{p}{(}\PYG{l+m+mi}{1}\PYG{p}{)}
\PYG{g+gp}{\PYGZgt{}\PYGZgt{}\PYGZgt{} }\PYG{n+nb}{print}\PYG{p}{(}\PYG{n}{reports}\PYG{o}{.}\PYG{n}{get\PYGZus{}report}\PYG{p}{(}\PYG{l+m+mi}{1}\PYG{p}{)}\PYG{p}{)}
\PYG{g+go}{Report: Reporte 1}
\end{sphinxVerbatim}
\begin{quote}\begin{description}
\item[{Devuelve}] \leavevmode
Reporte solicitado

\item[{Tipo del valor devuelto}] \leavevmode
{\hyperref[\detokenize{openerm.Report:module-openerm.Report}]{\sphinxcrossref{\sphinxcode{\sphinxupquote{openerm.Report}}}}}

\end{description}\end{quote}

\end{fulllineitems}


\end{fulllineitems}



\chapter{Herramientas}
\label{\detokenize{index:herramientas}}\phantomsection\label{\detokenize{splprocessor:splprocessor}}\phantomsection\label{\detokenize{splprocessor:module-splprocessor}}\phantomsection\label{\detokenize{splprocessor:splprocessor}}\index{splprocessor (módulo)@\spxentry{splprocessor}\spxextra{módulo}}

\section{splprocessor}
\label{\detokenize{splprocessor:id1}}\label{\detokenize{splprocessor::doc}}
\sphinxstylestrong{splprocessor}, es el monitor y procesador de archivos de colas de impresión
(spool) del proyecto OpenErm. Su trabajo consta de:
\begin{itemize}
\item {} 
Monitorear una o más carpetas

\item {} 
Detectar nuevos archivos en estas

\item {} 
Detectar por patrones regulares nombres de archivo a procesar

\item {} 
Verificar capacidad de bloqueo del archivo

\item {} \begin{description}
\item[{Definir parámetros de procesamiento en función de:}] \leavevmode\begin{itemize}
\item {} 
patrones regulares del nombre

\item {} 
patrones regulares dentro del contenido (limite de x bytes)

\end{itemize}

\end{description}

\item {} 
Renombrado de los mismos a .locked

\item {} 
Lectura de los archivos, proceso de paginas y generación de los reportes oerm

\item {} 
Generación de log de proceso

\item {} \begin{description}
\item[{Proceso final del Spool}] \leavevmode\begin{itemize}
\item {} 
Borrado

\item {} 
Copiado a otra carpeta

\item {} 
Renombrado

\item {} 
Compresión en otra carpeta

\end{itemize}

\end{description}

\end{itemize}
\index{init\_argparse() (en el módulo splprocessor)@\spxentry{init\_argparse()}\spxextra{en el módulo splprocessor}}

\begin{fulllineitems}
\phantomsection\label{\detokenize{splprocessor:splprocessor.init_argparse}}\pysiglinewithargsret{\sphinxcode{\sphinxupquote{splprocessor.}}\sphinxbfcode{\sphinxupquote{init\_argparse}}}{}{}
init\_argparse: Inicializar parametros del programa.

\end{fulllineitems}

\index{my\_gettext() (en el módulo splprocessor)@\spxentry{my\_gettext()}\spxextra{en el módulo splprocessor}}

\begin{fulllineitems}
\phantomsection\label{\detokenize{splprocessor:splprocessor.my_gettext}}\pysiglinewithargsret{\sphinxcode{\sphinxupquote{splprocessor.}}\sphinxbfcode{\sphinxupquote{my\_gettext}}}{\emph{s}}{}
my\_gettext: Traducir algunas cadenas de argparse.

\end{fulllineitems}

\phantomsection\label{\detokenize{spl2oerm:spl2oerm}}\phantomsection\label{\detokenize{spl2oerm:module-spl2oerm}}\phantomsection\label{\detokenize{spl2oerm:spl2oerm}}\index{spl2oerm (módulo)@\spxentry{spl2oerm}\spxextra{módulo}}

\section{spl2oerm}
\label{\detokenize{spl2oerm:id1}}\label{\detokenize{spl2oerm::doc}}
\sphinxtitleref{spl2oerm} es una aplicación de línea de comandos, parte de la familia de
herramientas del proyecto \sphinxhref{https://github.com/pmoracho/openerm}{Openerm}. Este
comando se utiliza para procesar archivos de «spool» (salidas de las colas de
impresión) y realiza las algunas de las siguientes operaciones:
\begin{itemize}
\item {} 
Lectura y decodificación del archivo

\item {} 
Detección de páginas

\item {} 
Detección de reportes

\item {} 
Compresión y/o encriptación de páginas

\item {} 
Almacenamiento final en Databases «oerm»

\end{itemize}

Básicamente esta herramienta es un «loader» o «cargador» de los reportes a la
base de datos de un sistema Oerm para su posterior uso y consulta.
\begin{quote}

Una ejecución sin parámetros muestra esta ayuda:

\begin{sphinxVerbatim}[commandchars=\\\{\}]
\PYG{n}{uso}\PYG{p}{:} \PYG{n}{spl2oerm} \PYG{p}{[}\PYG{o}{\PYGZhy{}}\PYG{n}{h}\PYG{p}{]} \PYG{p}{[}\PYG{o}{\PYGZhy{}}\PYG{o}{\PYGZhy{}}\PYG{n}{config}\PYG{o}{\PYGZhy{}}\PYG{n}{file} \PYG{n}{CONFIGFILE}\PYG{p}{]} \PYG{n}{inputfile}

\PYG{n}{Procesador} \PYG{n}{de} \PYG{n}{archivo} \PYG{n}{de} \PYG{n}{spool} \PYG{n}{a} \PYG{n}{Oerm} \PYG{p}{(}\PYG{n}{c}\PYG{p}{)} \PYG{l+m+mi}{2016}\PYG{p}{,} \PYG{n}{Patricio} \PYG{n}{Moracho}
\PYG{o}{\PYGZlt{}}\PYG{n}{pmoracho}\PYG{n+nd}{@gmail}\PYG{o}{.}\PYG{n}{com}\PYG{o}{\PYGZgt{}}

\PYG{n}{argumentos} \PYG{n}{posicionales}\PYG{p}{:}
\PYG{n}{inputfile}                                \PYG{n}{Archivo} \PYG{n}{a} \PYG{n}{procesar}

\PYG{n}{argumentos} \PYG{n}{opcionales}\PYG{p}{:}
\PYG{o}{\PYGZhy{}}\PYG{n}{h}\PYG{p}{,} \PYG{o}{\PYGZhy{}}\PYG{o}{\PYGZhy{}}\PYG{n}{help}                               \PYG{n}{mostrar} \PYG{n}{esta} \PYG{n}{ayuda} \PYG{n}{y} \PYG{n}{salir}
\PYG{o}{\PYGZhy{}}\PYG{o}{\PYGZhy{}}\PYG{n}{config}\PYG{o}{\PYGZhy{}}\PYG{n}{file} \PYG{n}{CONFIGFILE}\PYG{p}{,} \PYG{o}{\PYGZhy{}}\PYG{n}{f} \PYG{n}{CONFIGFILE}  \PYG{n}{Archivo} \PYG{n}{de} \PYG{n}{configuración} \PYG{k}{del} \PYG{n}{proceso}\PYG{o}{.}
\end{sphinxVerbatim}

Esto claramente nos dice, que para ejecutar esta herramienta solo requerimos
dos parámetros:
\begin{itemize}
\item {} 
El Archivo «spool» de entrada

\item {} 
Un archivo de configuración del proceso que explicaremos a continuación

\end{itemize}

El proceso en sí requiere una serie de datos para funcionar y llevar a cabo
exitosamente la carga de los reportes. Esta configuración se define en una
archivo de texto escrito en formato \sphinxhref{http://yaml.org/}{yaml}

Un ejemplo sencillo:

\begin{sphinxVerbatim}[commandchars=\\\{\}]
\PYG{c+c1}{\PYGZsh{} vi: ft=yaml}
\PYG{c+c1}{\PYGZsh{} Openerm spool file load config file}
\PYG{c+c1}{\PYGZsh{}}

\PYG{n}{load}\PYG{p}{:}
        \PYG{c+c1}{\PYGZsh{}}
        \PYG{c+c1}{\PYGZsh{}  Definición del archivo de input}
        \PYG{c+c1}{\PYGZsh{}}
        \PYG{n}{inputfile}\PYG{p}{:}
                \PYG{n}{encoding}\PYG{p}{:} \PYG{n}{cp500}             \PYG{c+c1}{\PYGZsh{} Codificación del archivo de entrada (cp500=EBCDIC)}
                \PYG{n}{record}\PYG{o}{\PYGZhy{}}\PYG{n}{length}\PYG{p}{:} \PYG{l+m+mi}{256}          \PYG{c+c1}{\PYGZsh{} Longitud del registro}
                \PYG{n}{file}\PYG{o}{\PYGZhy{}}\PYG{n+nb}{type}\PYG{p}{:} \PYG{n}{fixed}            \PYG{c+c1}{\PYGZsh{} Tipo de input fixed, fcfc}
                \PYG{n}{buffer}\PYG{o}{\PYGZhy{}}\PYG{n}{size}\PYG{p}{:} \PYG{l+m+mi}{102400}         \PYG{c+c1}{\PYGZsh{} Tamaño del buffer de lectura}

        \PYG{c+c1}{\PYGZsh{}}
        \PYG{c+c1}{\PYGZsh{} Definiciones del proceso}
        \PYG{c+c1}{\PYGZsh{}}
        \PYG{n}{process}\PYG{p}{:}
                \PYG{n}{EOP}\PYG{p}{:} \PYG{n}{NEVADO}                 \PYG{c+c1}{\PYGZsh{} Caracter o String que define el salto de página}
                \PYG{n}{report}\PYG{o}{\PYGZhy{}}\PYG{n}{cfg}\PYG{p}{:} \PYG{o}{.}\PYG{o}{/}\PYG{n}{reports}\PYG{o}{.}\PYG{n}{cfg}   \PYG{c+c1}{\PYGZsh{} Archivo de definición de los reportes}

        \PYG{c+c1}{\PYGZsh{}}
        \PYG{c+c1}{\PYGZsh{} Definiciones de la salida}
        \PYG{c+c1}{\PYGZsh{}}
        \PYG{n}{output}\PYG{p}{:}
                \PYG{n}{output}\PYG{o}{\PYGZhy{}}\PYG{n}{path}\PYG{p}{:} \PYG{n}{default}
                \PYG{n}{compress}\PYG{o}{\PYGZhy{}}\PYG{n+nb}{type}\PYG{p}{:} \PYG{l+m+mi}{10}
                \PYG{n}{compress}\PYG{o}{\PYGZhy{}}\PYG{n}{level}\PYG{p}{:} \PYG{l+m+mi}{1}
                \PYG{n}{cipher}\PYG{o}{\PYGZhy{}}\PYG{n+nb}{type}\PYG{p}{:} \PYG{l+m+mi}{0}
                \PYG{n}{pages}\PYG{o}{\PYGZhy{}}\PYG{o+ow}{in}\PYG{o}{\PYGZhy{}}\PYG{n}{group}\PYG{p}{:} \PYG{l+m+mi}{10}
\end{sphinxVerbatim}
\begin{itemize}
\item {} 
Toda línea o texto que comienza con el caracter \sphinxtitleref{\#} es considerada un comentario

\item {} 
La primer sección \sphinxtitleref{load} define completamente el proceso de carga

\item {} \begin{description}
\item[{En \sphinxtitleref{file} se configuran los parámetros para la interpretación básica del archivo}] \leavevmode\begin{itemize}
\item {} 
\sphinxtitleref{encoding} define la codificación del archivo, ver \sphinxhref{https://docs.python.org/3/library/codecs.html\#standard-encodings}{aqui} la lista de psoibles codecs

\item {} 
\sphinxtitleref{record-lenght}, para los tipos de archivo de registros de longitud fija, el tamaño de los mismos

\item {} 
\sphinxtitleref{file-type} tipo de archivo, actualmente hay dos implementaciones \sphinxtitleref{fixed} y \sphinxtitleref{fcfc}

\item {} 
\sphinxtitleref{buffer-size}, tamaño del buffer de lectura, para los archivos de registros de tamño variable.

\end{itemize}

\end{description}

\item {} \begin{description}
\item[{En \sphinxtitleref{process} se configuran los parámetros inherentes al proceso lógico del spool}] \leavevmode\begin{itemize}
\item {} 
\sphinxtitleref{EOP} define el caracter o cadena que determina el cambio de página dentro del reporte.

\item {} 
\sphinxtitleref{report-cfg} define el archivo de configuración de los reportes

\end{itemize}

\end{description}

\item {} \begin{description}
\item[{En \sphinxtitleref{output} se configuran los parámetros que definen el database físico a generar}] \leavevmode\begin{itemize}
\item {} 
\sphinxtitleref{output-path} define el archivo de configuración de los reportes

\item {} 
\sphinxtitleref{compress-type} define el tipo de compresión a usar. Ver más documentación en {\hyperref[\detokenize{openerm.Compressor:module-openerm.Compressor}]{\sphinxcrossref{\sphinxcode{\sphinxupquote{openerm.Compressor}}}}}

\item {} 
\sphinxtitleref{compress-level} define el nivel de compresión a usar. Ver más documentación en {\hyperref[\detokenize{openerm.Compressor:module-openerm.Compressor}]{\sphinxcrossref{\sphinxcode{\sphinxupquote{openerm.Compressor}}}}}

\item {} 
\sphinxtitleref{cipher-type} define el cifrado de los archivos. Ver más documentación en {\hyperref[\detokenize{openerm.Cipher:module-openerm.Cipher}]{\sphinxcrossref{\sphinxcode{\sphinxupquote{openerm.Cipher}}}}}

\item {} 
\sphinxtitleref{pages-in-group} define la cantidad de páginas que manejara el contenedor correspondiente. Ver más documentación en \sphinxcode{\sphinxupquote{openerm.Pagecontainer}}

\end{itemize}

\end{description}

\end{itemize}
\end{quote}
\index{init\_argparse() (en el módulo spl2oerm)@\spxentry{init\_argparse()}\spxextra{en el módulo spl2oerm}}

\begin{fulllineitems}
\phantomsection\label{\detokenize{spl2oerm:spl2oerm.init_argparse}}\pysiglinewithargsret{\sphinxcode{\sphinxupquote{spl2oerm.}}\sphinxbfcode{\sphinxupquote{init\_argparse}}}{}{}
init\_argparse: Inicializar parametros del programa.

\end{fulllineitems}

\index{my\_gettext() (en el módulo spl2oerm)@\spxentry{my\_gettext()}\spxextra{en el módulo spl2oerm}}

\begin{fulllineitems}
\phantomsection\label{\detokenize{spl2oerm:spl2oerm.my_gettext}}\pysiglinewithargsret{\sphinxcode{\sphinxupquote{spl2oerm.}}\sphinxbfcode{\sphinxupquote{my\_gettext}}}{\emph{s}}{}
my\_gettext: Traducir algunas cadenas de argparse.

\end{fulllineitems}

\phantomsection\label{\detokenize{catalogrepo:catalogrepo}}\phantomsection\label{\detokenize{catalogrepo:module-catalogrepo}}\phantomsection\label{\detokenize{catalogrepo:catalogrepo}}\index{catalogrepo (módulo)@\spxentry{catalogrepo}\spxextra{módulo}}

\section{catalogrepo}
\label{\detokenize{catalogrepo:id1}}\label{\detokenize{catalogrepo::doc}}
Esta herramienta realiza la indización o catálogo de los reportes
contenidos en un repositorio \sphinxstylestrong{OpenErm}. Un repositorio no es más
que una carpeta que contiene un conjunto de subcarpetas, y en estas
un \sphinxstylestrong{Database} Oerm
\index{init\_argparse() (en el módulo catalogrepo)@\spxentry{init\_argparse()}\spxextra{en el módulo catalogrepo}}

\begin{fulllineitems}
\phantomsection\label{\detokenize{catalogrepo:catalogrepo.init_argparse}}\pysiglinewithargsret{\sphinxcode{\sphinxupquote{catalogrepo.}}\sphinxbfcode{\sphinxupquote{init\_argparse}}}{}{}
Inicializar parametros del programa.

\end{fulllineitems}

\index{procces\_tree() (en el módulo catalogrepo)@\spxentry{procces\_tree()}\spxextra{en el módulo catalogrepo}}

\begin{fulllineitems}
\phantomsection\label{\detokenize{catalogrepo:catalogrepo.procces_tree}}\pysiglinewithargsret{\sphinxcode{\sphinxupquote{catalogrepo.}}\sphinxbfcode{\sphinxupquote{procces\_tree}}}{\emph{path}, \emph{update=False}}{}
Procesa el path de un repositorio de datbases Oerm
\begin{quote}\begin{description}
\item[{Parámetros}] \leavevmode\begin{itemize}
\item {} 
\sphinxstyleliteralstrong{\sphinxupquote{path}} (\sphinxstyleliteralemphasis{\sphinxupquote{string}}) \textendash{} Carpeta principal del repositorio

\item {} 
\sphinxstyleliteralstrong{\sphinxupquote{update}} (\sphinxstyleliteralemphasis{\sphinxupquote{bool}}) \textendash{} (Opcional) Se actualiza o regenera completamente el catalogo

\end{itemize}

\end{description}\end{quote}

\end{fulllineitems}

\phantomsection\label{\detokenize{checkoermdb:checkoermdb}}\phantomsection\label{\detokenize{checkoermdb:module-checkoermdb}}\phantomsection\label{\detokenize{checkoermdb:checkoermdb}}\index{checkoermdb (módulo)@\spxentry{checkoermdb}\spxextra{módulo}}

\section{checkoermdb}
\label{\detokenize{checkoermdb:id1}}\label{\detokenize{checkoermdb::doc}}
Verificación de databases Openermdb
\index{OermDataBase (clase en checkoermdb)@\spxentry{OermDataBase}\spxextra{clase en checkoermdb}}

\begin{fulllineitems}
\phantomsection\label{\detokenize{checkoermdb:checkoermdb.OermDataBase}}\pysiglinewithargsret{\sphinxbfcode{\sphinxupquote{class }}\sphinxcode{\sphinxupquote{checkoermdb.}}\sphinxbfcode{\sphinxupquote{OermDataBase}}}{\emph{inputfile}}{}
Bases: \sphinxcode{\sphinxupquote{object}}

\end{fulllineitems}

\index{init\_argparse() (en el módulo checkoermdb)@\spxentry{init\_argparse()}\spxextra{en el módulo checkoermdb}}

\begin{fulllineitems}
\phantomsection\label{\detokenize{checkoermdb:checkoermdb.init_argparse}}\pysiglinewithargsret{\sphinxcode{\sphinxupquote{checkoermdb.}}\sphinxbfcode{\sphinxupquote{init\_argparse}}}{}{}
init\_argparse: Inicializar parametros del programa.

\end{fulllineitems}

\index{my\_gettext() (en el módulo checkoermdb)@\spxentry{my\_gettext()}\spxextra{en el módulo checkoermdb}}

\begin{fulllineitems}
\phantomsection\label{\detokenize{checkoermdb:checkoermdb.my_gettext}}\pysiglinewithargsret{\sphinxcode{\sphinxupquote{checkoermdb.}}\sphinxbfcode{\sphinxupquote{my\_gettext}}}{\emph{s}}{}
my\_gettext: Traducir algunas cadenas de argparse.

\end{fulllineitems}



\section{make module}
\label{\detokenize{make:module-make}}\label{\detokenize{make:make-module}}\label{\detokenize{make::doc}}\index{make (módulo)@\spxentry{make}\spxextra{módulo}}
\# Copyright (c) 2014 Patricio Moracho \textless{}\sphinxhref{mailto:pmoracho@gmail.com}{pmoracho@gmail.com}\textgreater{}
\#
\# make.py
\#
\# This program is free software; you can redistribute it and/or
\# modify it under the terms of version 3 of the GNU General Public License
\# as published by the Free Software Foundation. A copy of this license should
\# be included in the file GPL-3.
\#
\# This program is distributed in the hope that it will be useful,
\# but WITHOUT ANY WARRANTY; without even the implied warranty of
\# MERCHANTABILITY or FITNESS FOR A PARTICULAR PURPOSE.  See the
\# GNU Library General Public License for more details.
\#
\# You should have received a copy of the GNU General Public License
\# along with this program; if not, write to the Free Software
\# Foundation, Inc., 59 Temple Place - Suite 330, Boston, MA 02111-1307, USA.
\index{MyMake (clase en make)@\spxentry{MyMake}\spxextra{clase en make}}

\begin{fulllineitems}
\phantomsection\label{\detokenize{make:make.MyMake}}\pysigline{\sphinxbfcode{\sphinxupquote{class }}\sphinxcode{\sphinxupquote{make.}}\sphinxbfcode{\sphinxupquote{MyMake}}}
Bases: \sphinxcode{\sphinxupquote{object}}
\index{build() (método de make.MyMake)@\spxentry{build()}\spxextra{método de make.MyMake}}

\begin{fulllineitems}
\phantomsection\label{\detokenize{make:make.MyMake.build}}\pysiglinewithargsret{\sphinxbfcode{\sphinxupquote{build}}}{\emph{tools}, \emph{msg=True}, \emph{debug=False}}{}
\end{fulllineitems}

\index{check\_packages() (método de make.MyMake)@\spxentry{check\_packages()}\spxextra{método de make.MyMake}}

\begin{fulllineitems}
\phantomsection\label{\detokenize{make:make.MyMake.check_packages}}\pysiglinewithargsret{\sphinxbfcode{\sphinxupquote{check\_packages}}}{\emph{packages}, \emph{msg=True}, \emph{debug=False}}{}
\end{fulllineitems}

\index{check\_tool() (método estático de make.MyMake)@\spxentry{check\_tool()}\spxextra{método estático de make.MyMake}}

\begin{fulllineitems}
\phantomsection\label{\detokenize{make:make.MyMake.check_tool}}\pysiglinewithargsret{\sphinxbfcode{\sphinxupquote{static }}\sphinxbfcode{\sphinxupquote{check\_tool}}}{\emph{tool}, \emph{command}, \emph{find}, \emph{debug=False}}{}
\end{fulllineitems}

\index{check\_virtualenv() (método de make.MyMake)@\spxentry{check\_virtualenv()}\spxextra{método de make.MyMake}}

\begin{fulllineitems}
\phantomsection\label{\detokenize{make:make.MyMake.check_virtualenv}}\pysiglinewithargsret{\sphinxbfcode{\sphinxupquote{check\_virtualenv}}}{\emph{msg=True}, \emph{debug=False}}{}
Verifica la activación del entorno virtual del proyecto, ejecutando el activate.bat
\begin{quote}\begin{description}
\item[{Parámetros}] \leavevmode\begin{itemize}
\item {} 
\sphinxstyleliteralstrong{\sphinxupquote{msg}} (\sphinxstyleliteralemphasis{\sphinxupquote{bool}}) \textendash{} Muestra los mensajes

\item {} 
\sphinxstyleliteralstrong{\sphinxupquote{debug}} \textendash{} (bool)   : Muestra info adicional de la ejecución

\end{itemize}

\item[{Devuelve}] \leavevmode
(bool) True si el entorno ha podido ser activado

\end{description}\end{quote}

\end{fulllineitems}

\index{clean() (método estático de make.MyMake)@\spxentry{clean()}\spxextra{método estático de make.MyMake}}

\begin{fulllineitems}
\phantomsection\label{\detokenize{make:make.MyMake.clean}}\pysiglinewithargsret{\sphinxbfcode{\sphinxupquote{static }}\sphinxbfcode{\sphinxupquote{clean}}}{\emph{pattern}, \emph{debug=False}}{}
\end{fulllineitems}

\index{dev\_install() (método de make.MyMake)@\spxentry{dev\_install()}\spxextra{método de make.MyMake}}

\begin{fulllineitems}
\phantomsection\label{\detokenize{make:make.MyMake.dev_install}}\pysiglinewithargsret{\sphinxbfcode{\sphinxupquote{dev\_install}}}{\emph{msg=True}, \emph{debug=False}}{}
Instalación del entorno de desarrollo en el proyecto
\begin{quote}\begin{description}
\item[{Parámetros}] \leavevmode\begin{itemize}
\item {} 
\sphinxstyleliteralstrong{\sphinxupquote{msg}} (\sphinxstyleliteralemphasis{\sphinxupquote{bool}}) \textendash{} Muestra los mensajes

\item {} 
\sphinxstyleliteralstrong{\sphinxupquote{debug}} \textendash{} (bool)   : Muestra info adicional de la ejecución

\end{itemize}

\end{description}\end{quote}

\end{fulllineitems}

\index{devcheck() (método de make.MyMake)@\spxentry{devcheck()}\spxextra{método de make.MyMake}}

\begin{fulllineitems}
\phantomsection\label{\detokenize{make:make.MyMake.devcheck}}\pysiglinewithargsret{\sphinxbfcode{\sphinxupquote{devcheck}}}{}{}
\end{fulllineitems}

\index{devinstall() (método de make.MyMake)@\spxentry{devinstall()}\spxextra{método de make.MyMake}}

\begin{fulllineitems}
\phantomsection\label{\detokenize{make:make.MyMake.devinstall}}\pysiglinewithargsret{\sphinxbfcode{\sphinxupquote{devinstall}}}{}{}
\end{fulllineitems}

\index{doc\_install() (método de make.MyMake)@\spxentry{doc\_install()}\spxextra{método de make.MyMake}}

\begin{fulllineitems}
\phantomsection\label{\detokenize{make:make.MyMake.doc_install}}\pysiglinewithargsret{\sphinxbfcode{\sphinxupquote{doc\_install}}}{\emph{msg=True}, \emph{debug=False}}{}
Instalación del entorno de documetación Sphinx
\begin{quote}\begin{description}
\item[{Parámetros}] \leavevmode\begin{itemize}
\item {} 
\sphinxstyleliteralstrong{\sphinxupquote{msg}} (\sphinxstyleliteralemphasis{\sphinxupquote{bool}}) \textendash{} Muestra los mensajes

\item {} 
\sphinxstyleliteralstrong{\sphinxupquote{debug}} \textendash{} (bool)   : Muestra info adicional de la ejecución

\end{itemize}

\end{description}\end{quote}

\end{fulllineitems}

\index{docinstall() (método de make.MyMake)@\spxentry{docinstall()}\spxextra{método de make.MyMake}}

\begin{fulllineitems}
\phantomsection\label{\detokenize{make:make.MyMake.docinstall}}\pysiglinewithargsret{\sphinxbfcode{\sphinxupquote{docinstall}}}{}{}
\end{fulllineitems}

\index{print\_error() (método de make.MyMake)@\spxentry{print\_error()}\spxextra{método de make.MyMake}}

\begin{fulllineitems}
\phantomsection\label{\detokenize{make:make.MyMake.print_error}}\pysiglinewithargsret{\sphinxbfcode{\sphinxupquote{print\_error}}}{}{}
\end{fulllineitems}

\index{run\_devcheck() (método de make.MyMake)@\spxentry{run\_devcheck()}\spxextra{método de make.MyMake}}

\begin{fulllineitems}
\phantomsection\label{\detokenize{make:make.MyMake.run_devcheck}}\pysiglinewithargsret{\sphinxbfcode{\sphinxupquote{run\_devcheck}}}{\emph{msg=True}, \emph{debug=False}}{}
\end{fulllineitems}

\index{run\_tests() (método de make.MyMake)@\spxentry{run\_tests()}\spxextra{método de make.MyMake}}

\begin{fulllineitems}
\phantomsection\label{\detokenize{make:make.MyMake.run_tests}}\pysiglinewithargsret{\sphinxbfcode{\sphinxupquote{run\_tests}}}{\emph{msg=True}, \emph{debug=False}}{}
\end{fulllineitems}

\index{test() (método de make.MyMake)@\spxentry{test()}\spxextra{método de make.MyMake}}

\begin{fulllineitems}
\phantomsection\label{\detokenize{make:make.MyMake.test}}\pysiglinewithargsret{\sphinxbfcode{\sphinxupquote{test}}}{}{}
\end{fulllineitems}

\index{tools() (método de make.MyMake)@\spxentry{tools()}\spxextra{método de make.MyMake}}

\begin{fulllineitems}
\phantomsection\label{\detokenize{make:make.MyMake.tools}}\pysiglinewithargsret{\sphinxbfcode{\sphinxupquote{tools}}}{}{}
\end{fulllineitems}


\end{fulllineitems}

\index{my\_gettext() (en el módulo make)@\spxentry{my\_gettext()}\spxextra{en el módulo make}}

\begin{fulllineitems}
\phantomsection\label{\detokenize{make:make.my_gettext}}\pysiglinewithargsret{\sphinxcode{\sphinxupquote{make.}}\sphinxbfcode{\sphinxupquote{my\_gettext}}}{\emph{s}}{}
my\_gettext: Traducir algunas cadenas de argparse.

\end{fulllineitems}

\index{subprocess\_cmd() (en el módulo make)@\spxentry{subprocess\_cmd()}\spxextra{en el módulo make}}

\begin{fulllineitems}
\phantomsection\label{\detokenize{make:make.subprocess_cmd}}\pysiglinewithargsret{\sphinxcode{\sphinxupquote{make.}}\sphinxbfcode{\sphinxupquote{subprocess\_cmd}}}{\emph{command}}{}
\end{fulllineitems}



\section{oerm\_hostreprint\_processor module}
\label{\detokenize{oerm_hostreprint_processor:module-oerm_hostreprint_processor}}\label{\detokenize{oerm_hostreprint_processor:oerm-hostreprint-processor-module}}\label{\detokenize{oerm_hostreprint_processor::doc}}\index{oerm\_hostreprint\_processor (módulo)@\spxentry{oerm\_hostreprint\_processor}\spxextra{módulo}}
\# Copyright (c) 2014 Patricio Moracho \textless{}\sphinxhref{mailto:pmoracho@gmail.com}{pmoracho@gmail.com}\textgreater{}
\#
\# oerm\_hostreprint\_processor
\#
\# This program is free software; you can redistribute it and/or
\# modify it under the terms of version 3 of the GNU General Public License
\# as published by the Free Software Foundation. A copy of this license should
\# be included in the file GPL-3.
\#
\# This program is distributed in the hope that it will be useful,
\# but WITHOUT ANY WARRANTY; without even the implied warranty of
\# MERCHANTABILITY or FITNESS FOR A PARTICULAR PURPOSE.  See the
\# GNU Library General Public License for more details.
\#
\# You should have received a copy of the GNU General Public License
\# along with this program; if not, write to the Free Software
\# Foundation, Inc., 59 Temple Place - Suite 330, Boston, MA 02111-1307, USA.
\index{Main() (en el módulo oerm\_hostreprint\_processor)@\spxentry{Main()}\spxextra{en el módulo oerm\_hostreprint\_processor}}

\begin{fulllineitems}
\phantomsection\label{\detokenize{oerm_hostreprint_processor:oerm_hostreprint_processor.Main}}\pysiglinewithargsret{\sphinxcode{\sphinxupquote{oerm\_hostreprint\_processor.}}\sphinxbfcode{\sphinxupquote{Main}}}{}{}
\end{fulllineitems}

\index{init\_argparse() (en el módulo oerm\_hostreprint\_processor)@\spxentry{init\_argparse()}\spxextra{en el módulo oerm\_hostreprint\_processor}}

\begin{fulllineitems}
\phantomsection\label{\detokenize{oerm_hostreprint_processor:oerm_hostreprint_processor.init_argparse}}\pysiglinewithargsret{\sphinxcode{\sphinxupquote{oerm\_hostreprint\_processor.}}\sphinxbfcode{\sphinxupquote{init\_argparse}}}{}{}
init\_argparse: Inicializar parametros del programa.

\end{fulllineitems}

\index{my\_gettext() (en el módulo oerm\_hostreprint\_processor)@\spxentry{my\_gettext()}\spxextra{en el módulo oerm\_hostreprint\_processor}}

\begin{fulllineitems}
\phantomsection\label{\detokenize{oerm_hostreprint_processor:oerm_hostreprint_processor.my_gettext}}\pysiglinewithargsret{\sphinxcode{\sphinxupquote{oerm\_hostreprint\_processor.}}\sphinxbfcode{\sphinxupquote{my\_gettext}}}{\emph{s}}{}
my\_gettext: Traducir algunas cadenas de argparse.

\end{fulllineitems}

\index{process\_file() (en el módulo oerm\_hostreprint\_processor)@\spxentry{process\_file()}\spxextra{en el módulo oerm\_hostreprint\_processor}}

\begin{fulllineitems}
\phantomsection\label{\detokenize{oerm_hostreprint_processor:oerm_hostreprint_processor.process_file}}\pysiglinewithargsret{\sphinxcode{\sphinxupquote{oerm\_hostreprint\_processor.}}\sphinxbfcode{\sphinxupquote{process\_file}}}{\emph{configfile}, \emph{inputfile}, \emph{outputfile}, \emph{compressiontype}, \emph{complevel}, \emph{ciphertype}, \emph{testall}, \emph{append}, \emph{pagesingroups}}{}
\end{fulllineitems}

\phantomsection\label{\detokenize{readoermdb:readoermdb}}\phantomsection\label{\detokenize{readoermdb:module-readoermdb}}\phantomsection\label{\detokenize{readoermdb:readoermdb}}\index{readoermdb (módulo)@\spxentry{readoermdb}\spxextra{módulo}}

\section{Readoermdb}
\label{\detokenize{readoermdb:id1}}\label{\detokenize{readoermdb::doc}}\index{init\_argparse() (en el módulo readoermdb)@\spxentry{init\_argparse()}\spxextra{en el módulo readoermdb}}

\begin{fulllineitems}
\phantomsection\label{\detokenize{readoermdb:readoermdb.init_argparse}}\pysiglinewithargsret{\sphinxcode{\sphinxupquote{readoermdb.}}\sphinxbfcode{\sphinxupquote{init\_argparse}}}{}{}
init\_argparse: Inicializar parametros del programa.

\end{fulllineitems}

\index{my\_gettext() (en el módulo readoermdb)@\spxentry{my\_gettext()}\spxextra{en el módulo readoermdb}}

\begin{fulllineitems}
\phantomsection\label{\detokenize{readoermdb:readoermdb.my_gettext}}\pysiglinewithargsret{\sphinxcode{\sphinxupquote{readoermdb.}}\sphinxbfcode{\sphinxupquote{my\_gettext}}}{\emph{s}}{}
my\_gettext: Traducir algunas cadenas de argparse.

\end{fulllineitems}



\chapter{Desarrollo}
\label{\detokenize{index:desarrollo}}
Algunos puntos importantes para el desarrollador


\section{Requisitos iniciales}
\label{\detokenize{desarrollo:requisitos-iniciales}}\label{\detokenize{desarrollo::doc}}
El proyecto \sphinxstylestrong{Openerm} esta construido usando el lenguaje \sphinxstylestrong{python} y varios «packages» o librerías adicionales para
dicho lenguaje. Para poder construir las herramientas del proyecto es necesario preparar antes que nada, un entorno
de desarrollo. A continuación expondremos en detalle cuales son los pasos para tener preparado el entorno de desarrollo.
Este detalle esta orientado a la implementación sobre Windows 32 bits, los pasos para versiones de 64 bits son
sustancialmente distintos, en particular por algunos de los «paquetes» que se construyen a partir de módulos en C o C++,
de igual forma la instalación sobre Linux tiene sus grandes diferencias. Eventualmente profundizaremos sobre estos
entornos, pero en principo volvemos a señalar que el siguiente detalle aplica a los ambientes Windows de 32 bits
\begin{itemize}
\item {} 
Obviamente en primer lugar necesitaremos \sphinxhref{https://www.python.org/ftp/python/3.4.0/python-3.4.0.msi}{Python},
por ahora únicamente la versión 3.4. La correcta instalación se debe verificar desde la línea de comandos:
\sphinxcode{\sphinxupquote{python -{-}version}}. Si todo se instaló correctamente se debe ver algo como esto \sphinxcode{\sphinxupquote{Python 3.4.0}}, sino verificar
que Python.exe se encuentre correctamente apuntado en el PATH.

\item {} 
Es conveniente pero no mandatorio hacer upgrade de la herramienta pip: \sphinxcode{\sphinxupquote{python -m pip install -{-}upgrade pip}}

\item {} 
\sphinxhref{https://virtualenv.pypa.io/en/stable/}{Virutalenv:}. Es la herramienta estándar para crear entornos «aislados»
de python. En nuestro ejemplo \sphinxstylestrong{Openerm}, requiere de Ptython 3.4 y de varios «paquetes» adcionales de
versiones especifícas. Para no tener conflictos de desarrollo lo que haremos mediante esta herramienta es
crear un «entorno virtual» de python en una carpeta del projecto (venv), dónde una vez «activado» dicho
entorno podremos instalarle los paquetes que requiere el proyecto. Este «entorno virtual» contendrá una
copia completa de Python y los paquetes mencionados, al activarlo se modifica el PATH al python.exe que
apuntará ahora a nuestra carpeta del entorno y nuestras propias librerías, evitando cualquier tipo de
conflicto con un entorno Python ya existente. La instalación de virtualenv se hara mediante \sphinxcode{\sphinxupquote{pip install virtualenv}}

\item {} 
Descargar el proyecto desde \sphinxhref{https://bitbucket.org/pmoracho/openerm}{Bitbucket}, se puede descargar
desde la página el proyecto como un archivo Zip, o si contamos con \sphinxhref{https://git-for-windows.github.io/}{Git}
sencillamente haremos un \sphinxcode{\sphinxupquote{git clone https://pmoracho@bitbucket.org/pmoracho/openerm.git}}.

\item {} 
El proyecto una vez descomprimido o luego del clonado del repositorio tendrá una estructura de directorios
similar a la siguiente::

\begin{sphinxVerbatim}[commandchars=\\\{\}]
\PYG{n}{openerm}
\PYG{o}{\textbar{}}\PYG{o}{\PYGZhy{}}\PYG{n}{build}
\PYG{o}{\textbar{}}\PYG{o}{\PYGZhy{}}\PYG{n}{dist}
\PYG{o}{\textbar{}}\PYG{o}{\PYGZhy{}}\PYG{n}{doc}
\PYG{o}{\textbar{}}\PYG{o}{\PYGZhy{}}\PYG{n}{openerm}
\PYG{o}{\textbar{}}\PYG{o}{\PYGZhy{}}\PYG{n}{tests}
\PYG{o}{\textbar{}}\PYG{o}{\PYGZhy{}}\PYG{n}{tools}
\PYG{o}{\textbar{}}\PYG{o}{\PYGZhy{}}\PYG{n}{var}
\PYG{o}{\textbar{}}\PYG{o}{\PYGZhy{}}\PYG{n}{wheels}
\end{sphinxVerbatim}

\end{itemize}


\section{Preparación del entorno virtual local}
\label{\detokenize{desarrollo:preparacion-del-entorno-virtual-local}}
Para poder ejecutar, o crear la distribución de la herramientas, lo primero que deberemos hacer es armar un entorno
python «virtual» que alojaremos en una subcarpeta del directorio principal que llamarems «venv». En el proyecto incorporamos
una herramienta de automatización de algunas tareas básicas. Se trata de \sphinxcode{\sphinxupquote{make.py}}, la forma de ejecutarlo es la siguiente:
\sphinxcode{\sphinxupquote{python tools\textbackslash{}make.py}} la ejecución si parámetros arrojará una salida como lo que sigue::

\begin{sphinxVerbatim}[commandchars=\\\{\}]
\PYG{n}{Automatización} \PYG{n}{de} \PYG{n}{tareas} \PYG{n}{para} \PYG{n}{el} \PYG{n}{proyecto} \PYG{n}{Openerm}
\PYG{p}{(}\PYG{n}{c}\PYG{p}{)} \PYG{l+m+mi}{2016}\PYG{p}{,} \PYG{n}{Patricio} \PYG{n}{Moracho} \PYG{o}{\PYGZlt{}}\PYG{n}{pmoracho}\PYG{n+nd}{@gmail}\PYG{o}{.}\PYG{n}{com}\PYG{o}{\PYGZgt{}}

\PYG{n}{Uso}\PYG{p}{:} \PYG{n}{make} \PYG{o}{\PYGZlt{}}\PYG{n}{command}\PYG{o}{\PYGZgt{}} \PYG{p}{[}\PYG{o}{\PYGZlt{}}\PYG{n}{args}\PYG{o}{\PYGZgt{}}\PYG{p}{]}

\PYG{n}{Los} \PYG{n}{comandos} \PYG{n}{más} \PYG{n}{usados}\PYG{p}{:}
\PYG{n}{devcheck}   \PYG{n}{Hace} \PYG{n}{una} \PYG{n}{verificación} \PYG{k}{del} \PYG{n}{entorno} \PYG{n}{de} \PYG{n}{desarrollo}
\PYG{n}{devinstall} \PYG{n}{Realiza} \PYG{n}{la} \PYG{n}{instalación} \PYG{k}{del} \PYG{n}{entorno} \PYG{n}{de} \PYG{n}{desarrollo} \PYG{n}{virtual} \PYG{n}{e} \PYG{n}{instala} \PYG{n}{los} \PYG{n}{requerimientos}
\PYG{n}{clear}      \PYG{n}{Elimina} \PYG{n}{archivos} \PYG{n}{innecesarios}
\PYG{n}{test}       \PYG{n}{Ejecuta} \PYG{n}{todos} \PYG{n}{los} \PYG{n}{tests} \PYG{n}{definidos} \PYG{k}{del} \PYG{n}{proyecto}
\PYG{n}{tools}      \PYG{n}{Construye} \PYG{n}{la} \PYG{n}{distribución} \PYG{n}{binaria} \PYG{n}{de} \PYG{n}{las} \PYG{n}{herramientas} \PYG{k}{del} \PYG{n}{proyecto}

\PYG{n}{make}\PYG{o}{.}\PYG{n}{py}\PYG{p}{:} \PYG{n}{error}\PYG{p}{:} \PYG{n}{los} \PYG{n}{siguientes} \PYG{n}{argumentos} \PYG{n}{son} \PYG{n}{requeridos}\PYG{p}{:} \PYG{n}{command}
\end{sphinxVerbatim}

Para preparar el entorno virtual simplemente haremos \sphinxcode{\sphinxupquote{python tools\textbackslash{}make.py devinstall}}, este proceso si resulta exitoso deberá
haber realizado las siguientes tareas:
\begin{itemize}
\item {} 
Creación de un entorno pyhton virtual en la carpeta «venv», invocable mediante \sphinxcode{\sphinxupquote{venv\textbackslash{}Scripts\textbackslash{}activate.bat}} en Windows
o \sphinxcode{\sphinxupquote{source venv/Scripts/activate}} en entornos Linux o Cygwin/Mingw (en Windows)

\item {} 
Instalado todas las dependencias necesarias

\end{itemize}


\section{Notas:}
\label{\detokenize{desarrollo:notas}}\begin{itemize}
\item {} 
Hay dependecias que son fácilmente instalables mediante el comando \sphinxcode{\sphinxupquote{pip}} y otras que no se instalan de la misma forma
tan fácilmente. Estás últimas son librerías o proyectos en C o C++ que requieren de la compilación de distintos módulos,
estos «paquetes», para poder instalarse mediante \sphinxcode{\sphinxupquote{pip}} requieren que dispongamos de un compilador C/C++, algo que no
siempre ocurre e incluso para ser más exactos, deberíamos contar con la misma versión del compilador que usa nuestra
distribución python. Por esto hemos optado por incluir los paquetes ya compilados en su distribución binaria. Los
requerimientos de este tipo podrán ser encontrados en la carpeta wheels.

\item {} 
Es recomendable y cómodo, pero entiendo que no es mandatorio, contar con un entorno estilo «Linux»,
por ejemplo \sphinxhref{http://www.mingw.org/}{MinGW}, tal como dice la página del proyecto: «MinGW provides a complete Open Source
programming tool set which is suitable for the development of native MS-Windows applications, and which do not depend on
any 3rd-party C-Runtime DLLs. (It does depend on a number of DLLs provided by Microsoft themselves, as components of the
operating system; most notable among these is MSVCRT.DLL, the Microsoft C runtime library. Additionally, threaded
applications must ship with a freely distributable thread support DLL, provided as part of MinGW itself).» De este
entorno requerimos algunas herramientas de desarrollo: Bash para la línea de comandos y Make para la automatización de
varias tareas del proyecto.

\end{itemize}


\section{Otras consideraciones}
\label{\detokenize{desarrollo:otras-consideraciones}}\begin{itemize}
\item {} 
Usar «soft tabs»: Con cualquier editor que usemos configurar el uso del \textless{}tab\textgreater{} en vez de los espacios, yo prefiero el \textless{}tab\textgreater{}
puro al espacio, entiendo que es válido el otro criterio pero ya los fuentes están con esta configuración, por lo que para
evitar problemas al compilar los .py recomiendo seguir usando este criterio. Asimismo configurar en 4 posiciones estos \textless{}tab\textgreater{}.

\end{itemize}


\chapter{Indices y tablas}
\label{\detokenize{index:indices-y-tablas}}\begin{itemize}
\item {} 
\DUrole{xref,std,std-ref}{genindex}

\item {} 
\DUrole{xref,std,std-ref}{modindex}

\item {} 
\DUrole{xref,std,std-ref}{search}

\end{itemize}


\renewcommand{\indexname}{Índice de Módulos Python}
\begin{sphinxtheindex}
\let\bigletter\sphinxstyleindexlettergroup
\bigletter{c}
\item\relax\sphinxstyleindexentry{catalogrepo}\sphinxstyleindexpageref{catalogrepo:\detokenize{module-catalogrepo}}
\item\relax\sphinxstyleindexentry{checkoermdb}\sphinxstyleindexpageref{checkoermdb:\detokenize{module-checkoermdb}}
\indexspace
\bigletter{m}
\item\relax\sphinxstyleindexentry{make}\sphinxstyleindexpageref{make:\detokenize{module-make}}
\indexspace
\bigletter{o}
\item\relax\sphinxstyleindexentry{oerm\_hostreprint\_processor}\sphinxstyleindexpageref{oerm_hostreprint_processor:\detokenize{module-oerm_hostreprint_processor}}
\item\relax\sphinxstyleindexentry{openerm.Block}\sphinxstyleindexpageref{openerm.Block:\detokenize{module-openerm.Block}}
\item\relax\sphinxstyleindexentry{openerm.Cipher}\sphinxstyleindexpageref{openerm.Cipher:\detokenize{module-openerm.Cipher}}
\item\relax\sphinxstyleindexentry{openerm.Compressor}\sphinxstyleindexpageref{openerm.Compressor:\detokenize{module-openerm.Compressor}}
\item\relax\sphinxstyleindexentry{openerm.Config}\sphinxstyleindexpageref{openerm.Config:\detokenize{module-openerm.Config}}
\item\relax\sphinxstyleindexentry{openerm.Database}\sphinxstyleindexpageref{openerm.Database:\detokenize{module-openerm.Database}}
\item\relax\sphinxstyleindexentry{openerm.Index}\sphinxstyleindexpageref{openerm.Index:\detokenize{module-openerm.Index}}
\item\relax\sphinxstyleindexentry{openerm.MetadataContainer}\sphinxstyleindexpageref{openerm.MetadataContainer:\detokenize{module-openerm.MetadataContainer}}
\item\relax\sphinxstyleindexentry{openerm.OermClient}\sphinxstyleindexpageref{openerm.OermClient:\detokenize{module-openerm.OermClient}}
\item\relax\sphinxstyleindexentry{openerm.PageContainer}\sphinxstyleindexpageref{openerm.PageContainer:\detokenize{module-openerm.PageContainer}}
\item\relax\sphinxstyleindexentry{openerm.Pages}\sphinxstyleindexpageref{openerm.Pages:\detokenize{module-openerm.Pages}}
\item\relax\sphinxstyleindexentry{openerm.Report}\sphinxstyleindexpageref{openerm.Report:\detokenize{module-openerm.Report}}
\item\relax\sphinxstyleindexentry{openerm.ReportMatcher}\sphinxstyleindexpageref{openerm.ReportMatcher:\detokenize{module-openerm.ReportMatcher}}
\item\relax\sphinxstyleindexentry{openerm.Reports}\sphinxstyleindexpageref{openerm.Reports:\detokenize{module-openerm.Reports}}
\item\relax\sphinxstyleindexentry{openerm.SpoolFixedRecordLength}\sphinxstyleindexpageref{openerm.SpoolFixedRecordLength:\detokenize{module-openerm.SpoolFixedRecordLength}}
\item\relax\sphinxstyleindexentry{openerm.SpoolHostReprint}\sphinxstyleindexpageref{openerm.SpoolHostReprint:\detokenize{module-openerm.SpoolHostReprint}}
\item\relax\sphinxstyleindexentry{openerm.Utils}\sphinxstyleindexpageref{openerm.Utils:\detokenize{module-openerm.Utils}}
\indexspace
\bigletter{r}
\item\relax\sphinxstyleindexentry{readoermdb}\sphinxstyleindexpageref{readoermdb:\detokenize{module-readoermdb}}
\indexspace
\bigletter{s}
\item\relax\sphinxstyleindexentry{spl2oerm}\sphinxstyleindexpageref{spl2oerm:\detokenize{module-spl2oerm}}
\item\relax\sphinxstyleindexentry{splprocessor}\sphinxstyleindexpageref{splprocessor:\detokenize{module-splprocessor}}
\end{sphinxtheindex}

\renewcommand{\indexname}{Índice}
\printindex
\end{document}